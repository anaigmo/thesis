\chapter{Introduction}
\label{chapter:intro}


* General context about knowledge graphs, they are everywhere, representing every kind of information, used for many different purposes. 

* General context about how to create KGs: code libraries, tools like openrefine, and mappings. Then on and on with mappings, how they have evolved blabla, increasing support by different engines blabla

* Mappping-copmliant technologies are increasingly proving to be convenient/useful/good for KGs: maintainability, understandability, engines already optimized, no need to code (accessibility for wider range of profiles). 

However, data is still complex, the requirements for KG evolve and these languages are sometimes limited. **Then first reference to a whole: gather which are these necessities**

Then about how to write mappings, still hard for many people who find it difficult or blabla, how to increase the adoption by making the user experience better: visual approaches do not scale and blabla. user friendly serializations are popular, we keep on updating them with needs already identified (above), and develop a new approach that can be for bot domain experts and experienced practitioners.

\ana{Por aquí hay que leer y meter cosas de KG life cycle} teniendo en cuenta que los mappings están en construcción, se podría plantear cómo pueden ayudar en el mantenimiento y evolución de KGs, son suficientemente robustos los current approaches para este purpose?

\section{Thesis structure}

The remainder of this thesis is structured as follows:

\begin{itemize}
    \item \cref{chapter:sota}
    \item \cref{chapter:objectives}
    \item \cref{chapter:mappings}
    \item \cref{chapter:creation}
    \item \cref{chapter:evolution}
    \item \cref{chapter:conclusions}
\end{itemize}


\section{Related publications}



