\chapter{Introduction}
\label{chapter:intro}

%\textit{intro sobre la web, cómo surgió web semantica?}


%\textit{* General context about knowledge graphs, they are everywhere, representing every kind of information, used for many different purposes. }

%\ana{empezar introduciendo RDF y semtech a más alto nivel antes de pasar a KGs}

The nature of the World Wide Web has made it possible for every user and organization across the world to publish and access data on a global scale. 
The size of the data available on the Web has increased hand-in-hand with its heterogeneity. 
These data are often provided in different formats (e.g., plain text, CSV or JSON) and with diverse entry points, (e.g., data portals, APIs, or databases). 
This diversity makes it increasingly difficult to enable large-scale data access and processing for both humans and machines.
The Semantic Web~\parencite{berners2001semantic} was envisioned to represent information as \textit{Linked Data} or \textit{Knowledge Graphs} that can be automatically exploited, becoming a remarkably active research field in the last decades.
Multiple recommendations have been released from the World Wide Web Consortium (W3C) for modelling data in a uniform model (i.e., RDF~\parencite{rdf}), querying (i.e., SPARQL~\parencite{harris2013sparql}), constructing from other data sources (i.e., R2RML~\parencite{das2012r2rml}) and validating against a set of restrictions (i.e., SHACL~\parencite{SHACL}) among others. 

%Knowledge graphs (KGs) can be defined as ``graphs of data intended to accumulate and convey knowledge of the real world, whose nodes represent entities of interest and whose edges represent relations between these entities"~\parencite{hogan2021kg}. 
Knowledge graphs (KGs) integrate and convey real-world knowledge, representing entities and how they relate to each other.
KGs are usually compliant with the W3C recommendations, and adequate for downstream consumption applications~\parencite{hogan2021kg}.
%Over the last few decades, knowledge graphs have gained momentum, enabling the community to structure, exploit and publish data on the web. 
There are countless success stories of knowledge graphs being currently used, from publicly available and representing common knowledge or domain-specific knowledge, to privately owned graphs. 
Largely known and widely used Open Knowledge Graphs include examples containing common knowledge, such as Wikidata~\parencite{vrandevcic2014wikidata}, DBPedia~\parencite{auer2007dbpedia}, YAGO~\parencite{pellissier2020yago} or Freebase~\parencite{bollacker2007freebase}; 
%, EncycNet\footnote{\url{https://encycnet.github.io/}}, 
and about a specific domain, such as 
libraries~\parencite{vila2013datos}, 
scientific articles~\parencite{stocker2023orkg,farber2023semopenalex}, 
tourism~\parencite{karle2018building}, 
%geography~\parencite{stadler2012linkedgeodata}, 
cultural heritage~\parencite{carriero2019arco}, 
life sciences~\parencite{dumontier2014bio2rdf,pinero2020disgenet},
railway infrastructure~\parencite{rojas2021leveraging}, among many others. 
Not only within the scientific community, but also companies realized their potential and are taking advantage of them. 
Therefore, companies such as
Google\footnote{\url{https://blog.google/products/search/introducing-knowledge-graph-things-not/}},
Microsoft~\parencite{farber2019microsoft},
LinkedIn\footnote{\url{https://engineering.linkedin.com/blog/2016/10/building-the-linkedin-knowledge-graph}}, 
Pinterest~\parencite{goncalves2019pinterest},
produce their own knowledge graphs. 

%\textit{* General context about how to create KGs: code libraries, tools like openrefine, and mappings. Then on and on with mappings, how they have evolved blabla, increasing support by different engines blabla}

%\textcolor{red}{The adoption of KGs comes hand in hand with the ease of constructing them by the community~\parencite{hogan2020twodecades,karger2014semantic}. }
There is currently a high variety of ways for constructing knowledge graphs: requiring a different level of manual effort, suitable for diverse data sizes and formats, from (semi)structured to unstructured. %\ana{extender cada approach en un párrafo}
Even within the systematic construction from semi-structured data~\parencite{Poggi2008}, there are likewise multiple manners to carry out the task, using (i)~programming languages, (ii)~visual interactive interfaces, and (iii)~declarative mapping rules. 

The use of \textit{programming languages} for KG construction is a common practice. 
For instance, libraries such as RDFLib\footnote{\url{https://rdflib.readthedocs.io/en/stable/}} and Pyoxigraph\footnote{\url{https://pyoxigraph.readthedocs.io/en/latest/}} for Python or Jena\footnote{\url{https://jena.apache.org/}} for Java enable the source data processing and extraction of relationships to construct KGs. 
This method brings benefits regarding flexibility to address data heterogeneity, processing and generation. 
However, it supposes an ad-hoc approach restricted for users with specific technical background, and can hinder the maintainability, understandability and reusability of the construction resources~\parencite{iglesias2019bio2rdf}.

However, constructing KGs should not only be restricted to practitioners with technical skills~\parencite{karger2014semantic}. 
As a result, different \textit{visual interactive interfaces} and methods were developed to facilitate the KG construction process for a wider range of users.
For instance, OpenRefine\footnote{\url{https://openrefine.org/}} comprises a framework to upload tabular data and manually edit them, including the capability to generate RDF datasets. \url{Semantify.it}~\parencite{karle2017semantifyit} provides a user-friendly editor to manually annotate data to create structured content with \url{Schema.org}~\parencite{guha2016schema.org} for webpages. 
One setback of this kind of editors is that they cannot usually manage large data sizes\footnote{\url{https://github.com/OpenRefine/openrefine.org/issues/136}}~\parencite{petrova2020data}. 
In addition, similarly to the \textit{ad-hoc programming} approach, providing manual environments for performing the transformations runs against the reproducibility, maintainability and automation of the KG construction process. 
%Despite this fact, this option usually results the most suitable for new users or practitioners with non-technical backgrounds, as they are designed to be intuitive and easy to use. \ana{buscar cita}

%\textit{* Mappping-compliant technologies are increasingly proving to be convenient/useful/good for KGs: maintainability, understandability, engines already optimized, no need to code (accessibility for wider range of profiles). }

A compromise between these two approaches emerged with \textit{declarative mapping rules}. 
These mappings define the rules that hold of the correspondences between source data and a target ontology to create a knowledge graph. 
The main milestone in the progress of these technologies came with the standardization of the RDB2RDF Mapping Language (R2RML) in 2012~\parencite{das2012r2rml}. 
This language, based on previous efforts such as XLWrap~\parencite{xlwrap}, D2RQ~\parencite{bizer2004d2rq} or R2O~\parencite{barrasa2004r2o}, focused on describing the transformation of data in relational databases (RDBs) into RDF. 
Since R2RML started to be adopted and used in real-world scenarios, its limitations became evident. 
Starting from the limited input data format it is able to describe (RDBs), the first extensions appeared to extend the scope for more heterogeneous data formats (e.g. RML~\parencite{Dimou2014rml}, xR2RML~\parencite{michel2015xr2rml}). 
R2RML relies on SQL queries and views for performing data transformation functions over the original data, a feature that cannot be applied to other data formats. 
For this reason, other extensions appeared to allow data transformation description in the language (e.g. FunUL~\parencite{junior2016funul}, KR2RML~\parencite{slepicka2015kr2rml}). 
Likewise, with new needs and requirements, additional extensions and languages were developed subsequently (e.g. RML-Target~\parencite{VanAssche2021LeveragingWebThings}, 
SPARQL-Generate~\parencite{Lefrancois2017sparqlgenerate}, 
%Helio~\parencite{cimmino2022helio}, 
ShExML~\parencite{Garcia-Gonzalez2020shexml}, 
SPARQL-Anything~\parencite{asprino2023sparql-anything}).
The use of mapping languages over this decade has proven to be a suitable approach for semi-automated KG construction, maintainable in the long term and scalable for large data sizes~\parencite{vidal2023knowledge,iglesias2023scaling,xiao2020virtual,iglesias2022empowering}.  
%\textcolor{red}{Although it was increasingly adopted and used in real-world scenarios, its limitations became evident.} 
%This triggered the release of a considerable amount of extensions (e.g. RML~\parencite{Dimou2014rml}, xR2RML~\parencite{michel2015xr2rml}, FunUL~\parencite{junior2016funul}, KR2RML~\parencite{slepicka2015kr2rml})
%and new languages (e.g., %Tarql\footnote{\url{https://tarql.github.io/}}, 
%SPARQL-Generate~\parencite{Lefrancois2017sparqlgenerate}, 
%Helio~\parencite{cimmino2022helio}, 
%ShExML~\parencite{Garcia-Gonzalez2020shexml}, 
%SPARQL-Anything~\parencite{asprino2023sparql-anything}), 
%capable of addressing a wider heterogeneity of data and use cases with multiple additional features. 
%The use of mapping languages over this decade has proven to be a suitable approach for semi-automated KG construction, maintainable in the long term and scalable for large data sizes~\parencite{vidal2023knowledge,iglesias2023scaling,xiao2020virtual,iglesias2022empowering}.  %with a vibrant community of users actively supporting it\footnote{\url{https://www.w3.org/community/kg-construct/}}. 

%\textit{However, data is still complex, the requirements for KG evolve and these languages are sometimes limited. **Then first reference to a whole: gather which are these necessities**}

 Hence, mapping languages provide a means (usually a vocabulary or syntax) to describe these transformation rules declaratively in a file. 
 All languages share a common core of characteristics: (i) the description of the input data, and (ii) the specification on how to generate the output triples in the graph according to the schema of an ontology. 
 The extent, level of detail and variability on how to perform this task is related to the heterogeneity of the proposed languages. 
 They were developed taking into account different needs and use cases, resulting in diverse features, therefore providing a wide set of possibilities for users.
 Knowing the capabilities of each language can help users decide which one to use depending on their needs. 
 Moreover, despite the efforts over the years to improve and refine the languages, there are still use cases that cannot be solved using them. 
 For instance, in RML, datatypes or language tags can only be generated as constant values, not dynamically from the source data; and there is not much support for generating RDF-star graphs~\parencite{hartig2017foundations}, an RDF extension for allowing triples to be placed as subjects and/or objects of other triples. 
 Understanding their limitations and open issues to address these uncovered use cases can help develop further features for an enhanced KG construction process.
 However, \textbf{there is a lack of comprehensive and extensive studies analyzing the expressiveness of mapping languages in fine-grain detail}, which can greatly help in (i) the choice of a language depending on the use case requirements, and (ii) identifying the challenges and open issues to further improve the languages. 
 
 %However, \textbf{studies analyzing mapping languages lack some currently used languages and depth in feature analysis}, which can help understand the current needs in KG construction, facilitate their update with so far unsupported needs and help to choose among them depending on each use case requirements.
 %\textbf{there is no study collecting and analyzing the current needs for knowledge graph construction}, which can facilitate the update of existing languages to address them. 

%% PÁRRAFO DE RDF-STAR ---> DESCARTADO POR AHORA
%Despite the variety of extensions released over the years that progressively addressed more limitations, the needs in KG construction keep evolving with the progress of related technology. This is the case, for instance, of RDF-star. This proposal~\parencite{hartig2017foundations} introduced recursiveness in RDF statements, to empower triples to become subjects and/or objects of other triples, offering a compact syntax for RDF statement reification. At the time of writing, RDF-star is becoming part of the RDF 1.2 specification~\parencite{hartig2023rdf}. This triggers the need to \textbf{update current mapping languages to be able to describe the generation of RDF-star graphs}.

%\ana{por aquí un párrafo de que tienen limiraciones, tanto en cosas que no podían hacer por complejidad del caso de uso que se ha ido viendo con los años y el uso; tanto como porque el mundo alrededor evoluciona y surgen nuevas necesidades, como la de incluir RDF-star porque va a formar parte d ela nueva spec de RDF}

%\textit{Then about how to write mappings, still hard for many people who find it difficult or blabla, how to increase the adoption by making the user experience better: visual approaches do not scale and blabla. user friendly serializations are popular, we keep on updating them with needs already identified (above), and develop a new approach that can be for bot domain experts and experienced practitioners.}

Mapping files are processed by a compliant engine along with the input data to construct KGs. 
Thus, users are not required to have coding skills, but in turn, they need to learn the mapping language. 
Most mapping languages are defined as an ontology (e.g. R2RML and its extensions). 
Therefore, they can use RDF serializations, usually Turtle~\parencite{turtle}, as the syntax to write the file.
Other popular approaches use SPARQL~\parencite{harris2013sparql} as the basis, extending the language to tackle input data description (e.g. SPARQL-Generate~\parencite{Lefrancois2017sparqlgenerate}, SPARQL-Anything~\parencite{asprino2023sparql-anything}, Tarql\footnote{\url{https://tarql.github.io/}}). 
For practitioners with a background in semantic technologies, the barrier to learn the mapping languages is lower, as they probably have previous know-how in these syntaxes. 
Without this, the learning curve for any non-expert user is high. 
For this reason, to help in the adoption of these technologies and make them more accessible for non-expert users, several different approaches were developed, ranging from visual approaches (e.g. Karma~\parencite{gupta2012karma}, RMLEditor~\parencite{heyvaert2016rmleditor}) to user-friendly syntaxes (e.g. YARRRML~\parencite{Heyvaert2018yarrrml}, SMS2~\parencite{sms2}). 
While these syntaxes present a wide adoption by more experienced practitioners, they still pose a barrier for learning the language's grammar, which may hinder its use by novel users. 
In addition, visual approaches are limited for complex or large use cases. 
Hence, further research is needed to \textbf{identify approaches that facilitate the writing of mappings, reducing the need to learn a language or syntax, while remaining scalable for large use cases}, in order to facilitate their adoption. 
In addition, each language is processed by different engines with different capabilities. When dealing with complex use cases or evolving KGs, it is sometimes necessary to use different engines, which may not implement the same language. This kind of situations forces the user to learn more than one single language, as there is little support for translating mapping rules among different mapping languages~\parencite{corcho2020towards}. 
Hence, \textbf{it is required to improve the interoperability between existing mapping languages} and bridge the gap between user-friendly mapping creation and existing KG construction implementations. 


%\ana{orientar a: si se ha estudiado como mejoran el proceso desde este ey este sentido, pero no si intervienen en la evolución}

The technologies revolving mappings are suitable for composing declarative, modular, automated approaches to be integrated into larger pipelines that manage knowledge graphs~\parencite{simsek2021knowledge,cimmino2022helio,grassi2023composable}. 
Additionally, these declarative approaches can play an important role in other different tasks of the life cycle apart from the construction. 
For instance, they enable metadata annotation about the data sources, which can help enhance the process in terms of transparency, completeness and performance~\parencite{chaves2021morph-csv,vidal2023knowledge}; as well as 
data processing and cleaning, thanks to the inclusion of data transformation functions in the language~\parencite{debruyne2016r2rmlf,junior2016funul,jozashoori2020funmap,DeMeester2017fno_dbpedia}.
However, \textbf{there is a lack of research to assess how these technologies can play a beneficial role in the evolution of knowledge graphs}, apart from their construction.

%involved not only the construction of knowledge graphs, but also the beneficial role they can play in their evolution}.


%\textcolor{red}{However, these technologies are relatively recent, and as mentioned before, struggle with a wider adoption. Still, the last few years have witnessed an increased effort for being refined and optimized~\parencite{calvanese2017ontop,chaves2019parameters,arenas2022morphkgc,iglesias2023scaling}, to reduce the learning curve and to prove their benefits with respect to ad-hoc approaches~\parencite{iglesias2019bio2rdf}. }


\section{Main Contributions}

The overall objective of this thesis is to improve the understanding and operational management of declarative KG construction languages. The contributions of this thesis are listed below, organized according to the main thesis sub-objectives:

\begin{enumerate}
    \item The first objective consists on gathering, understanding and implementing the current needs for knowledge graph construction from heterogeneous data sources with mapping languages. 

    \begin{itemize}
        \item \textbf{A comparison framework with a fine-grained analysis of the characteristics of current mapping languages.} We design a framework with the features that a mapping language may provide, from data source description to triple transformation and additional rules that apply to triple creation, and check whether current mapping languages provide such features in the language itself or supported by a compliant engine. 

        \item \textbf{A set of requirements for declarative mapping languages}, extracted from the comparison framework and the needs of the community of practitioners. These requirements are implemented in a formal language as an ontology. 

        \item \textbf{Feature update of the RML mapping language with new requirements.} We present how some of these requirements have served to update RML, providing a novel solution for constructing RDF-star graphs.
        
        %\item \textcolor{red}{\textbf{The implementation of these new requirements in the RML mapping language.}} The requirements gathered in previous contributions are implemented in RML, a widely adopted language, focusing on the update that involves enabling the construction of RDF-star graphs.
    \end{itemize}

    \item The second objective consists on improving the user experience for knowledge engineers and domain experts to write mapping rules.

    \begin{itemize}
        \item \textbf{A spreadsheet-based approach to write mapping rules}. This approach enables users to write mapping rules in well-known spreadsheet editors (e.g. MS Excel, Google Spreadsheets) without the need to learn a language's constructs or syntax, to be automatically translated into a correctly formatted mapping file with the support of a compliant tool. 

        \item \textbf{The update of the user-friendly serialization YARRRML for RML}, incorporating the latest features of the language, along with a compliant tool to translate them into human-readable [R2]RML mapping files. 
    \end{itemize}

    \item The third objective consists on assessing the role of declarative KG construction technologies to support the evolution of knowledge graphs.

    \begin{itemize}
        \item \textbf{An evaluation of how declarative KG construction technologies can be beneficial when the schema of a KG changes.} We test these approaches for re-constructing a knowledge graph, comparing them with triplestore-based re-construction. We assess the performance of each approach and the features that intervene in different scenarios. This study sheds light on how KG construction technologies can play a role in supporting the evolution of knowledge graphs.
        
        %\item \textbf{An evaluation of declarative KG construction technologies with well-known triplestores performing the re-construction of knowledge graphs} with different metadata representation models. We test the performance of each approach and the features needed for re-constructing each representation to evaluate which approach is more suitable in each case. This study sheds light on how KG construction technologies can be beneficial in supporting the evolution of knowledge graphs.
    \end{itemize}
\end{enumerate}


\section{Thesis Structure}

The remainder of this thesis is structured as follows:

\begin{itemize}
    \item \cref{chapter:sota} describes the main concepts used in this thesis, reviews the state of the art of the topics of interest and identifies the current limitations. It first introduces the basic notions about knowledge representation on the Web, to then proceed to describe the current landscape of mapping languages and methods to write them in a user-friendly manner. Then it presents the related works assessing the role of these declarative mappings in the different phases that take place within the knowledge graph life cycle. The chapter concludes identifying the gaps in the state of the art, leading to the work towards obtaining the main contributions of the thesis.
    
    \item \cref{chapter:objectives} presents the main objectives and contributions of this thesis, along with the assumptions, hypotheses and restrictions of the work and research methodology. 
    
    \item In \cref{chapter:mappings} we define a comparison framework to analyze the features of current mapping languages. This framework helps to extract a set of requirements for knowledge graph construction, that are gathered and represented in an ontology. Additionally, we present a particular case of the update of the RML mapping language to address some of the extracted requirements, specifically the RML-star module to construct RDF-star graphs.
    
    \item In \cref{chapter:creation} we present two approaches to facilitate the creation of mapping documents for users. The first relies on spreadsheets to write mapping rules, and it is tested with real users with different backgrounds to evaluate its usability. The second supposes an update on the YARRRML user-friendly syntax with respect to recent modifications in the RML language. Both approaches are supported by tools to create mapping documents in target languages. 
    
    \item In \cref{chapter:evolution} we assess the role that mapping-compliant technologies can play in knowledge graph evolution. We conduct an empirical evaluation performing schema changes with KG construction engines that use mappings, and SPARQL \texttt{CONSTRUCT} queries, and evaluate which approach is more suitable in different situations. 
    
    \item Finally, \cref{chapter:conclusions} draws the main conclusions of the presented work, and outlines the future research directions. 
    
    %\item \ana{Appendix? cuando estén más seguros}
\end{itemize}

\section{Derived Outcomes}

This section lists the publications derived from the work in this thesis.


\begin{itemize}
    \item \textbf{Journal Publications}
    \begin{itemize}
        \item Arenas-Guerrero, J.*, \textbf{Iglesias-Molina, A.}*, Chaves-Fraga, D., Garijo, D., Corcho, O. and Dimou, A. (2024) Declarative generation of RDF-star graphs from heterogeneous data. \textit{Semantic Web}, in press. This work presents and validates the latest version of RML-star described in \cref{chapter:mappings}.
        
        \item \textbf{Iglesias-Molina, A.}, Cimmino, A., Ruckhaus, E., Chaves-Fraga, D., García-Castro, R., Corcho, O. (2024) An Ontological Approach for Representing Declarative Mapping Languages. \textit{Semantic Web}, \textit{5} (1), 191–221. This work analyzes the mapping languages capabilities and limitations, gathers the requirements for KG construction and presents their formalization as an ontology described in \cref{chapter:mappings}.
    \end{itemize}
\end{itemize}




\begin{itemize}
    \item \textbf{Conference Publications}
    \begin{itemize}
        \item \textbf{Iglesias-Molina A.}, Toledo J., Corcho O. and Chaves-Fraga D. (2023) Re-Construction Impact on Metadata Representation Models. In \textit{Proceedings of The Twelfth International Conference on Knowledge Capture (K-CAP23)}, December 5 - 7, Pensacola. This work assesses the value of declarative KG construction in the evolution of the KG's schemas described in \cref{chapter:evolution}.
    
        \item \textbf{Iglesias-Molina, A.}, Van Assche, D., Arenas-Guerrero, J., De Meester, B., Debruyne, C., Jozashoori, S., Maria, P., Michel, F., Chaves-Fraga, D. and Dimou, A. (2023) The RML Ontology: A Community-Driven Modular Redesign After a Decade of Experience in Mapping Heterogeneous Data to RDF. In \textit{Proceedings of the 22nd International Semantic Web Conference (ISWC2023)}, November 6--10, Athens. This work presents the updates in RML with the new needs and requirements identified for declarative KG construction described in \cref{chapter:mappings}. 
    
        \item \textbf{Iglesias-Molina, A.}, Ahrabian, K., Ilievski, F., Pujara, J. and Corcho, O. (2023) Comparison of Knowledge Graph Representations for Consumer Scenarios. In \textit{Proceedings of the 22nd International Semantic Web Conference (ISWC2023)}, November 6--10, Athens. This work motivates the contribution presented in \cref{chapter:evolution}.
    
        \item \textbf{Iglesias-Molina, A.}, Chaves-Fraga, D., Dasoulas, I. and Dimou, A. (2023) Human-Friendly RDF Graph Construction: Which One Do You Chose?. In \textit{Proceedings of the 23rd International Conference on Web Engineering 2023 (ICWE2023)}, June 6--9, Alicante. This work updates the YARRRML syntax with the new features described in \cref{chapter:creation}.
    \end{itemize}
\end{itemize}

\begin{itemize}
    \item \textbf{Workshop Publications}
    \begin{itemize}
        \item \textbf{Iglesias-Molina, A.}, Cimmino, A., Corcho, O. (2022) Devising Mapping Interoperability with Mapping Translation. In \textit{Proceedings of the Third International Workshop on Knowledge Graph Construction, co-located with the 19th Extended Semantic Web Conference}. May 29 -- June 2, Hersonissos. This works looks into the interoperability status of current mapping languages, presented in \cref{chapter:sota}.
        
        \item \textbf{Iglesias-Molina, A.}, Chaves-Fraga, D., Priyatna, F. and Corcho, O. (2019) Towards the definition of a language-independent mapping template for knowledge graph creation. In \textit{Proceedings of the Third International Workshop on Capturing Scientific Knowledge co-located with the Eleventh International Conference on Knowledge Capture}. November 19--21, Marina del Rey. This work presents the first version of the spreadsheet-based approach for writing declarative mapping rules described in \cref{chapter:creation}.
    \end{itemize}
\end{itemize}

\begin{itemize}
    \item \textbf{Posters and demos}
    \begin{itemize}
        \item \textbf{Iglesias-Molina, A.} and Garijo D. (2023) Towards Assessing FAIR Research Software Best Practices in an Organization Using RDF-star. In \textit{Proceedings of the Semantics 2023 Posters and Demos Track}, September 19--22, Leipzig.
        
        \item Delva, T., Arenas-Guerrero, J., \textbf{Iglesias-Molina, A.}, Corcho, O., Chaves-Fraga, D., and Dimou, A. (2021) RML-star: A declarative mapping language for RDF-star generation. In \textit{Proceedings of the ISWC 2021 Posters, Demos and Industry Tracks}, October 24--28, online. This work presents the first version of RML-star described in \cref{chapter:mappings}.
    
        \item \textbf{Iglesias-Molina, A.}, Pozo-Gilo, L., Dona, D., Ruckhaus, E., Chaves-Fraga, D. and Corcho, O. (2020) Mapeathor: Simplifying the Specification of Declarative Rules for Knowledge Graph Construction. In \textit{Proceedings of the ISWC 2020 Demos and Industry Tracks}, November 2--6, online. This work refines the spreadsheet-based approach for writing declarative mapping rules and presents its implementation in a tool described in \cref{chapter:creation}.
    \end{itemize}
\end{itemize}


\section{Related Research Activities}

\subsection*{Research Stay}

During the development of this thesis, one research stay took place in the following institution:

\begin{itemize}
    \item 06/07/2022 -- 06/10/2022. Research stay at the \textbf{Information Sciences Institute of the University of Southern California}, supervised by Prof. Dr. Filip Ilievski. During this stay, we analyzed the differential impact of diverse knowledge graph representation over different consumption scenarios: knowledge exploration performed by users, systematic query performance and graph completion tasks. This stay was funded by a scholarship from Programa Propio I+D+i of UPM, oriented to research personnel in predoctoral formation for doing an international research stay equal to or superior to three months. This stay lead to the publication "Comparison of Knowledge Graph Representations for Consumer Scenarios", presented in the International Semantic Web Conference (ISWC) 2023~\parencite{iglesias2023kgconsumption}.
\end{itemize}

\subsection*{Organization of Workshops and Tutorials}
As part of the research activity and interest in contributing further to this community, the author has participated in the organization of the following events:
\begin{itemize}
    \item Organization of the \textbf{Fourth and Fifth Editions of the Knowledge Graph Construction Workshop (KGCW)}\footnote{\url{https://w3id.org/kg-construct/workshop/2023}}\textsuperscript{,}\footnote{\url{https://w3id.org/kg-construct/workshop/2024}}, co-located in the Extended Semantic Web Conference (ESWC 2023 and 2024), celebrated both times in Hersonissos, Greece, organized together with Anastasia Dimou (KU Leuven), David Chaves-Fraga (Universidade de Santiago de Compostela), Umutcan Serles (STI Innsbruck) and Dylan Van Assche (Universiteit Gent). This workshop gathers in every edition the community of KG Construction to present relevant research, breakthroughs and resources in the field, complementing the activities of the Knowledge Graph Construction W3C Community Group. The number of attendees was around 40-60 people. 

    \item Organization of the \textbf{Tutorial on Knowledge Graph Construction using Declarative Mapping Rules}\footnote{\url{https://oeg-dataintegration.github.io/kgc-tutorial-2020/}}, co-located in the 19th International Semantic Web Conference (ISWC2020), celebrated on November 2-6 2020 online, organized together with Oscar Corcho, David Chaves-Fraga and Andrea Cimmino (UPM); 
    and the \textbf{Tutorial on Declarative Construction and Validation of Knowledge Graphs}\footnote{\url{https://w3id.org/kg-construct/tutorials/kcap2023}}, co-located in the 12th International Conference on Knowledge Capture (K-CAP2023), celebrated on December 5-7 2023 in Pensacola, USA, organized together with Xuemin Duan (KU Leuven). 
    Presenter in the \textbf{Knowledge Graph Construction Tutorial}\footnote{\url{https://w3id.org/kg-construct/costdkg-eswc-tutorial}}, co-located in the 19th Extended Semantic Web Conference (ESWC2022), celebrated on May 29 – June 2 2022.
    All tutorials focused on explaining in detail, from a practical perspective, the process of constructing knowledge graphs, from writing mappings to their execution, publication and validation with suitable tools. The number of attendees in all tutorials was around 20-30 people. 
\end{itemize}

%\section{Projects}
%During the development of this thesis, the author participated in the following research and innovation projects:

%\begin{itemize}
%    \item \textit{SolarChem 5.0: Towards Digital Transition in Solar Chemistry: AI-assisted robotized platform for the development of efficient photoelectrodes}, reference TED2021-130173B-C41, within the State Program to Promote Scientific-Technical Research and its Transfer (Strategic Projects Oriented to Ecological Transition and Digital Transition).

%    \item \textit{DRUGS4COVID++: Servicios de Inteligencia Artificial para la creación de un grafo de conocimientos sobre fármacos usados en el control clínico de la enfermedad, a partir de la explotación de grandes corpus de documentación científica sobre SARS-COV-2 y COV}, funded by BBVA grants for Teams of Scientific Research on SARS-CoV-2 and COVID-19.

%    \item \textit{Ontology Extension about Insurances and Construction of Knowledge Graphs}, funded by REALE. 
    
%    \item \textit{Governance Model, Best Practices and Standards for BASF Ontologies}, funded by BASF. 
%\end{itemize}