\chapter{Introduction}
\label{chapter:intro}


* General context about knowledge graphs, they are everywhere, representing every kind of information, used for many different purposes. 

* General context about how to create KGs: code libraries, tools like openrefine, and mappings. Then on and on with mappings, how they have evolved blabla, increasing support by different engines blabla

* Mappping-compliant technologies are increasingly proving to be convenient/useful/good for KGs: maintainability, understandability, engines already optimized, no need to code (accessibility for wider range of profiles). 

However, data is still complex, the requirements for KG evolve and these languages are sometimes limited. **Then first reference to a whole: gather which are these necessities**

Then about how to write mappings, still hard for many people who find it difficult or blabla, how to increase the adoption by making the user experience better: visual approaches do not scale and blabla. user friendly serializations are popular, we keep on updating them with needs already identified (above), and develop a new approach that can be for bot domain experts and experienced practitioners.

\ana{Por aquí hay que leer y meter cosas de KG life cycle } teniendo en cuenta que los mappings están en construcción, se podría plantear cómo pueden ayudar en el mantenimiento y evolución de KGs, son suficientemente robustos los current approaches para este purpose?

\section{Thesis structure}

The remainder of this thesis is structured as follows:

\begin{itemize}
    \item \cref{chapter:sota}
    \item \cref{chapter:objectives}
    \item \cref{chapter:mappings}
    \item \cref{chapter:creation}
    \item \cref{chapter:evolution}
    \item \cref{chapter:conclusions}
\end{itemize}


\section{Related publications}

The work presented in this document has been disseminated in thw following international workshops, conferences and journals:

\begin{itemize}
    \item The first contribution in \cref{chapter:mappings} was published in: \textbf{Iglesias-Molina, A.}, Cimmino, A., Ruckhaus, E., Chaves-Fraga, D., García-Castro, R., Corcho, O.: An Ontological Approach for Representing Declarative Mapping Languages, Semantic Web Journal 2022.
    
    \item The second contribution  in \cref{chapter:mappings} was preliminary published in: Delva, T., Arenas-Guerrero, J.,\textbf{ Iglesias-Molina, A.}, Corcho, O., Chaves-Fraga, D., and Dimou, A.: RML-star: A declarative mapping language for RDF-star generation, in Proceedings of the ISWC 2021 Posters, Demos and Industry Tracks (ISWC2021). The extended version of this contribution was published in: \textbf{Iglesias-Molina, A.}, Van Assche, D., Arenas-Guerrero, J., De Meester, B., Debruyne, C., Jozashoori, S., Maria, P., Michel, F., Chaves-Fraga, D. and Dimou, A., 2023. The RML Ontology: A Community-Driven Modular Redesign After a Decade of Experience in Mapping Heterogeneous Data to RDF, in Proceedings of the 22st International Semantic Web Conference (ISWC2023); and in: Arenas-Guerrero, J., \textbf{Iglesias-Molina, A.}, Chaves-Fraga, D., Garijo, D., Corcho, O. and Dimou, A., Declarative generation of RDF-star graphs from heterogeneous data, Semantic Web Journal, 2023 (the first two authors contributed equally to this publication). This contribution is the result of the collaboration with the Knowledge Graph Construction W3C Community Group. 

    \item The first contribution in \cref{chapter:creation} was initially published \ana{cambiar si se envia la ultima parte} in: \textbf{Iglesias-Molina, A.}, Chaves-Fraga, D., Priyatna, F. and Corcho, O., 2019, November. Towards the definition of a language-independent mapping template for knowledge graph creation, in Proceedings of the Third International Workshop on Capturing Scientific Knowledge co-located with the Eleventh International Conference on Knowledge Capture 2021 (K-CAP2021); and in: \textbf{Iglesias-Molina, A.}, Pozo-Gilo, L., Dona, D., Ruckhaus, E., Chaves-Fraga, D. and Corcho, O., Mapeathor: Simplifying the Specification of Declarative Rules for Knowledge Graph Construction, in Proceedings of the ISWC 2020 Demos and Industry Tracks (ISWC2020).

    \item The second contribution in \cref{chapter:creation} was published in: \textbf{Iglesias-Molina, A.}, Chaves-Fraga, D., Dasoulas, I. and Dimou, A., Human-Friendly RDF Graph Construction: Which One Do You Chose?, in Proceedings of the 23rd International Conference on Web Engineering 2023 (ICWE2023). This contribution is the result of the collaboration with KU Leuven.

    \item The first part of the contribution in \cref{chapter:evolution} was published 
    in: \textbf{Iglesias-Molina, A.}, Ahrabian, K., Ilievski, F., Pujara, J. and Corcho, O., Comparison of Knowledge Graph Representations for Consumer Scenarios, in Proceedings of the 22st International Semantic Web Conference (ISWC2023), as a result of a collaboration with the Information Sciences Institute (ISI-USC) thanks to a research stay in the institution.
    The second part of the contribution was published in: \textbf{Iglesias-Molina A.}, Toledo J., Corcho O. and Chaves-Fraga D., Re-Construction Impact on Metadata Representation Models, \ana{venue :)}, 2023.

\end{itemize}


