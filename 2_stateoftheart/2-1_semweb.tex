\section{Knowledge Representation in the Semantic Web \textcolor{red}{-- By 19/1}}
\label{sec:chp2_semweb}

\textit{This section describes some background knowledge about concepts that play a relevant role in knowledge representation on the web. First we look into RDF, used for representing knowledge graphs on the web. Then we go over different approaches to reify knowledge in RDF, useful for representing additional information beyond triple paradigm. ??}

\subsection{Resource Description Framework (RDF)}

\ana{intro, what is rdf, triple paradigm, uris, serializations, examples (very much like David's})

The Resource Description Framework (RDF)~\parencite{rdf} is a standard model for data interchange on the web. Its basic unit are triples: two concepts linked by a relationship. These triples are represented in the form of $<$ \textit{s,p,o} $>$, where \textit{s} is the subject, \textit{p} is the predicate and \textit{o} the object. This model relies on the use on the linking structure of the Web, using IRIs to identify uniquely every resource and relationship. 

\subsection{Statements about statements in RDF}

\ana{basic intro about what is this for and mot example about info that cannot be plainly represented in RDF. Then present all approaches used along the document: std reification, n-ary relationships, singleton properties, rdf-star, named graphs, i'd leave wikidata out of this}