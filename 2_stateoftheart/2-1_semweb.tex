\section{Knowledge Representation in the Semantic Web \textcolor{red}{By 12/1}}
\label{sec:chp2_semweb}

\textit{This section describes some background knowledge about concepts that play a relevant role in knowledge representation on the web. First we look into RDF, used for representing knowledge graphs on the web. Then we go over different approaches to reify knowledge in RDF, useful for representing additional information beyond triple paradigm. ??}

\subsection{Resource Description Framework (RDF)}
\ana{take from the spec of some basic books about semweb or sth}

\ana{intro, what is rdf, triple paradigm, uris, serializations, examples (very much like David's}

\subsection{Statements about statements in RDF}

\ana{basic intro about what is this for and mot example about info that cannot be plainly represented in RDF. Then present all approaches used along the document: std reification, n-ary relationships, singleton properties, rdf-star, named graphs, i'd leave wikidata out of this}