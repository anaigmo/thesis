\chapter{State of the Art \textcolor{red}{-- By 2/2}}
\label{chapter:sota}

In this chapter, we discuss the state of the art in declarative construction of knowledge graphs. We first present \textcolor{red}{the representation of knowledge on the web} (\cref{sec:chp2_semweb}). Then we focus on describing current techniques to construct knowledge graphs, focusing on the declarative approaches (\cref{sec:chp2_declarative_kgc}), and the user-friendly approaches developed to facilitate this task (\cref{sec:chp2_easy_kgc}). We also describe the current proposals and methodologies for supporting the knowledge graph life cycle \ana{queda algo suelto, encajar algo de construcción y porqué tiene esto relevancia con el resto} (\cref{sec:chp2_kg_lifecycle}).



\section{Knowledge Representation in the Semantic Web \textcolor{red}{-- By 19/1}}
\label{sec:chp2_semweb}

\textit{This section describes some background knowledge about concepts that play a relevant role in knowledge representation on the web. First we look into RDF, used for representing knowledge graphs on the web. Then we go over different approaches to reify knowledge in RDF, useful for representing additional information beyond triple paradigm. ??}

\subsection{Resource Description Framework (RDF)}

The Resource Description Framework (RDF)~\parencite{rdf} is a standard model for data interchange on the web. Its basic unit are triples: two concepts linked by a relationship. These triples are represented in the form of $<$ \textit{s,p,o} $>$, where \textit{s} is the subject, \textit{p} is the predicate and \textit{o} the object. This model relies on the use on the linking structure of the Web, using IRIs to identify uniquely every resource and relationship. 

\cref{lst:chp2_rdf-example} presents an example of an RDF graph composed of a set of triples about pole vault records, which is also visually depicted in \cref{fig:chp2_rdf-example}. In total there are four triples. They all share the same subject, which is a resource uniquely identified by the IRI $<$\texttt{http://example.com/athlete/1}$>$. Likewise, all predicates in the triples are also defined by an IRI, e.g. $<$\texttt{http://example.com/ns\#name}$>$. The first triple (Line 5) defines that the subject is an instance of the class \texttt{ex:Athlete} with the predicate \texttt{rdf:type}. The rest of the triples define attributes to this instance with name (\texttt{ex:name}) and position in the ranking (\texttt{ex:rank}). The objects of both triples are literals, that may be strings ("Yelena Isinbayeva", Line 6) or typed literals ("1"\scalebox{.8}{\textsuperscript{$\wedge\wedge$}}\texttt{xsd:integer}, Line 7). Type literals refer to data values attached with a tag that represents their data type.

\begin{minipage}{\textwidth}
\begin{captionedlisting}{lst:chp2_rdf-example}{Example of RDF graph.}
\centering
{\begin{lstlisting}[language=r2rml]
@prefix rdf: <http://www.w3.org/1999/02/22-rdf-syntax-ns#>.
@prefix xsd: <http://www.w3.org/2001/XMLSchema#>.
@prefix ex: <http://example.com/ns#>.

<http://example.com/athlete/1> rdf:type ex:Athlete .
<http://example.com/athlete/1> ex:name "Yelena Isinbayeva" .
<http://example.com/athlete/1> ex:rank "1"^^xsd:integer .
\end{lstlisting}}
\end{captionedlisting}
\end{minipage}

There are different syntaxes to serialize RDF graphs. RDF/XML\footnote{\url{https://www.w3.org/TR/rdf-syntax-grammar}} was the first to be proposed, and relies on XML. Notation3 (N3)\footnote{\url{https://www.w3.org/TeamSubmission/n3/}} was developed as a human readable syntax, but its use is not common. Instead, the N-Triples\footnote{\url{https://www.w3.org/TR/n-triples}} and Turtle\footnote{\url{https://www.w3.org/TR/turtle}} serializations, subsets of N3, are more widely used. JSON-LD\footnote{\url{https://www.w3.org/TR/json-ld11}} was developed for programming environments, as it comprised in JSON documents. Lastly, RDFa\footnote{\url{https://www.w3.org/TR/rdfa-primer}} extends HTML to markup structured content in webpages, with the objective of improving the results retrieved by search engines. 



\begin{figure*}[t]
\centering
\includegraphics[width=0.6\linewidth]{figures/chp2_rdf-example.pdf}
\caption[RDF graph example]{Visual representation of the RDF graph shown in \cref{lst:chp2_rdf-example}.}
\label{fig:chp2_rdf-example}
\end{figure*}

\subsection{Statements about statements in RDF}

%\ana{basic intro about what is this for and mot example about info that cannot be plainly represented in RDF. Then present all approaches used along the document: std reification, n-ary relationships, singleton properties, rdf-star, named graphs, i'd leave wikidata out of this}

Plain triples are not always able to represent the complexity of certain knowledge. This is the case of statement annotation, when it is required to add information to a triple that does not correspond to a single resource, but the whole. This situation has triggered the development different approaches to achieve triple annotation (or reification). \ana{figure } illustrates instances of these models for the main statement \textit{Yelena Isinbayeva obtained the first position in the rank}, annotated with the additional statement \textit{in the season of 2009}. 

\begin{figure*}[t]
\centering
\includegraphics[width=\linewidth]{figures/chp2_reifications.pdf}
\caption[Approaches for statement reification in RDF]{Approaches for statement reification in RDF: (a) Standard Reification, (b) N-Ary Relationships, (c) Singleton Properties, (d) RDF-star and (e) Named Graphs.}
\label{fig:chp2_reification}
\end{figure*}

\noindent\textbf{Standard Reification}~\cite{lassila1999rdf} explicitly declares a resource to denote an \texttt{rdf:Statement}.
This statement has \texttt{rdf:subject}, \texttt{rdf:predicate}, and \texttt{rdf:object} attached to it and can be further annotated with additional statements. The resource is typically a blank node, but an IRI can be used. \textcolor{red}{In \cref{fig:chp2_reification}a, the resource \texttt{:entity-stm/407} is an \texttt{rdf:Statement} with four associated triples, where the objects of the triples are the actual values of the triple (i.e. \texttt{:entity/407} for the subject, \texttt{:semanticType} for the predicate, and \texttt{"orga"} for the object). The property \texttt{:score} is used with its own value as object.}


\noindent\textbf{N-Ary Relationships}~\cite{naryw3c2006} converts a relationship into an instance that describes the relation, which can have attached both the main object and additional statements.
This representation is widely used in ontology engineering as an ontology design pattern~\cite{gangemi2013multi}.\textcolor{red}{ In \cref{fig:chp2_reification}b, the entity \texttt{:entity/407} points to an intermediate node (\texttt{:entity-semtype/ 407}) which holds the triples for both the assignment of the semantic type and the score.}

\noindent\textbf{Singleton Properties}~\cite{nguyen2014don}

\noindent\textbf{RDF-star}~\cite{hartig2017foundations,hartig2023rdf12} extends RDF to introduce a new syntax for compact triple reification. 
It introduces the notion of triple recursiveness with \texttt{Quoted Triples}, which can be used as subjects and/or objects of other triples. This is the only approach that extends the standard RDF features. 
%, having a potential impact on the development of supporting tools, triplestores, etc. 
This representation is currently being incorporated into the RDF 1.2 specification~\cite{hartig2023rdf12}, which is currently being developed under the RDF-star W3C Working Group\footnote{\url{https://www.w3.org/groups/wg/rdf-star/}}.
\textcolor{red}{In Figure \ref{fig:chp2_reification}d we observe the example as an RDF-star graph, and it is represented in RDF as \texttt{{<<:entity/407 :semanticType "orga">> :score 0.8}}.}


\noindent\textbf{Named Graphs}~\parencite{cyganiak2014rdf11} are a SPARQL 1.1 feature that allows the assignment of an IRI to one or several triples as a graph identifier. Hence, graph IRIs allow the unique identification of triples. These IRIs can be used as subjects to add additional statements. \textcolor{red}{In \cref{fig:chp2_reification}e, the triple indicating the semantic type of an entity is assigned the named graph \texttt{:entity-graph/407}. The graph IRI is subsequently used as subject in a triple that annotates the confidence score of the information within the graph. }

\section{Declarative Knowledge Graph Construction \textcolor{red}{By 12/1}}
\label{sec:chp2_declarative_kgc}

Knowledge graphs can be constructed in diverse manners. One way comprises collecting knowledge from contributions form the community, such as Wikidata. Other way involves transforming data from heterogeneous formats and sources, unstructured or (semi-)structured, into RDF. This section focuses on providing an overview of the latter, to declaratively construct knowledge graphs from heterogeneous data sources. \ana{decir que pasa en cada subsección?}

%\ana{overview de distintas formas y métodos en general de construir KGs. Debería ser algo larga, ampliando lo que hay en la intro. Luego terminar con enfocarse en los approaches declarativos, que se extienden en la siguiente subsección}

%\ana{RAW} In this section, the current scene of mapping languages is described first, regardless of the approach they follow, i.e., RDF materialization or virtualization. Then, previous works comparing mapping languages are surveyed. 



\subsection{Declarative Mapping Rules}

Constructing RDF knowledge graphs from heterogeneous data sources involve a schema transformation of the data to the desired graph structure. This transformation may be done in several different ways, usually involving mapping languages that allow expressing the transformation rules to create the target graphs. Hence, these mappings hold declaratively the relationships between the source and target data schemas. This comprises an agnostic approach that can (and has been) applied to multiple different use cases. \ana{refs}

The usual workflow in which declarative approaches are involved is depicted in \ana{figure}. They enable both materialization and virtualization of knowledge graphs. In materialization scenarios, data is transformed into the target graph, usually following the schema provided by an ontology. In virtualization scenarios, data is not transformed; instead, the original data source is accessed also following the schema of the target (virtual) graph. This process requires translating the original query into the language of the original data source. \ana{mention OBDA?}

Following, we present an overview of existing mapping languages, listed in \cref{tab:chp2_languages_summary}. We classify these languages in three categories, based on the schema they are based on or extend: (i)~RDF-based, (ii)~SPARQL-based, and (iii)~based on alternative schemas. An overview of the evolution, extensions and influences of these languages is depicted in \cref{fig:chp2_mapping_languages}.



\begin{figure*}[h]
\centering
\includegraphics[width=1\linewidth]{figures/chp2_mapping_languages}
\caption[Existing mapping languages and the relationships among them]{Existing mapping languages and the relationships among them.}
\label{fig:chp2_mapping_languages}
\end{figure*}

\begin{table}[t]
\caption[Mapping languages overview]{Analyzed mapping languages and their corresponding references. \ana{pensar en eliminar helio, la versión nueva igual no entra dentro de esto y puede que cuadre quitar SMS2 y ponerlo luego con el análisis de las sintaxis}}
\label{tab:chp2_languages_summary}
\begin{tabular}{c|c|c}
\hline
%\rowcolor[HTML]{EFEFEF} 
Classification                 & Language        & Reference(s) \\ \hline
\multirow{11}{*}{RDF-based}   & D2RQ            & \parencite{bizer2004d2rq,d2rq}\\ \cline{2-3} 
                              & R$_2$O          & \parencite{barrasa2004r2o}\\ \cline{2-3} 
                              & R2RML           & \parencite{das2012r2rml}\\ \cline{2-3} 
                              & xR2RML          & \parencite{michel2015xr2rml,xr2rml}\\ \cline{2-3} 
                              & RML             & \parencite{Dimou2014rml,rml}\\ \cline{2-3} 
                              & KR2RML          & \parencite{slepicka2015kr2rml}\\ \cline{2-3} 
                              & FunUL           & \parencite{junior2016funul}\\ \cline{2-3} 
                              & R2RML-f         & \parencite{debruyne2016r2rmlf}\\ \cline{2-3} 
                              & D2RML           & \parencite{chortaras2018d2rml}\\ \cline{2-3} 
                              & R2RML for collections & \parencite{debruyne2017R2RML-collections}\\ \cline{2-3}   
                              & XLWrap          & \parencite{langegger2009xlwrap,xlwrap}\\ \cline{2-3} 
                              & CSVW            & \parencite{Tennison2015csvw}\\ \hline
\multirow{4}{*}{SPARQL-based} & SPARQL-Generate &     
                              \parencite{Lefrancois2017sparqlgenerate,sparqlgenerate}\\ \cline{2-3} 
                              & XSPARQL         & \parencite{Bischof2012xsparql,xsparql}\\ \cline{2-3} 
                              & TARQL           & \parencite{tarql}\\ \cline{2-3}
                              & Facade-X        & \parencite{asprino2023sparql-anything,sparqlanything}\\ \cline{2-3}
                              & Sansa            & \parencite{stadler2023spark}\\ \hline
\multirow{3}{*}{Others}       & Helio mappings  & \parencite{cimmino2022helio}\\ \cline{2-3} 
                              & D-REPR          & \parencite{Vu2019d-repr}\\ \cline{2-3} 
                              & ShExML          & \parencite{Garcia-Gonzalez2020shexml,shexml}\\  \hline
\end{tabular}
\end{table}




\subsubsection{RDF-based mapping languages.} 

This group of languages are specified as ontologies or vocabularies able to describe the transformation rules of heterogeneous data into RDF. They are written in RDF documents, usually using the Turtle syntax~\parencite{turtle}. 



\noindent\textbf{D2RQ}~\parencite{bizer2004d2rq}

\noindent\textbf{XLWrap}~\parencite{langegger2009xlwrap}

\noindent\textbf{R2RML}~\parencite{das2012r2rml} 

\noindent\textbf{RML}~\parencite{Dimou2014rml} + extensions (RML-star, RML fields, RML target, RML+FnO)

\noindent\textbf{KR2RML}~\parencite{slepicka2015kr2rml}

\noindent\textbf{xR2RML}~\parencite{michel2015xr2rml}

\noindent\textbf{FunUL}~\parencite{junior2016funul}

\noindent\textbf{R2RML-F}~\parencite{debruyne2016r2rmlf}

\noindent\textbf{R2RML for collections and containers}~\parencite{debruyne2017R2RML-collections}

\noindent\textbf{D2RML}~\parencite{chortaras2018d2rml}

\noindent\textbf{CSVW}~\parencite{Tennison2015csvw}

\textit{The most well-known language in this category is R2RML~\parencite{das2012r2rml}, which allows mapping of data stored in relational databases to RDF. This language is heavily influenced by previous languages (R$_2$O~\parencite{barrasa2004r2o} and D2RQ~\parencite{bizer2004d2rq}). Some serializations (e.g. SML~\parencite{Stadler2015sml}, OBDA mappings from Ontop~\parencite{rodriguez2015efficient}) and several extensions of R2RML were developed in the following years after its release: R2RML-f~\parencite{debruyne2016r2rmlf} extends R2RML to include functions to be applied over the data; RML~\parencite{Dimou2014rml} and its user-friendly compact syntax YARRRML~\parencite{Heyvaert2018yarrrml} provide the possibility of covering additional data formats (CSV, XML and JSON); this language also considers the use of functions for data transformation (e.g. lowercase, replace, trim) by using the Function Ontology (FnO)\footnote{\url{https://fno.io/rml/}}~\parencite{DeMeester2017fno_dbpedia}; FunUL~\parencite{junior2016funul} proposes an extension to also incorporate functions, but focusing on the CSV format; KR2RML~\parencite{slepicka2015kr2rml} is also an extension for CSV, XML and JSON, with the addition of representing all sources with the Nested Relational Model as an intermediate model and the possibility of cleaning data with Python functions; xR2RML~\parencite{michel2015xr2rml} extends R2RML and RML to include NoSQL databases and incorporates more features to handle tree-like data; D2RML~\parencite{chortaras2018d2rml}, also based on R2RML and RML, is able to transform data from XML, JSON, CSVs and REST/SPARQL endpoints, and enables functions and conditions to create triples. }

\textit{In this category, we can also find more languages not related to R2RML. XLWrap~\parencite{langegger2009xlwrap} is focused on transforming spreadsheets into different formats. CSVW~\parencite{Tennison2015csvw} enables tabular data annotation on the Web with metadata, but also supports the generation of RDF. Finally, WoT Mappings~\parencite{cimmino2020ewot} are oriented to be used in the context of the Web of Things.}




\subsubsection{SPARQL-based mapping languages.} 

This group is integrated by languages that leverage the SPARQL query language, usually by extending its features to describe non-RDF data sources~\parencite{harris2013sparql}. 


\noindent\textbf{XSPARQL}~\parencite{Bischof2012xsparql}

\noindent\textbf{SPARQL-Generate}~\parencite{Lefrancois2017sparqlgenerate}

\noindent\textbf{TARQL}~\parencite{tarql}

\noindent\textbf{SPARQL-Anything}~\parencite{asprino2023sparql-anything}

\noindent\textbf{Sansa/whatever}~\parencite{stadler2023spark}




\textit{XSPARQL~\parencite{Bischof2012xsparql} merges SPARQL and XQuery to transform XML into RDF. TARQL~\parencite{tarql} uses the SPARQL syntax to generate RDF from CSV files. SPARQL-Generate~\parencite{Lefrancois2017sparqlgenerate} is capable of generating RDF and document streams from a wide variety of data formats and access protocols. Most recently, Facade-X has been developed, not as a new language, but as a "\textit{facade} to wrap the original resource and to make it queryable as if it was RDF"~\parencite{asprino2023sparql-anything}. It does not extend the SPARQL language, instead it overrides the SERVICE operator. Lastly, authors would like to highlight a loosely SPARQL-based language, Stardog Mapping Syntax 2 (SMS2)~\parencite{sms2}, which represents virtual Stardog graphs and is able to support sources such as JSON, CSV, RDB, MongoDB and Elasticsearch.}




\subsubsection{Based on other schemas.} 

Lastly, the languages of this group rely on schemas different from RDF or SPARQL, equally able and expressive enough to enable users to write transformation rules. 


\noindent\textbf{R$_2$O}~\parencite{barrasa2004r2o} XML-based~\footnote{https://pdfs.semanticscholar.org/4c47/0826aafc07fc6d37ca7e2474c1d3b290ade1.pdf}

\noindent\textbf{D-REPR}~\parencite{Vu2019d-repr}

\noindent\textbf{ShExML}~\parencite{Garcia-Gonzalez2020shexml,garcia2021shexml-challenges}

\noindent\textbf{Helio??}~\parencite{cimmino2022helio}

\textit{ShExML~\parencite{Garcia-Gonzalez2020shexml,garcia2021shexml-challenges} uses Shape Expressions (ShEx)~\parencite{prud2014shex} to map data sources in RDBs, CSV, JSON, XML and RDF using SPARQL queries. The Helio mapping language~\parencite{cimmino2022helio} is based on JSON and provides the capability of using functions for data transformation and data linking~\parencite{cimmino2018hybrid}. D-REPR~\parencite{Vu2019d-repr} focuses on describing heterogeneous data with JSONPath and allows the use of data transformation functions. XRM (Expressive RDF Mapper)~\parencite{xrm} is a commercial language that provides a unique user-friendly syntax to create mappings in R2RML, CSVW and RML.}










\subsection{Mapping Languages Comparison}
As the number of mapping languages increased and their adoption grew wider, comparisons between these languages inevitably occurred. This is the case of, for instance, SPARQL-Generate~\parencite{Lefrancois2017sparqlgenerate}, which is compared to RML in terms of query/mapping complexity; and ShExML~\parencite{Garcia-Gonzalez2020shexml}, which is compared to SPARQL-Generate and YARRRML from a usability perspective.

Some studies dig deeper, providing qualitative complex comparison frameworks. Hert et al.~\parencite{hert2011comparison} provide a comparison framework for mapping languages focused on transforming relational databases to RDF. The framework is composed of 15 features, and the languages are evaluated based on the presence or absence of these features.% (Logical table to class, M:N relationships, project attributes, select conditions, user-defined instance URIs, literal to URIs, vocabulary reuse, transformation functions, datatypes, named graphs, blank nodes, integrity constraints, static metadata, one table to \textit{n} classes, and write support). 
The results lead authors to divide the mappings into four categories (direct mapping, read-only general-purpose mapping, read-write general-purpose mapping, and special-purpose mapping), and ponder on the heavy reliance of most languages on SQL to implement the mapping, and the usefulness of read-write mappings (i.e., mappings able to write data in the database). De Meester et al.~\parencite{DeMeester2019comparison} show an initial analysis of 5 similar languages (RML+FnO, xR2RML, FunUL, SPARQL-Generate, YARRRML) discussing their characteristics, according to three categories: non-functional, functional and data source support. The study concludes by remarking on the need to build a more complete and precise comparative framework and asking for a more active participation from the community to build it. To the best of our knowledge, there is no comprehensive work in the literature comparing all existing languages. \ana{ojo con esto que con el survey de ghent ya no es del todo cierto :(} 


\section{User-Friendly Knowledge Graph Construction Approaches}
\label{sec:chp2_easy_kgc}
 
This section presents the different approaches developed for easing the writing of mapping rules for practitioners. We divide these approaches into two categories, (i) visual editors (\cref{sec:chp2_visual-editors}) that rely in an interactive application to graphically depict and edit mappings, and (ii) serializations (\cref{sec:chp2_serializations}) that rely on a simplified syntax for writing the mapping rules. \cref{tab:chp2_easy-mappings} lists the approaches described in the remaining of the section.



\subsection{Visual Editors}
\label{sec:chp2_visual-editors}

Visual editors were first developed for enhancing and easing the mapping writing process. We can subdivide the proposals released over the years based on how the mapping is graphically represented in the tool: either with a tree or graph layout, or using building blocks (\cref{fig:chp2_visual-editors}).


The tree layout (\cref{fig:chp2_visual-editors}a) succeeded in the first approaches developed. 
The first one was developed for the R$_2$O mapping language in \textbf{ODEMapster}~\parencite{barrasa2006odemapster}, providing a graphical interface to visualize and edit the mappings by linking the database elements with the ontology resources. 
After the release of R2RML, \textbf{Karma}~\parencite{gupta2012karma} and \textbf{RBA} (R2RML by assertion)~\parencite{neto2013rba} were developed for this language. Both work similarly to ODEMapster, and Karma additionally provides automatic mapping suggestions. 

\begin{table}[t]
\caption[Approaches for easy mapping creation]{Approaches for facilitating the mapping creation process for users. The proposals are divided into (i) visual editors and (ii) text-based serializations. For each proposal, the reference, compliant mapping language and subtype is provided.}
\label{tab:chp2_easy-mappings}
\resizebox{\columnwidth}{!}{
\centering
\begin{tabular}{cccc}
%\rowcolor[HTML]{EFEFEF} 
\textbf{Classification} & \textbf{Approach} & \textbf{Mapping Language} & \textbf{Type} \\ \midrule
\multirow{15}{*}{\textbf{Visual Editor}} & ODEMapster~\parencite{barrasa2006odemapster} & R2O & Tree \\
 & Karma~\parencite{gupta2012karma} & R2RML & Tree \\
 & RBA~\parencite{neto2013rba} & R2RML & Tree \\
 & \cite{sengupta2013editing} & R2RML & Building blocks \\
 & \cite{lembo2014visualization} & R2RML & Graph \\
 & RMLEditor~\parencite{heyvaert2016rmleditor} & R2RML/RML & Graph \\
 & SQuaRE~\parencite{blinkiewicz2016square} & R2RML & Graph \\
 & OntopPro~\parencite{calvanese2017ontop} & Proprietary & Building blocks \\
 & Juma~\parencite{junior2017juma} & R2RML & Building blocks \\
 & RMLx~\parencite{aryan2017rmlx} & RML & Building blocks \\
 & Map-On~\parencite{sicilia2017map} & R2RML & Graph \\
 & gra.fo\tablefootnote{\label{foot:gra.fo}\url{https://gra.fo/}} & R2RML & Graph \\
 & Stardog designer\tablefootnote{\label{foot:stardog-designer}\url{https://www.stardog.com/designer/}} & SMS2 & Graph \\
 & Ontopic Studio\tablefootnote{\label{foot:ontopic-studio}\url{https://ontopic.ai/en/ontopic-studio/}} & R2RML/Proprietary & Building blocks \\
 & Eccenca Corporate Memory\tablefootnote{\label{foot:eccenca}\url{https://documentation.eccenca.com/latest/build/lift-data-from-tabular-data-such-as-csv-xslx-or-database-tables}} & Proprietary & Building blocks \\ \midrule
\multirow{5}{*}{\textbf{Serialization}} & SML~\parencite{Stadler2015sml} & R2RML & SPARQL-based \\
 & Ontop proprietary~\parencite{calvanese2017ontop} & R2RML & TTL- and SQL-based \\
 & YARRRML~\parencite{Heyvaert2018yarrrml} & RML & YAML-based \\
 & SMS2\tablefootnote{\label{foot:sms2}\url{https://docs.stardog.com/archive/7.5.0/virtual-graphs/mapping-data-sources.html\#sms2-stardog-mapping-syntax-2}} & R2RML & SPARQL-based \\
 & XRM\tablefootnote{\label{foot:xrm}\url{https://zazuko.com/products/expressive-rdf-mapper/}} & R2RML/RML/CSVW & Proprietary syntax \\ \bottomrule
\end{tabular}
}
\end{table}


\begin{figure*}[t]
\centering
\includegraphics[width=0.9\linewidth]{figures/chp2_visual-editors.pdf}
\caption[Graphical representations approaches in visual mapping editors.]{Graphical representations approaches in visual mapping editors: (a) tree layout, (b) graph layout, and (c) building blocks.}
\label{fig:chp2_visual-editors}
\end{figure*}


The graph display (\cref{fig:chp2_visual-editors}b) was later adopted by multiple editors, such as in \cite{lembo2014visualization}, \textbf{SQuaRE}~\parencite{blinkiewicz2016square}, \textbf{RMLEditor}~\parencite{heyvaert2016rmleditor} and \textbf{Map-On}~\parencite{sicilia2017map}. These editors provide a graph overview of the mapping while constructing it, while also showing the data sources and ontology. All of them are able to create R2RML mappings, and in addition, the RMLEditor can also produce RML mappings. There are also non open-source editors developed by companies that use a graph layout, such as \textbf{\url{gra.fo}}\cref{foot:gra.fo} and the \textbf{Stardog designer}\cref{foot:stardog-designer}. The former works with R2RML, and the latter with the Stardog proprietary syntax, SMS2, described in the next section. 

%~\parencite{fu2013tree-vs-graph} 

Some tools were also developed that follow an alternative approach to the tree and graph layouts, broadly used in the semantic web supporting applications. This approach comprises the use of building blocks or templates (\cref{fig:chp2_visual-editors}c), that are the components to build a mapping between data source and ontology. The first editor following this approach was proposed in \cite{sengupta2013editing}, later refined in \cite{pinkel2014best}. \textbf{OntopPro}~\parencite{calvanese2017ontop} released a Protege plugin to create and edit mappings (in their proprietary language) with templates, allowing also the creation of RDF triples and running SPARQL queries. \textbf{Juma}~\parencite{junior2017juma} and \textbf{RMLx}~\parencite{aryan2017rmlx} allow building R2RML and RML mappings respectively with building blocks, correspondent to the different parts of the mappings. There are also a couple of examples that enable this kind of visualization for mapping construction but using a proprietary language, \textbf{Ontopic Studio}\cref{foot:ontopic-studio} and \textbf{Eccenca Corporate Memory}\cref{foot:eccenca}. However, despite providing a friendly interface for users with non-technical profiles, the uptake of visual editors is limited.  

\subsection{Serializations}
\label{sec:chp2_serializations}

As an alternative to visual approaches, text-based user-friendly serializations were developed. This approach suited most practitioners with technical profiles or with preferences for a text oriented environment. 
\textbf{SML}~\parencite{Stadler2015sml} was developed as user-friendly syntax for R2RML. It provides an SPARQL-based syntax, enhancing the simplicity for writing mappings but maintaining the same expressiveness. 
The virtual KG processor \textbf{Ontop}~\parencite{calvanese2017ontop} also provides a simplified serialization for R2RML, which combines the Turtle syntax for triple generation and SQL for data access. 
The Stardog triplestore developed another SPARQL-based proprietary serialization, the Stardog Mapping Syntax 2 (\textbf{SMS2})\cref{foot:sms2}. 

Later on, as more mapping languages emerged, new serializations with a broader language compliance range emerged. This is the case of \textbf{XRM}\cref{foot:xrm} (Expressive RDF Mapper), that provides a unique syntax for writing R2RML, RML and CSVW mappings. It can be used with a service plugin integrated into common text editors, Visual Studio Code and Eclipse. This service warns the users about errors in the mapping while writing, and translates the mapping rules into one of the aforementioned languages. 

\textbf{YARRRML}~\parencite{Heyvaert2018yarrrml} was developed as a YAML-based\footnote{\url{https://yaml.org/spec/1.2.2/}} user-friendly syntax for RML. 
\cref{lst:chp2_yarrrml-mapping} shows an example of a YARRRML mapping equivalent to the RML example shown in \cref{lst:chp2_rml-mapping}. 
YARRRML mappings need as well the declaration of the prefixes at the beginning of the document, which is done using the \texttt{prefixes} key (Lines 1-2). The mapping rules are defined within the \texttt{mappings} key (Lines 4-13), which are aggregated in a rule set with keys named by the user (e.g. \texttt{athletes} key, Line 5). Each mapping rule set describes the input data sources (\texttt{sources} key, Lines 6-7), subject (\texttt{s} key, Line 8) and predicate-object pairs (\texttt{po} key, Lines 9-11).
This serialization can be used with the Matey\footnote{\url{https://rml.io/yarrrml/matey/}}, a web service able to translate YARRRML into RML mapping files, or directly generate RDF triples. 

\begin{captionedlisting}{lst:chp2_yarrrml-mapping}{YARRRML mapping to generate the RDF graph in \cref{lst:chp2_r2rml-result-rdf} with data from the JSON file shown in \cref{lst:chp2_json-example}. This mapping translates into the RML mapping shown in \cref{lst:chp2_rml-mapping}.}
\centering
{\begin{lstlisting}[language=yarrrml]
prefixes:
 ex: "http://example.com/ns#"

mappings:
  athletes:
    sources:
      - ["data.json~jsonpath", "$\dollar$.*"]
    s: http://example.com/athlete/$\dollar$(RANK)
    po:
      - [a, ex:Athlete]
      - [ex:name, $\dollar$(NAME)]
      - [ex:rank, $\dollar$(RANK)]
      - [ex:mark, $\dollar$(MARK)]
\end{lstlisting}}
\end{captionedlisting}

These serializations are in general widely adopted as they have proven useful for facilitating the writing of mapping rules. Hence, as mapping languages change and incorporate new features, these serializations need to be updated as well.

\section{Knowledge Graph Lifecycle}
\label{sec:chp2_kg_lifecycle}

\section{Conclusions and limitations of the State of the Art \textcolor{red}{-- By 2/2}}

\ana{hay que reformular, pero estas serian las ideas} 

\begin{itemize}
    \item \textit{identification and characterization of expressivenes and features of current mapping languages to identify needs and challenges in KGC with declarative approaches}
    \item \textit{for mapping creation: visual approaches not really adopted, but serializations not really suitable for non-technical profies, find the middle ground there}
    \item \textit{to motivate the adoption of declarative KGC technologies, see in which parts or how can be involved in other parts of KG life cycle and if it really can improve the process}
\end{itemize}