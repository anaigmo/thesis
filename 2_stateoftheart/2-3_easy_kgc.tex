\section{User-Friendly Knowledge Graph Construction Approaches \textcolor{red}{-- By 25/1}}
\label{sec:chp2_easy_kgc}

This section presents the different approaches developed for easing the writing of mapping rules for practitioners. We divide these approaches into two categories, (i) visual editors (\cref{sec:chp2_visual-editors}) that rely in an interactive application to graphically depict and edit mappings, and (ii) serializations (\cref{sec:chp2_serializations}) that rely on a simplified syntax for writing the mapping rules. \cref{tab:chp2_easy-mappings} lists the approaches described in the remaining of the section.

\begin{table}[t]
\caption[Approaches for easy mapping creation]{Approaches for facilitating the mapping creation process for users. The proposals are divided into (i) visual editors and (ii) text-based serializations. For each proposal, the reference, compliant mapping language and subtype is provided.}
\label{tab:chp2_easy-mappings}
\resizebox{\columnwidth}{!}{
\centering
\begin{tabular}{cccc}
%\rowcolor[HTML]{EFEFEF} 
\textbf{Classification} & \textbf{Approach} & \textbf{Mapping Language} & \textbf{Type} \\ \midrule
\multirow{13}{*}{\textbf{Visual Editor}} & ODEMapster~\parencite{barrasa2006odemapster} & R2O & Tree \\
 & Karma~\parencite{gupta2012karma} & R2RML & Tree \\
 & RBA~\parencite{neto2013rba} & R2RML & Tree \\
 & \cite{sengupta2013editing} & R2RML & Building blocks \\
 & \cite{lembo2014visualization} & R2RML & Graph \\
 & RMLEditor~\parencite{heyvaert2016rmleditor} & R2RML/RML & Graph \\
 & Square~\parencite{blinkiewicz2016square} & R2RML & Graph \\
 & Juma~\parencite{crotti2017juma} & R2RML & Building blocks \\
 & Map-On~\parencite{sicilia2017map} & R2RML & Graph \\
 & gra.fo\tablefootnote{\url{https://gra.fo/}} & R2RML & Graph \\
 & Stardog designer\tablefootnote{\url{https://www.stardog.com/designer/}} & SMS2 & Graph \\
 & Ontopic Studio\tablefootnote{\url{https://ontopic.ai/en/ontopic-studio/}} & Proprietary & Building blocks \\
 & Eccenca Corporate Memory\tablefootnote{\url{https://documentation.eccenca.com/latest/build/lift-data-from-tabular-data-such-as-csv-xslx-or-database-tables}} & Proprietary & Building blocks \\ \midrule
\multirow{5}{*}{\textbf{Serialization}} & SML~\parencite{Stadler2015sml} & R2RML & SPARQL-based \\
 & Ontop proprietary~\parencite{calvanese2017ontop} & R2RML & TTL- and SQL-based \\
 & YARRRML~\parencite{Heyvaert2018yarrrml} & RML & YAML-based \\
 & SMS2\tablefootnote{\label{foot:sms2}\url{https://docs.stardog.com/archive/7.5.0/virtual-graphs/mapping-data-sources.html\#sms2-stardog-mapping-syntax-2}} & R2RML & SPARQL-based \\
 & XRM\tablefootnote{\label{foot:xrm}\url{https://zazuko.com/products/expressive-rdf-mapper/}} & R2RML/RML/CSVW & Proprietary syntax \\ \bottomrule
\end{tabular}
}
\end{table}


\subsection{Visual Editors}
\label{sec:chp2_visual-editors}

Visual editors were first developed for enhancing and easing the mapping writing process. We can subdivide the proposals released over the years based on how the mapping is graphically represented in the tool: either with a tree or graph layout, or using building blocks.

The tree layout succeeded in the first approaches developed. 
The first one was developed for the R$_2$O mapping language in ODEMapster~\parencite{barrasa2006odemapster}, providing a graphical interface to visualize and edit the mappings by link the database elements with the ontology resources. 
After the release of R2RML, Karma~\parencite{gupta2012karma} and RBA (R2RML by assertion)~\parencite{neto2013rba} were developed for this language. Both work similarly to ODEMapster, with the addition of Karma providing automatic mapping suggestions. 

The graph display was later adopted by multiple editors. ~\parencite{fu2013tree-vs-graph} 

\textit{Los de grafo son los más mayoritarios y hay mas ejemplos}

\textit{Los de template son menos comunes pero hay un par de ellos, también ambos de industria y academia}

Map-On~\parencite{sicilia2017map}



\cite{sengupta2013editing} evolved into \cite{pinkel2014best}

\cite{lembo2014visualization}

RMLEditor~\parencite{heyvaert2016rmleditor}

Juma~\parencite{crotti2017juma}

Square~\parencite{blinkiewicz2016square}




\subsection{Serializations}
\label{sec:chp2_serializations}

As an alternative to visual approaches, text-based user-friendly serializations were developed. This approach suited most practitioners with technical profiles or with preferences for a text oriented environment. 
SML~\parencite{Stadler2015sml} was developed as user-friendly syntax for R2RML. It provides an SPARQL-based syntax, enhancing the simplicity for writing mappings but maintaining the same expressiveness. 
The virtual KG processor Ontop~\parencite{calvanese2017ontop} also provides a simplified serialization for R2RML, which combines the Turtle syntax for triple generation and SQL for data access. 
The Stardog triplestore developed another SPARQL-based proprietary serialization, the Stardog Mapping Syntax 2 (SMS2)\cref{foot:sms2}. 

Later on, as more mapping languages emerged, new serializations with a broader language range emerged. This is the case of YARRRML~\parencite{Heyvaert2018yarrrml}, the YAML-based user-friendly syntax created for RML. This serialization can be used with the Matey~\footnote{\url{https://rml.io/yarrrml/matey/}} to translate the RML mapping file, or directly generate RDF triples. 
XRM\cref{foot:xrm} provides a unique syntax for writing R2RML, RML and CSVW mappings. It can be used with a service plugin integrated into common text editors, Visual Studio Code and Eclipse. This service warns the users about writing errors while writing, and translates the mapping rules into one of the languages abovementioned. 




