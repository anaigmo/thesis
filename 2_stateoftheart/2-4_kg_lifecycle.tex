\section{Knowledge Graph Lifecycle \textcolor{red}{-- By 2/2}}
\label{sec:chp2_kg_lifecycle}

\ana{repasar un poco los papers que hablen de ciclo de vida de KGs, dónde encajan los mappings ahí y qué se ha estudiado en su intervención en el proceso, para decir que en la evolución no se ha estudiado}

\ana{de aquí lo que se quiere destacar es qué hueco cubrimos: el rol de declarative approaches en evolución de KG. Enmarcado en el KG life cyle, se ha estudiado cómo mejora su construcción y evaluación (randles); pues esto iría en otra etapa del lyfe cycle que no está estudiada hasta ahora}

\textit{Primera parte: un overview de los lyfe cycles/dev processes propuestos hasta ahora? Comentar las fases comunes, y decidirse por una: que sería una mezcla entre umut's y groth's. Las aprtes que interesan al final van a ser la de knowledge creation/KGconstruction; y la iteración en el mantenimiento, porque son las que involucran los declarative approaches pero se pueden comentar más.  }

\textit{continuando con la de antes: enmarcado dentro de este life cycle dónde intervendrían los mappings: en la creación y en cada iteración que haya que hacer (cuadra más groth's aquí eso sí). Señalar paper de benchmark de todas las fases en algún momento. Dentro de la creación de cero, por supuesto porque esa es su función principal. Ahí qué hay de mejora: el de bio2rdf contra php, y todas las optimizaciones del proceso, comentar si acaso del paper de vir vs mat de dylan si está publicado. }

\textit{Mirar si randles et al ha mirado algo de quality gracias a mappings; y comentar si es así. Si no se señala el gap en cómo pueden intervenir estos approaches en más fases, por ej en la evolución.}