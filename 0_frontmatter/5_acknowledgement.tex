% called by main.tex
%
\section*{Acknowledgement}
\label{sec::acknowledgement}
\addcontentsline{toc}{section}{Acknowledgement}


Al igual que no concebimos una vida en completa soledad y aislamiento, tampoco se puede entender un trabajo de tantos años sin la ayuda y la influencia de nuestro entorno. 
Y he tenido la suerte de encontrarme y estar rodeada de personas maravillosas durante el camino del doctorado, sin las cuales, no habría llegado a escribir estas palabras. 
Como hay una parte de mi en esta tesis, hay pedacitos de mucha más gente también.

Quiero empezar dando las gracias a Oscar, por darme la oportunidad de entrar en el grupo y poder realizar la tesis junto con compañeros y mentes brillantes. 
Especialmente a Oscar, por motivarme desde el principio y acompañarme hasta esta "madurez" académica. 
He aprendido mucho bajo tu dirección, a afrontar nuevos retos tanto en colaboración como independientemente (de investigación, y no ;)), siempre con optimismo.  

%Otro pilar fundamental estos años ha sido el grupo que me ha acogido y junto al que he crecido.
El camino hubiera sido infinitamente más duro si no hubiera aterrizado en el OEG, y sin todos los compañeros que me han acompañado y motivado desde el principio.
A la "antigua" generación, tanto los que están como los que ya se fueron, nuevos y antiguos profesores, María, Carlos, Patri, Pablo, Paola, Alba, Andrea, Esteban, Elvira, Elena, Raúl, Víctor, siempre referentes y siguiendo vuestros pasos. %gracias por crear un espacio tan motivador y enriquecedor de trabajo y engancharme a empezar el doctorado. 
A la "nueva" generación, que tanto me ha amenizado los últimos años, a Salva, Ibai, Diego, Clark (about time to learn spanish), Carlos, Sergio, Isam y Lucía Palacios. 
Especialmente, a las mamarrachas de Lucía Sánchez, Paula, Alejandra y Camilla. 
Habéis sido un gran apoyo en momentos que más lo necesitaba, con las largas conversaciones de las meriendas, los ánimos dando en la fibra en las crisis, las faltadas y todos los momentos que me habéis dado, gracias por hacerlo tan fácil.
A Dani, por siempre estar tan dispuesto a echar una mano, y dejarte convencer tan fácilmente tanto para papers como para cañas. 
A lo que fué del \textit{data (des)integration group}: 
A Edna, por el cariño, la paciencia y la compañía en los momentos más difíciles, 
y a Jhon, por las reuniones disfrutonas, qué harían las sesiones de post-pádel sin nosotros.
A David, por todo, desde revisarme los emails que no me atrevía a mandar, a los cafés para ponernos al día. 
Me has acompañado desde el principio, mostrándome lo motivamente que puede ser este camino, descubriéndome nuevos retos, abriéndome puertas y siempre dispuesto a ayudar y avanzar. 
Ha sido (y sigue siendo!) un placer trabajar a vuestro lado.

I also want to thank Pedro, Filip and Jay for granting me the opportunity to work at ISI and discover new horizons in research, showing me so many different perspectives; and Kian, I'll never forget the long conversations at your office, that I took as mine. 
%Also to the participants Knowledge Graph Construction Community Group, specially 
To my good colleague in Belgium, Dylan, I promise I will be a better tourist guide if you ever come back to Madrid. 
And to Anastasia, working with you is always a pleasure, thanks for showing me a different way of research and all that I have learnt from you.

Por último, quiero agradecer a mi familia y amigos, que se preguntan qué hago desde que empecé la carrera y seguirán preguntando hasta después de terminar la tesis. 
A Eva, Arturo y Xana por ser la constante en mi vida que me aterriza, y me recuerda la vida detrás de la tesis. 
A Evelyn, por todos los años de amistad a pesar lo opuestas que somos, por las noches de gyozas, los momentos de fotosíntesis en la terraza, de aventuras, de risas y dramas. 
A Laura y Patri, mis compañeras en paralelo, ¿quién nos iba a decir que iba a terminar haciendo la tesis con todo lo que me escuchasteis renegar de ella? Gracias por las risas, los viajes y siempre estar ahí. 
A mis padres, Sara y Antonio, por vuestro cariño y apoyo incondicional, gracias por haberme dado siempre alas para poder llegar a donde quisiera. 
A mi hermano Miguel, por tu energía y optimismo, por hacerme reflexionar y rebatir, hacerme ver la vida de otra forma. 
A pesar de la distancia, y lo que echo de menos sentarnos por las noches a hablar, y que vengas a molestarme a cualquier hora, nunca estás lejos. 

Gracias a todos los que estáis y los que habéis estado por motivarme a seguir cada día. 
No me olvido de ninguna las personas que han pasado por mi vida y que de una forma u otra, han sido una luz en este camino. 




