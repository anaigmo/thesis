\subsubsection{Reification with RML}

There are several reification approaches for RDF as presented in \cref{chp2_reification}, such as standard reification, singleton properties, N-ary relationships or named graphs. 
These approaches use strategies that add metadata to triples
without modifying the original RDF syntax.
Thus, they can be used with RML without any further modification. RML mapping rules enable the generation of blank nodes (required for the standard reification approach), dynamically generated predicates (required for the singleton properties approach) and named graphs (required for the named graphs approach). %\ana{si voy con todos, influye en el ejemplo de aquí abajo y en los use cases. En ese caso, quitar somef (demasiado grande el mapping, con semmeddb es suficiente)}

%\ana{probablemente en sota estarán explicados cada uno de los modelos, probablemente no haga falta ponerlos aquí también.}


\noindent\textbf{\textit{Standard Reification}}~\citep{lassila1999rdf} was proposed in the first W3C Recommendation of RDF.
It assigns statements to unique identifiers (typically blank nodes) typed with \texttt{rdf:Statement} and described using the properties \texttt{rdf:subject}, \texttt{rdf:predicate} and \texttt{rdf:object}.
This way, the unique identifier representing the statement can be further annotated with additional statements. \cref{lst:chp4_res-std-reif} shows an example of standard reification for the data in \cref{lst:chp4_csv_star}, created with the RML mapping rules in \cref{lst:chp4_std-reif}. 
This mapping creates blank nodes in the subject with the \texttt{ID} data field, typed with \texttt{rdf:Statement}; and has three predicate object maps to generate the \texttt{rdf:subject}, \texttt{rdf:predicate}, \texttt{rdf:object} of the triples (\textit{An athlete jumps certain height}) and a predicate object map to annotate the statements with \texttt{:date} (\textit{in a specific date}).




\begin{minipage}{\linewidth}
\begin{captionedlisting}{lst:chp4_std-reif}{Example RML mapping using standard reification that transforms data in \cref{lst:chp4_csv_star}.}
\centering
\begin{multicols}{2}
{\begin{lstlisting}[basicstyle=\ttfamily\small,label={list:example1},columns=flexible]
<#TM> a rr:TriplesMap ;
  rml:logicalSource :marks ;
  rr:subjectMap [ 
    rml:reference "ID" ;
    rr:termType rr:BlankNode ;
    rr:class rdf:Statement  ] ;
  rr:predicateObjectMap [ 
    rr:predicate rdf:subject ;
    rr:objectMap [
      rr:template ":{PERSON}" ] ] ;
  rr:predicateObjectMap [ 
    rr:predicate rdf:predicate ;
    rr:object :jumps ] ;
  rr:predicateObjectMap [ 
    rr:predicate rdf:object ;
    rr:objectMap [
      rml:reference "MARK" ] ] ;
  rr:predicateObjectMap [ 
    rr:predicate :date ;
    rr:objectMap [
      rml:reference "DATE" ] ] .
\end{lstlisting}}
\end{multicols}
\end{captionedlisting}
\end{minipage}

\begin{minipage}{\linewidth}
\begin{captionedlisting}{lst:chp4_res-std-reif}{RDF triples generated by the mapping in \cref{lst:chp4_std-reif}.}
\centering
\begin{multicols}{2}
{\begin{lstlisting}[basicstyle=\ttfamily\small,label={list:example1},columns=flexible]
_:1 rdf:type      rdf:Statement .
_:1 rdf:subject   :Angelica .
_:1 rdf:predicate :jumps .
_:1 rdf:object    "4.80" .
_:1 :date         "2022-03-21" .
_:2 rdf:type      rdf:Statement .
_:2 rdf:subject   :Katerina .
_:2 rdf:predicate :jumps .
_:2 rdf:object    "4.85" .
_:2 :date         "2022-03-19" .
\end{lstlisting}}
\end{multicols}
\end{captionedlisting}
\end{minipage}



\noindent\textbf{\textit{Singleton Properties}}~\citep{nguyen2014don}. This approach uses unique predicates linked with \texttt{rdf:singletonPropertyOf} to the original predicate. 
This unique predicate can then be annotated as the subject of additional statements. 
\cref{lst:chp4_res-sp-reif} shows the reified triples for the data in \cref{lst:chp4_csv_star} created with the RML mapping rules in \cref{lst:chp4_sp-reif}. 
It uses a singleton property dynamically generated with the \texttt{ID} data field for the property \texttt{:jumps} (\textit{An athlete jumps certain height}), annotated with \texttt{:date} (\textit{in a specific date}).


\begin{minipage}{\linewidth}
\begin{captionedlisting}{lst:chp4_sp-reif}{Example RML mapping using a singleton property that transforms data in \cref{lst:chp4_csv_star}.}
\centering
\begin{multicols}{2}
{\begin{lstlisting}[basicstyle=\ttfamily\small,label={list:example1},columns=flexible]
<#TM> a rr:TriplesMap ;
  rml:logicalSource :marks ;
  rr:subjectMap [ 
    rr:template ":{PERSON}" ] ;
  rr:predicateObjectMap [ 
    rr:predicateMap [
     rr:template ":jumps#{ID}" ] ;
    rr:objectMap [
      rml:reference "MARK" ] ] .
<#TM-SP> a rr:TriplesMap ;
  rr:logicalSource :marks ;
  rr:subjectMap [ 
    rr:template ":jumps#{ID}" ] ;
  rr:predicateObjectMap [ 
    rr:predicate rdf:singletonPropertyOf;
    rr:object :jumps ] ;
  rr:predicateObjectMap [ 
    rr:predicate :date ;
    rr:objectMap [
      rml:reference "DATE" ] ] .
\end{lstlisting}}
\end{multicols}
\end{captionedlisting}
\end{minipage}



\noindent\hspace{0.12\linewidth}\begin{minipage}{\linewidth}
\begin{captionedlisting}{lst:chp4_res-sp-reif}{RDF triples generated by the mapping in \cref{lst:chp4_sp-reif}.}
\centering
\begin{tabular}{c}
\hspace{4em}
{\begin{lstlisting}[basicstyle=\ttfamily\small,label={list:example1},columns=flexible]
:Angelica   :jumps#1  "4.80" .
:jumps#1    :date     "2022-03-21" .
:jumps#1    rdf:singletonPropertyOf :jumps .
:Katerina   :jumps#2  "4.85" .
:jumps#2    :date     "2022-03-19" .
:jumps#2    rdf:singletonPropertyOf :jumps .
\end{lstlisting}}
\end{tabular}
\end{captionedlisting}
\end{minipage}



\textit{\textbf{Named Graphs}}~\citep{carroll2005namedgraphs}. This approach encloses each reified triple inside a unique named graph. This named graph can then be annotated with an additional triple. In \cref{lst:chp4_graph-reif} a graph is created for the triples that state the height of the jump (\texttt{:G} with the field \texttt{ID}) (\textit{An athlete jumps certain height}). A second graph contains the annotated triple and references the first graph (\texttt{:GA} with the field \texttt{ID}) (\textit{in a specific date}). The resulting RDF triples in TriG syntax\footnote{\url{https://www.w3.org/TR/trig/}} are shown in \cref{lst:chp4_res-graph-reif}.

\begin{minipage}{\linewidth}
\begin{captionedlisting}{lst:chp4_graph-reif}{Example RML mapping using named graphs that transforms data in \cref{lst:chp4_csv_star}.}
\centering
\begin{multicols}{2}
{\begin{lstlisting}[basicstyle=\ttfamily\small,label={list:example1},columns=flexible]
<#TM> a rr:TriplesMap ;
  rml:logicalSource :marks ;
  rr:subjectMap [ 
    rr:template ":{PERSON}" ;
    rr:graphMap [
      rr:template ":G{ID}" ] ] ;
  rr:predicateObjectMap [ 
    rr:predicateMap [
     rr:template ":jumps" ] ;
    rr:objectMap [
      rml:reference "MARK" ] ] .
<#TM-GA> a rr:TriplesMap ;
  rr:logicalSource :marks ;
  rr:subjectMap [ 
    rr:template ":{ID}" ;
    rr:graphMap [
      rr:template ":GA{ID}"] ] ;
  rr:predicateObjectMap [ 
    rr:predicate :date ;
    rr:objectMap [
      rml:reference "DATE" ] ] .
\end{lstlisting}}
\end{multicols}
\end{captionedlisting}
\end{minipage}

\noindent\hspace{0.12\linewidth}\begin{minipage}{\linewidth}
\begin{captionedlisting}{lst:chp4_res-graph-reif}{RDF triples generated by the mapping in \cref{lst:chp4_graph-reif}.}
\centering
\begin{tabular}{c}
\hspace{4em}
{\begin{lstlisting}[basicstyle=\ttfamily\small,label={list:example1},columns=flexible]
:G1  {:Angelica :jumps  "4.80" .}
:GA1 {:G1       :date   "2022-03-21" .}
:G2  {:Katerina :jumps#2  "4.85" .}
:GA2 {:G2       :date     "2022-03-19" .}
\end{lstlisting}}
\end{tabular}
\end{captionedlisting}
\end{minipage}


\textit{\textbf{N-Ary Relationships}}~\citep{naryw3c2006}. This approach  converts a relationship into an instance that describes the relation, which can have attached both the main object and additional statements. The mapping shown in \cref{lst:chp4_nary-reif} creates instances of the relationship ``jump" with the field \texttt{ID}, which contain the date and height of the jump.  (\textit{An athlete jumps a jump with certain height and in a specific date}). 
The resulting RDF triples are shown in \cref{lst:chp4_res-nary-reif}.

\begin{minipage}{\linewidth}
\begin{captionedlisting}{lst:chp4_nary-reif}{Example RML mapping using a N-ary relationship that transforms data in \cref{lst:chp4_csv_star}.}
\centering
\begin{multicols}{2}
{\begin{lstlisting}[basicstyle=\ttfamily\small,label={list:example1},columns=flexible]
<#TM> a rr:TriplesMap ;
  rml:logicalSource :marks ;
  rr:subjectMap [ 
    rr:template ":{PERSON}" ] ;
  rr:predicateObjectMap [ 
    rr:predicateMap [
     rr:template ":jumps" ] ;
    rr:objectMap [
      rr:template ":Jump{ID}";
      rr:termType rr:IRI] ] .
<#TM-JUMP> a rr:TriplesMap ;
  rr:logicalSource :marks ;
  rr:subjectMap [ 
    rr:template ":Jump{ID}" ] ;
  rr:predicateObjectMap [ 
    rr:predicate :height ;
    rr:objectMap [
      rml:reference "MARK" ] ] ;
  rr:predicateObjectMap [ 
    rr:predicate :date ;
    rr:objectMap [
      rml:reference "DATE" ] ] .
\end{lstlisting}}
\end{multicols}
\end{captionedlisting}
\end{minipage}

\begin{minipage}{\linewidth}
\begin{captionedlisting}{lst:chp4_res-nary-reif}{RDF triples generated by the mapping in \cref{lst:chp4_nary-reif}.}
\centering
\begin{multicols}{2}
{\begin{lstlisting}[basicstyle=\ttfamily\small,label={list:example1},columns=flexible]
:Angelica  :jumps   :Jump1 .
:Jump1     :date    "2022-03-21" .
:Jump1     :height  "4.80" .
:Katerina  :jumps   :Jump2 .
:Jump2     :date    "2022-03-19" .
:Jump2     :height  "4.85" .
\end{lstlisting}}
\end{multicols}
\end{captionedlisting}
\end{minipage}