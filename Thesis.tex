                
% ----------------------------------------------------------------------
%                   LATEX TEMPLATE FOR PhD THESIS
% ----------------------------------------------------------------------

% based on Harish Bhanderi's PhD/MPhil template, then Uni Cambridge
% http://www-h.eng.cam.ac.uk/help/tpl/textprocessing/ThesisStyle/
% corrected and extended in 2007 by Jakob Suckale, then MPI-CBG PhD programme
% and made available through OpenWetWare.org - the free biology wiki


%: Style file for Latex
% Most style definitions are in the external file PhDthesisPSnPDF.
% In this template package, it can be found in ./Latex/Classes/
\documentclass[twoside,11pt,table,xcdraw]{Latex/Classes/PhDthesisPSnPDF}

\newcommand{\ana}[2][inline]{\color{red} [AI]: #2\color{black}}

\usepackage[utf8]{inputenc}
\usepackage[english]{babel}
\usepackage{lipsum}  
\usepackage{multirow}
\usepackage{xcolor}
\usepackage{rotating}
\usepackage{caption}
\usepackage{lscape}
\usepackage{hvfloat}
%\usepackage[table,xcdraw]{xcolor}
\usepackage{balance}

\usepackage[ruled,vlined]{algorithm2e}

\usepackage[normalem]{ulem}
\useunder{\uline}{\ul}{}
\usepackage{color}
\usepackage{makecell}
\usepackage{tabu, booktabs}


\usepackage{listings,chngcntr}
\def\dollar{\$}
\lstset{mathescape=true}
\lstset{%
	basicstyle=\scriptsize\ttfamily,%
	basewidth=.5em,%
	captionpos=b,
	frame=single,
	tabsize=4,
	numbers=left}%
\lstset{
	escapechar={(@}{^)}
}
\lstset{
   literate=
        {í}{{\'i}}1
        {ñ}{{\~n}}1
}
%% -- Listings

\colorlet{punct}{red!60!black}
%\definecolor{background}{HTML}{EEEEEE}
%\definecolor{delim}{RGB}{20,105,176}
\colorlet{numb}{magenta!60!black}
\definecolor{burgundy}{rgb}{0.5, 0.0, 0.13}
\lstdefinelanguage{json}{
	keywords=[1]{},
	keywordstyle=[1]\color{teal}\bfseries,
	keywords=[2]{resource_rules, id, datasource_ids, subject, properties, predicate, object, is_literal, link_rules, condition, source, target},	% types; money and time units
	keywordstyle=[2]\color{burgundy}\bfseries,
    basicstyle=\footnotesize\ttfamily,
    numbers=left,
    numberstyle=\scriptsize,
    stepnumber=1,
    numbersep=8pt,
    showstringspaces=false,
    identifierstyle=\color{black},
    breaklines=true,
    %backgroundcolor=\color{verylightgray},
    literate=
     *{:}{{{\color{black}{:}}}}{1}
      {\$}{{{\color{punct}{\$}}}}{1}
      {,}{{{\color{red}{,}}}}{1}
      {\{}{{{\color{red}{\{}}}}{1}
      {\}}{{{\color{red}{\}}}}}{1}
      {[}{{{\color{red}{[}}}}{1}
      {]}{{{\color{red}{]}}}}{1},
}
\definecolor{bittersweet}{rgb}{1.0, 0.44, 0.37}
\lstdefinelanguage{Shex}{
	keywords=[1]{},
	keywordstyle=[1]\color{teal}\bfseries,
	keywords=[2]{SOURCE, ITERATOR, FIELD},	% types; money and time units
	keywordstyle=[2]\color{bittersweet}\bfseries,
	keywords=[3]{c_city, cities_rdb, coord_json, it_cities, population, year, zipcode, it_coord, lat, long, loc_city, long},
	keywordstyle=[3]\color{teal}\bfseries,
	identifierstyle=\color{black},
	sensitive=true,
	comment=[l]{//},
	morecomment=[s]{/*}{*/},
	commentstyle=\color{black}\ttfamily,
	stringstyle=\color{fulvous}\bfseries,
	morestring=[b]',
	morestring=[b]",
	literate=
     *{\{}{{{\color{bittersweet}{\{}}}}{1}
      {\}}{{{\color{bittersweet}{\}}}}}{1}
      {<}{{{\color{bittersweet}{<}}}}{1}
      {>}{{{\color{bittersweet}{>}}}}{1},
}

\lstdefinelanguage{yarrrml}{
	keywords=[1]{},
	keywordstyle=[1]\color{teal}\bfseries,
	keywords=[2]{mappings, sources, s, po, function, parameters, value, parameter},	% types; money and time units
	keywordstyle=[2]\color{violet}\bfseries,
	keywords=[3]{Locations, Cities},
	keywordstyle=[3]\color{teal}\bfseries,	
	keywords=[4]{xsd, double, integer, boolean, decimal, string},
	keywordstyle=[4]\color{applegreen}\bfseries,
	identifierstyle=\color{black},
	sensitive=true,
	comment=[l]{//},
	morecomment=[s]{/*}{*/},
	commentstyle=\color{black}\ttfamily,
	stringstyle=\color{fulvous}\bfseries,
	morestring=[b]',
	morestring=[b]",
	literate=
     *{[}{{{\color{violet}{[}}}}{1}
      {]}{{{\color{violet}{]}}}}{1}
       {-}{{{\color{violet}\bfseries{-}}}}{1},
}

\lstdefinelanguage{sparql}{
	keywords=[1]{} % generic keywords including crypto operations
	keywordstyle=[1]\color{blue}\bfseries,
	keywords=[2]{},	% types; money and time units
	keywordstyle=[2]\color{teal}\bfseries,
	keywords=[3]{block, blockhash, coinbase, difficulty, gaslimit, number, timestamp, msg, data, gas, sender, sig, value, now, tx, gasprice, origin},	% environment variables
	keywordstyle=[3]\color{violet}\bfseries,
	keywords=[4]{xsd, double, integer, boolean, decimal, string},
	keywordstyle=[4]\color{applegreen}\bfseries,
	keywordstyle=[5]\color{iris}\bfseries,
	keywords=[5]{GENERATE, REPLACE, WHERE, FILTER, BIND, CONSTRUCT, GRAPH},
	identifierstyle=\color{black},
	sensitive=false,
	comment=[l]{//},
	morecomment=[s]{/*}{*/},
	commentstyle=\color{gray}\ttfamily,
	stringstyle=\color{fulvous}\bfseries,
	morestring=[b]',
	morestring=[b]"
}

\lstdefinelanguage{concm}{
	keywords=[1]{} % generic keywords including crypto operations
	keywordstyle=[1]\color{blue}\bfseries,
	keywords=[2]{Locations, Cities, Sources, Rules},	% types; money and time units
	keywordstyle=[2]\color{violet}\bfseries,
	keywords=[3]{cm, dcat, expression, hasField, hasDataSource, mediaType, SourceFrame, accessService, hasProtocol, endpointURL, hasSecurityScheme, DataField, field, hasNestedFrame, ReferenceNodeMap, hasEvaluableExpression, hasFunctionName, hasInput, subject, predicate, hasFrame, constantValue, Literal, object, FunctionExpression, functionName, hasBooleanCondition, hasDatatype, source, target, property, LinkingExpression, ConditionalStatementMap, SynchronousSource, StatementMap, JoinCombination, datatype, language, combinesFrame, joinsBy, ListMap, hasLanguage},	% environment variables
	keywordstyle=[3]\color{steelblue}\bfseries,
	keywords=[4]{xsd, double, integer, boolean, decimal, string, time},
	keywordstyle=[4]\color{applegreen}\bfseries,
	identifierstyle=\color{black},
	sensitive=false,
	comment=[l]{//},
	morecomment=[s]{/*}{*/},
	commentstyle=\color{gray}\ttfamily,
	stringstyle=\color{fulvous}\bfseries,
	morestring=[b]',
	morestring=[b]",
}

\definecolor{verylightgray}{rgb}{.97,.97,.97}
\definecolor{applegreen}{rgb}{0.55, 0.71, 0.0}
\definecolor{iris}{rgb}{0.35, 0.31, 0.81}
\definecolor{richelectricblue}{rgb}{0.03, 0.57, 0.82}
\definecolor{shamrockgreen}{rgb}{0.0, 0.62, 0.38}
\definecolor{fulvous}{rgb}{0.86, 0.52, 0.0}
\definecolor{steelblue}{rgb}{0.27, 0.51, 0.71}

\lstdefinelanguage{Solidity}{
	%keywords=[1]{} % generic keywords including crypto operations
	%keywordstyle=[1]\color{blue}\bfseries,
	%keywords=[2]{CitiesSource, LocationSource, Locations, Cities},	% types; money and time units
	%keywordstyle=[2]\color{richelectricblue}\bfseries,
	%keywords=[3]{rr, xrr, logicalSource, subjectMap, predicateObjectMap, tableName, predicate, object, objectMap, template, reference, class, datatype, TriplesMap, LogicalTable, termType, rml, BlankNode,  predicateMap, NonAssertedTriplesMap, quotedTriplesMap},	% environment variables
	%keywordstyle=[3]\color{shamrockgreen}\bfseries,
	%keywords=[4]{xsd, double, integer, boolean, decimal, string, time},
	%keywordstyle=[4]\color{applegreen}\bfseries,
	identifierstyle=\color{black},
	sensitive=false,
	comment=[l]{//},
	morecomment=[s]{/*}{*/},
	commentstyle=\color{gray}\ttfamily,
	%stringstyle=\color{fulvous}\bfseries,
	morestring=[b]',
	morestring=[b]",
}

\lstset{
	language=Solidity,
	%backgroundcolor=\color{verylightgray},
	extendedchars=true,
	basicstyle=\footnotesize\ttfamily,
	showstringspaces=false,
	showspaces=false,
	numbers=left,
	numberstyle=\footnotesize,
	numbersep=9pt,
	tabsize=2,
	breaklines=true,
	showtabs=false,
	captionpos=b,
	frame=none
}

%% --

\usepackage{xparse}
\newcounter{captionedlistingcounter}
\NewDocumentEnvironment{captionedlisting}{mmm}{
	\refstepcounter{captionedlistingcounter}
	% commented out the figure part bc we don't need it and they couldn't be used in minipages
	% not sure if this is actually different from using the default listing captions/labels now
	%\begin{figure}[#3]
	{\centering\label{#1}}

}{

	\vspace{0.5\baselineskip}
	Listing \thecaptionedlistingcounter: #2
	\\%\end{figure}
}

\usepackage[capitalize,nameinlink]{cleveref}
\crefname{line}{line}{lines}
\Crefname{Line}{Line}{Lines}
\crefname{captionedlistingcounter}{Listing}{Listings}
\Crefname{captionedlistingcounter}{Listing}{Listings}
\crefformat{footnote}{#2\footnotemark[#1]#3}

%: Macro file for Latex
% Macros help you summarise frequently repeated Latex commands.
% Here, they are placed in an external file /Latex/Macros/MacroFile1.tex
% An macro that you may use frequently is the figuremacro (see introduction.tex)
%\include{Latex/Macros/MacroFile1}



%: ----------------------------------------------------------------------
%:                  TITLE PAGE: name, degree,..
% ----------------------------------------------------------------------
% below is to generate the title page with crest and author name

%if output to PDF then put the following in PDF header
\ifpdf  
    \pdfinfo { /Title 
               /Creator (TeX)
               /Producer (pdfTeX)
               /Author (Ana Iglesias Molina ana.iglesiasm@upm.es)
               /CreationDate (D:YYYYMMDDhhmmss)  %format D:YYYYMMDDhhmmss
               /ModDate (D:YYYYMMDDhhmm)
               /Subject (xyz)
               /Keywords () }
    \pdfcatalog { /PageMode (/UseOutlines)
                  /OpenAction (fitbh)  }
\fi

  
\title{Knowledge Graph Construction and Evolution using Declarative Mapping Languages}

% The role of mappings in Knowledge Graph construction and evolution



% ----------------------------------------------------------------------
% The section below defines www links/email for author and institutions
% They will appear on the title page of the PDF and can be clicked
\ifpdf
 
  \author{{\hspace{7mm} Ana Iglesias Molina}}
  \supervisor{Dr. Oscar Corcho}
	%
	
%  \cityofbirth{born in XYZ} % uncomment this if your university requires this
%  % If city of birth is required, also uncomment 2 sections in PhDthesisPSnPDF
%  % Just search for the "city" and you'll find them.
%  \collegeordept{{Programa de Doctorado de Inteligencia Artificial \\ Facultad de Inform\'atica}}  
%  \university{\href{http://www.upm.es}{Universidad Polit\'ecnica de Madrid}}  
  % The crest is a graphics file of the logo of your research institution.
  % Place it in ./0_frontmatter/figures and specify the width
%  \crest{\includegraphics[width=3cm]{escudofi.pdf}}
\crest{  
		\centering
 		\begin{tabular}{l c}
			\multirow{3}{*}{\includegraphics[width=2cm]{0_frontmatter/figures/escudofi.pdf}} & \\
			& \textbf{Programa de Doctorado de Inteligencia Artificial} \\ 
			& \textbf{Escuela T\'ecnica Superior de Ingenieros Inform\'aticos} \\
		\end{tabular}
}
% If you are not creating a PDF then use the following. The default is PDF.
\else
  \author{YourName}
%  \cityofbirth{born in XYZ}
  \collegeordept{CollegeOrDept}
  \university{University}
  \crest{\includegraphics[width=4cm]{logo}}
\fi

%\renewcommand{\submittedtext}{change the default text here if needed}
\degree{Philosophi\ae Doctor (PhD), DPhil,..}
\degreedate{XXX, 2023}


% ----------------------------------------------------------------------
       
% turn of those nasty overfull and underfull hboxes
%\tolerance=1
%\emergencystretch=\maxdimen
%\hyphenpenalty=10000






%% Sections

\newcommand{\lsection}[2]{
    \section{#2}
    \label{sec:#1}
}

\newcommand{\lsubsection}[2]{
    \subsection{#2}
    \label{sec:#1}
}

\newcommand{\lsubsubsection}[2]{
    \subsubsection{#2}
    \label{sec:#1}
}

\newcommand{\lsecref}[1]{\ref{sec:#1}}

\newcommand{\csubfloat}[2][]{%
  \makebox[0pt]{\subfloat[#1]{#2}}%
}
\newcommand{\centerhfill}[1][\quad]{\hspace{\stretch{0.5}}#1\hspace{\stretch{0.5}}}

\makeatletter
\newcommand\resetstackedplots{
\makeatletter
\pgfplots@stacked@isfirstplottrue
\makeatother
\addplot [forget plot,draw=none] coordinates{(1,0) (2,0) (3,0)};
}

%% /PIERRE

%: --------------------------------------------------------------
%:                  FRONT MATTER: dedications, abstract,..
% --------------------------------------------------------------

\begin{document}
\counterwithin{captionedlistingcounter}{chapter}

%\language{english}

% sets line spacing
\renewcommand\baselinestretch{1.2}
\baselineskip=18pt plus1pt

%\theoremstyle{plain}
%\newtheorem{thm}{Theorem}[chapter] % reset theorem numbering for each chapter


%\theoremstyle{definition}
%\newtheorem{definition}[thm]{Definition} % definition numbers are dependent on theorem numbers

\newcommand{\attention}[1]{{\color{red}\textbf{#1}}}

\renewcommand\appendixname{ANNEX}


%: ----------------------- generate cover page ------------------------
\frontmatter
\maketitle  % command to print the title page with above variables


%: ----------------------- cover page back side ------------------------
% Your research institution may require reviewer names, etc.
% This cover back side is required by Dresden Med Fac; uncomment if needed.

% ALVARO: I'll have to change this

%\newpage
\pagestyle{plain}
\cleardoublepage
\pagestyle{plain}


\noindent Tribunal nombrado por el Sr. Rector Magfco. de la Universidad Polit\'{e}cnica de
Madrid, el d\'{i}a TBD9 de Junio de 2021. %\mynote{DD de YYYY de 2015}

\vspace{10mm}
Presidente:\hspace{0.3mm} TBD%Dr. Marcelo Arenas

\vspace{5mm}
Vocal: \hspace{6.7mm} TBD%Dr. Juan Sequeda 

\vspace{5mm}
Vocal: \hspace{6.7mm} TBD%DrDra. Anastasia Dimou

\vspace{5mm}
Vocal: \hspace{6.7mm} TBD%DrDr. Álvaro Sicilia Gómez 


\vspace{5mm}
Secretario:\hspace{0.67mm} TBD%DrDr. Raúl García Castro

\vspace{5mm}
Suplente: \hspace{1.5mm} TBD%DrDr. Mariano Fernández López   

\vspace{5mm}
Suplente: \hspace{1.5mm} TBD%DrDr. Alberto Bugarín Díz

\vspace{10mm}
\noindent Realizado el acto de defensa y lectura de la Tesis el d\'{i}a \textcolor{red}{28 de Junio de 2021} en la Escuela T\'ecnica Superior de Ingenieros Inform\'aticos

\vspace{5mm}
\noindent Calificaci\'{o}n: \rule{123mm}{0.2mm}
\vspace{20mm}

EL PRESIDENTE \hspace{30mm} VOCAL 1 \hspace{30mm} VOCAL 2

\vspace{30mm}
%\begin{center}
\hspace{15mm} VOCAL 3 \hspace{45mm} EL SECRETARIO
%\end{center}




%: ----------------------- tie in front matter ------------------------

%\frontmatter
\cleardoublepage
\input{0_frontmatter/dedication}
\cleardoublepage
\input{0_frontmatter/acknowledgement}


%: ----------------------- abstract ------------------------

% Your institution may have specific regulations if you need an abstract and where it is to be placed in the document. The default here is just after title.
\cleardoublepage
\input{0_frontmatter/abstract}

% The original template provides and abstractseparate environment, if your institution requires them to be separate. I think it's easier to print the abstract from the complete thesis by restricting printing to the relevant page.
% \begin{abstractseparate}
%   \input{Abstract/abstract}
% \end{abstractseparate}


%: ----------------------- contents ------------------------
\cleardoublepage
\setcounter{secnumdepth}{3} % organisational level that receives a numbers
\setcounter{tocdepth}{2}    % print table of contents for level 3
\tableofcontents           % print the table of contents
% levels are: 0 - chapter, 1 - section, 2 - subsection, 3 - subsection


%: ----------------------- list of figures/tables ------------------------

\listoffigures	% print list of figures

\listoftables  % print list of tables

\listofalgorithms % Print list of algorithms


%: ----------------------- glossary ------------------------

% Tie in external source file for definitions: /0_frontmatter/glossary.tex
% Glossary entries can also be defined in the main text. See glossary.tex

%\include{0_frontmatter/glossary} 
%
%\begin{multicols}{2} % \begin{multicols}{#columns}[header text][space]
%\begin{footnotesize} % scriptsize(7) < footnotesize(8) < small (9) < normal (10)
%
%\printnomenclature[1.5cm] % [] = distance between entry and description
%\label{nom} % target name for links to glossary
%
%\end{footnotesize}
%\end{multicols}



%: --------------------------------------------------------------
%:                  MAIN DOCUMENT SECTION
% --------------------------------------------------------------

% the main text starts here with the introduction, 1st chapter,...
\mainmatter

%\renewcommand{\chaptername}{} % uncomment to print only "1" not "Chapter 1"
%\titleformat{\chapter}[display]{\normalfont\bfseries\}{\chaptertitlename\ \thechapter}{5pt}{\Huge}

%-#####-> Uncomment to have Andres' style for chapter title

%\titleformat{\chapter}[display]
%{\bfseries\Large}
%{\filleft\MakeUppercase{\chaptertitlename} \Huge\thechapter}
%{4ex}
%{\titlerule
%\vspace{2ex}%
%\filright}
%[\vspace{2ex}%
%\titlerule]

%-#####-> End uncomment

%"In a world of infinite content, knowledge becomes valuable" Denny Vrandečić

\chapter{Introduction}
\label{chapter:intro}

%\textit{intro sobre la web, cómo surgió web semantica?}


%\textit{* General context about knowledge graphs, they are everywhere, representing every kind of information, used for many different purposes. }

%\ana{empezar introduciendo RDF y semtech a más alto nivel antes de pasar a KGs}

The nature of the World Wide Web has made it possible for every user across the world to publish and access data on a global scale. 
The size of the data available on the web has increased hand-in-hand with its heterogeneity. 
These data are often provided in different formats (e.g., plain text, CSV or JSON) and with diverse entry points, (e.g., data portals, APIs, or databases). 
This diversity makes it increasingly difficult to enable large-scale data access and processing for both humans and machines.
In this regard, the semantic web~\parencite{berners2001semantic} was envisioned to represent information in the form of knowledge graphs that can be automatically manipulated, becoming a remarkably active research field in the last decades.
Multiple recommendations have been released from the World Wide Web Consortium (W3C) to model data in a uniform model (RDF~\parencite{rdf}), querying (SPARQL~\parencite{harris2013sparql}), constructing from other data sources (R2RML~\parencite{das2012r2rml}) and validating against a set of restrictions (SHACL~\parencite{SHACL}) among others. 

%Knowledge graphs (KGs) can be defined as ``graphs of data intended to accumulate and convey knowledge of the real world, whose nodes represent entities of interest and whose edges represent relations between these entities"~\parencite{hogan2021kg}. 
Knowledge graphs integrate and convey real-world knowledge, representing entities and how they relate to each other~\parencite{hogan2021kg}.
They are compliant with the W3C recommendations, and adequate for downstream consumption applications.
%Over the last few decades, knowledge graphs have gained momentum, enabling the community to structure, exploit and publish data on the web. 
There are countless success stories of knowledge graphs being currently used, from publicly available and representing common knowledge or domain-specific knowledge, to privately owned graphs. 
Largely known and widely used Open Knowledge Graphs include examples containing common knowledge, such as Wikidata~\parencite{erxleben2014introducing}, DBPedia~\parencite{lehmann2015dbpedia}, YAGO~\parencite{pellissier2020yago} or Freebase~\parencite{bollacker2007freebase}; 
%, EncycNet\footnote{\url{https://encycnet.github.io/}}, 
and about a specific domain, such as 
libraries~\parencite{vila2013datos}, 
scientific articles~\parencite{stocker2023orkg}, 
tourism~\parencite{karle2018building,alonso2018rioja}, 
geography~\parencite{stadler2012linkedgeodata}, 
cultural heritage~\parencite{carriero2019arco}, 
life sciences~\parencite{dumontier2014bio2rdf,pinero2020disgenet}, among many others. 
Not only within the scientific community, but also companies realized their potential and are taking advantage of it. Therefore, companies such as
Google\footnote{\url{https://blog.google/products/search/introducing-knowledge-graph-things-not/}},
Microsoft~\parencite{farber2019microsoft},
LinkedIn\footnote{\url{https://engineering.linkedin.com/blog/2016/10/building-the-linkedin-knowledge-graph}}, 
Pinterest~\parencite{goncalves2019pinterest},
produce their own knowledge graphs. 

%\textit{* General context about how to create KGs: code libraries, tools like openrefine, and mappings. Then on and on with mappings, how they have evolved blabla, increasing support by different engines blabla}

%\textcolor{red}{The adoption of KGs comes hand in hand with the ease of constructing them by the community~\parencite{hogan2020twodecades,karger2014semantic}. }
The wide-spread use of KGs has incurred in a high variety of ways for constructing them: requiring a different level of manual effort, suitable for diverse data sizes and formats, from (semi)structured to unstructured. 
Within the systematic construction from semi-structured data~\parencite{Poggi2008}, there are likewise multiple manners to carry out the task.
One approach consists of using coding libraries, such as rdflib\footnote{\url{https://rdflib.readthedocs.io/en/stable/}} for Python or Jena\footnote{\url{https://jena.apache.org/}} for Java. 
This method brings benefits regarding flexibility to address data heterogeneity, processing and generation. 
However, it supposes an ad-hoc approach restricted for users with specific technical background, and can hinder the maintainability, understandability and reusability of the construction resources~\parencite{iglesias2019bio2rdf}.

Not only practitioners with technical profiles are desired to generate KGs~\parencite{karger2014semantic}. 
To widen the extent of users to make their construction possible, different visual interactive interfaces and methods were developed. 
For instance, OpenRefine\footnote{\url{https://openrefine.org/}} comprises a framework to upload tabular data and manually edit them to generate RDF datasets; while \url{Semantify.it}~\parencite{karle2017semantifyit} provides a user-friendly editor to manually annotate data to create structured content with \url{Schema.org}~\parencite{guha2016schema.org} for webpages. 
One setback of this kind of editors is that they cannot usually manage large data sizes\footnote{\url{https://github.com/OpenRefine/openrefine.org/issues/136}}~\parencite{petrova2020data}. 
In addition, similarly to the previous approach, providing an ad-hoc manual environment for performing the transformations runs against the reproducibility, maintainability and automation of the process. 
%Despite this fact, this option usually results the most suitable for new users or practitioners with non-technical backgrounds, as they are designed to be intuitive and easy to use. \ana{buscar cita}

%\textit{* Mappping-compliant technologies are increasingly proving to be convenient/useful/good for KGs: maintainability, understandability, engines already optimized, no need to code (accessibility for wider range of profiles). }

A compromise between these two approaches emerged with declarative mapping languages. 
These mappings define the rules that hold of the correspondences between source data and a target ontology to create a knowledge graph. 
The main milestone in the progress of these technologies came with the standardization of the RDB2RDF Mapping Language (R2RML) in 2012~\parencite{das2012r2rml}. 
This language, based on previous efforts such as XLWrap~\parencite{xlwrap}, D2RQ~\parencite{bizer2004d2rq} or R2O~\parencite{barrasa2004r2o}, focused on describing the transformation of data in relational databases (RDBs). 
As this language was adopted and used in real-world scenarios, its limitations became evident. 
Starting from the limited input data format it is able to describe (RDBs), the first extensions appeared to extend the scope for more heterogeneous data formats (e.g. RML~\parencite{Dimou2014rml}, xR2RML~\parencite{michel2015xr2rml}). As R2RML relies on SQL queries and views for performing data transformation functions over the original data, this possibility is lost with other data formats. For this reason, other extensions appeared allow data transformation description in the language (e.g. FunUL~\parencite{junior2016funul}, KR2RML~\parencite{slepicka2015kr2rml}). Likewise, with new needs and requirements, additional extensions and languages were developed subsequently (e.g. RML-Target~\parencite{VanAssche2021LeveragingWebThings}, 
SPARQL-Generate~\parencite{Lefrancois2017sparqlgenerate}, 
%Helio~\parencite{cimmino2022helio}, 
ShExML~\parencite{Garcia-Gonzalez2020shexml}, 
SPARQL-Anything~\parencite{asprino2023sparql-anything}).
The use of mapping languages over this decade has proven to be a suitable approach for semi-automated KG construction, maintainable in the long term and scalable for large data sizes~\parencite{vidal2023knowledge,iglesias2023scaling,xiao2020virtual,iglesias2022empowering}.  
%\textcolor{red}{Although it was increasingly adopted and used in real-world scenarios, its limitations became evident.} 
%This triggered the release of a considerable amount of extensions (e.g. RML~\parencite{Dimou2014rml}, xR2RML~\parencite{michel2015xr2rml}, FunUL~\parencite{junior2016funul}, KR2RML~\parencite{slepicka2015kr2rml})
%and new languages (e.g., %Tarql\footnote{\url{https://tarql.github.io/}}, 
%SPARQL-Generate~\parencite{Lefrancois2017sparqlgenerate}, 
%Helio~\parencite{cimmino2022helio}, 
%ShExML~\parencite{Garcia-Gonzalez2020shexml}, 
%SPARQL-Anything~\parencite{asprino2023sparql-anything}), 
%capable of addressing a wider heterogeneity of data and use cases with multiple additional features. 
%The use of mapping languages over this decade has proven to be a suitable approach for semi-automated KG construction, maintainable in the long term and scalable for large data sizes~\parencite{vidal2023knowledge,iglesias2023scaling,xiao2020virtual,iglesias2022empowering}.  %with a vibrant community of users actively supporting it\footnote{\url{https://www.w3.org/community/kg-construct/}}. 

%\textit{However, data is still complex, the requirements for KG evolve and these languages are sometimes limited. **Then first reference to a whole: gather which are these necessities**}

 Hence, mapping languages provide a means (usually a vocabulary or syntax) to describe these transformation rules declaratively in a file. 
 All languages share a common core of characteristics: (i) the description of the input data, and (ii) specification on how to generate the output triples in the graph according to the schema of an ontology. 
 The extent, level of detail and variability on how to perform this task relies on the heterogeneity of languages proposed. 
 They were developed from different needs and use cases, resulting in diverse features and hence, providing a wide set of possibilities for users.
 Knowing the capabilities of each language can help users decide which one to use depending on their needs
 Moreover, despite the efforts over the years improving and refining the languages, there are still use cases that cannot be solved using them. Understanding their limitations and open issues to address these uncovered use cases can help develop further features for an enhanced KG construction process.
 However, \textbf{there is a lack of comprehensive and extensive studies analyzing the expressiveness of mapping languages in fine-grain detail}, which can greatly help in (i) the choice of language depending on the use case requirements, and (ii) identifying the challenges and open issues to improve further the languages. 
 
 %However, \textbf{studies analyzing mapping languages lack some currently used languages and depth in feature analysis}, which can help understand the current needs in KG construction, facilitate their update with so far unsupported needs and help to choose among them depending on each use case requirements.
 %\textbf{there is no study collecting and analyzing the current needs for knowledge graph construction}, which can facilitate the update of existing languages to address them. 

%% PÁRRAFO DE RDF-STAR ---> DESCARTADO POR AHORA
%Despite the variety of extensions released over the years that progressively addressed more limitations, the needs in KG construction keep evolving with the progress of related technology. This is the case, for instance, of RDF-star. This proposal~\parencite{hartig2017foundations} introduced recursiveness in RDF statements, to empower triples to become subjects and/or objects of other triples, offering a compact syntax for RDF statement reification. At the time of writing, RDF-star is becoming part of the RDF 1.2 specification~\parencite{hartig2023rdf}. This triggers the need to \textbf{update current mapping languages to be able to describe the generation of RDF-star graphs}.

%\ana{por aquí un párrafo de que tienen limiraciones, tanto en cosas que no podían hacer por complejidad del caso de uso que se ha ido viendo con los años y el uso; tanto como porque el mundo alrededor evoluciona y surgen nuevas necesidades, como la de incluir RDF-star porque va a formar parte d ela nueva spec de RDF}

%\textit{Then about how to write mappings, still hard for many people who find it difficult or blabla, how to increase the adoption by making the user experience better: visual approaches do not scale and blabla. user friendly serializations are popular, we keep on updating them with needs already identified (above), and develop a new approach that can be for bot domain experts and experienced practitioners.}

Mapping files are processed by a compliant engine along with the input data to construct KGs. 
Then, users are not required to have coding skills, but in turn, they need to learn the mapping language. 
Most mapping languages are defined as an ontology (e.g. R2RML and its extensions). 
Therefore, they can use RDF serializations, usually Turtle~\parencite{turtle}, as the syntax to write the file.
Other popular approaches use SPARQL~\parencite{harris2013sparql} as the basis, extending the language to tackle input data description (e.g. SPARQL-Generate~\parencite{Lefrancois2017sparqlgenerate}, SPARQL-Anything~\parencite{asprino2023sparql-anything}, Tarql\footnote{\url{https://tarql.github.io/}}). 
For practitioners with a background in semantic technologies the barrier is lower to learn the mapping languages, as they are more prone to have previous knowledge about these syntaxes. 
Without this advantage, the learning curve for any other user is increased. 
For this reason, to help in the adoption of these technologies and make them more accessible for non-expert users, several different approaches were develop, ranging from visual approaches (e.g. Karma~\parencite{gupta2012karma}, RMLEditor~\parencite{heyvaert2016rmleditor}) to user-friendly syntaxes (e.g. YARRRML~\parencite{Heyvaert2018yarrrml}, SMS2~\parencite{sms2}). 
While the latter have shown a wider adoption by more experienced practitioners, new users still have difficulties learning them. 
In addition, visual approaches are limited for complex or large user cases. 
Hence, further research is needed to \textbf{identify approaches that facilitate the writing of mappings, reducing the need to learn a language or syntax, while remaining scalable for large use cases}, in order to facilitate their adoption. 
In addition, since each language is processed by different engines with different capabilities, it is not unusual to need to learn more than one single language, when use cases' needs change and cannot be addressed by one engine~\parencite{corcho2020towards}. 
Hence, \textbf{it is required to improve the interoperability between existing mapping languages} and bridge the gap between user-friendly mapping creation and existing KG construction implementations. 


\ana{orientar a: si se ha estudiado como mejoran el proceso desde este ey este sentido, pero no si intervienen en la evolución}

The technologies revolving mappings are a suitable for composing modular, automated approaches to be integrated into larger pipelines that manage knowledge graphs~\parencite{simsek2021knowledge,cimmino2022helio,grassi2023composable}. 
However, these declarative approaches can play an important role in other different tasks of the life cycle apart from the construction. 
For instance, they enable metadata annotation about the data sources, which can help enhance the process in terms of transparency, completeness and performance~\parencite{chaves2021morph-csv,vidal2023knowledge}; as well as 
data processing and cleaning, thanks to the inclusion of data transformation functions in the language~\parencite{debruyne2016r2rmlf,junior2016funul,jozashoori2020funmap,DeMeester2017fno_dbpedia}.
However, \textbf{ there is a lack of research to assess how these technologies can be involved not only the construction of knowledge graphs, but also the beneficial role they can play in their evolution}.

\ana{terminar con un párrafo (como patri): In general, this thesis aims to ... Eso o que sea el que inicie la siguiente sección.}

%\textcolor{red}{However, these technologies are relatively recent, and as mentioned before, struggle with a wider adoption. Still, the last few years have witnessed an increased effort for being refined and optimized~\parencite{calvanese2017ontop,chaves2019parameters,arenas2022morphkgc,iglesias2023scaling}, to reduce the learning curve and to prove their benefits with respect to ad-hoc approaches~\parencite{iglesias2019bio2rdf}. }


\section{Main contributions}

The contributions of this thesis are listed below, organized according to the main thesis objectives:

\begin{enumerate}
    \item The first objective consists of gathering, understanding and implementing the current needs for knowledge graph construction from heterogeneous data sources with mapping languages. 

    \begin{itemize}
        \item \textbf{A comparison framework with a fine-grained analysis of the characteristics of current mapping languages.} We design a framework with the features that a mapping language may provide, from data source description to triple transformation and additional rules that apply to triple creation, and check whether current mapping languages provide such features in the language itself or supported by a compliant engine. 

        \item \textbf{A set of requirements for knowledge graph construction}, extracted from the comparison framework and the needs of the community of practitioners. These requirements are implemented in a formal language, an ontology. 

        \item \textbf{The implementation of these requirements in the RML mapping language.} The requirements gathered in previous contributions are implemented in RML, a widely adopted language, focusing on the update that involves enabling the construction of RDF-star graphs.
    \end{itemize}

    \item The second objective consists of improving the user experience for knowledge engineers and domain experts to write mapping rules.

    \begin{itemize}
        \item \textbf{A spreadsheet-based approach to write mapping rules along with Mapeathor, the compliant tool to generate mapping files in diverse languages.} This approach enables users to write mapping rules in well-known spreadsheet editors (e.g. MS Excel, Google Spreadsheets) without the need to learn a language's constructs or syntax, to be automatically translated into a correctly formatted mapping file. 

        \item \textbf{The update of the user-friendly serialization YARRRML for RML}, incorporating the latest features of the language, along with the compliant tool Yatter to translate them into readable [R2]RML mapping files. 
    \end{itemize}

    \item The third objective consists of assessing the role of mapping-compliant technologies to support the evolution of knowledge graphs.

    \begin{itemize}
        \item \textbf{An evaluation of mapping-compliant systems with well-known triplestores performing re-construction of knowledge graphs} with different metadata representation models. We test the performance of each approach and the features needed for re-constructing each representation to evaluate which approach is more suitable in each case. 
    \end{itemize}
\end{enumerate}


\section{Thesis structure}

The remainder of this thesis is structured as follows:

\begin{itemize}
    \item \cref{chapter:sota} describes the main concepts used in this thesis, and reviews the state of the art of the topics of interest related to identify the current limitations. It first introduces the basic notions about knowledge representations on the web, to then proceed to describe the current landscape of mapping languages and methods to write them in a user-friendly manner; and then it presents the related works assessing the role of these declarative mappings in the different phases that take place within the knowledge graph life cycle. The chapter concludes identifying the gaps in the state of the art, leading to the contribution of the thesis.
    
    \item \cref{chapter:objectives} presents the main objectives and contributions of this thesis, along with the assumptions, hypotheses and restrictions of the work. 
    
    \item In \cref{chapter:mappings} we define a comparison framework to analyze the features of current mapping languages. This framework helps extract a set of requirements for knowledge graph construction, that are gathered and represented in an ontology. Additionally, we present a particular case of the update of the RML mapping language to address some of the presented requirements, specifically the RML-star module to construct RDF-star graphs.
    
    \item In \cref{chapter:creation} we present two approaches to facilitate the creation of mapping documents for users. The first relies on spreadsheets to write mapping rules, and is with real users with different backgrounds to test its usability. The second supposes an update on the YARRRML user-friendly syntax with recent modifications over RML. Both approaches are presented with compliant tools to create mapping documents in target languages. 
    
    \item In \cref{chapter:evolution} we assess the role that mapping-compliant technologies can play in knowledge graph evolution. We conduct an empirical evaluation performing schema changes with KG construction engines that use mappings, and SPARQL \texttt{CONSTRUCT} queries, and evaluate which approach is more suitable in different situations. 
    
    \item Finally, \cref{chapter:conclusions} draws the main conclusions of the presented work, and outlines the future research directions. 
    
    %\item \ana{Appendix? cuando estén más seguros}
\end{itemize}


\section{Derived outcomes}

This section lists the publications derived from the work in this thesis.


\begin{itemize}
    \item \textbf{Journal Publications}
    \begin{itemize}
        \item \textbf{Iglesias-Molina, A.}, Cimmino, A., Ruckhaus, E., Chaves-Fraga, D., García-Castro, R., Corcho, O. (2024) An Ontological Approach for Representing Declarative Mapping Languages. \textit{Semantic Web}, \textit{5} (1), 191–221. 
    
        \item Arenas-Guerrero, J., \textbf{Iglesias-Molina, A.}, Chaves-Fraga, D., Garijo, D., Corcho, O. and Dimou, A. (2023) Declarative generation of RDF-star graphs from heterogeneous data. \textit{Semantic Web}, in press.
    \end{itemize}
\end{itemize}




\begin{itemize}
    \item \textbf{Conference Publications}
    \begin{itemize}
        \item \textbf{Iglesias-Molina A.}, Toledo J., Corcho O. and Chaves-Fraga D. (2023) Re-Construction Impact on Metadata Representation Models. In \textit{Proceedings of The Twelfth International Conference on Knowledge Capture (K-CAP23)}, December 5 - 7, Pensacola.
    
        \item \textbf{Iglesias-Molina, A.}, Van Assche, D., Arenas-Guerrero, J., De Meester, B., Debruyne, C., Jozashoori, S., Maria, P., Michel, F., Chaves-Fraga, D. and Dimou, A. (2023) The RML Ontology: A Community-Driven Modular Redesign After a Decade of Experience in Mapping Heterogeneous Data to RDF. In \textit{Proceedings of the 22nd International Semantic Web Conference (ISWC2023)}, November 6--10, Athens.
    
        \item \textbf{Iglesias-Molina, A.}, Ahrabian, K., Ilievski, F., Pujara, J. and Corcho, O. (2023) Comparison of Knowledge Graph Representations for Consumer Scenarios. In \textit{Proceedings of the 22nd International Semantic Web Conference (ISWC2023)}, November 6--10, Athens.
    
        \item \textbf{Iglesias-Molina, A.}, Chaves-Fraga, D., Dasoulas, I. and Dimou, A. (2023) Human-Friendly RDF Graph Construction: Which One Do You Chose?. In \textit{Proceedings of the 23rd International Conference on Web Engineering 2023 (ICWE2023)}, June 6--9, Alicante.
    \end{itemize}
\end{itemize}

\begin{itemize}
    \item \textbf{Workshop Publications}
    \begin{itemize}
        \item \textbf{Iglesias-Molina, A.}, Cimmino, A., Corcho, O. (2022) Devising Mapping Interoperability with Mapping Translation. In \textit{Proceedings of the Third International Workshop on Knowledge Graph Construction, co-located with the 19th Extended Semantic Web Conference}. May 29 -- June 2, Hersonissos. 
    
        %\item Arenas-Guerrero J., Scrocca M., \textbf{Iglesias-Molina A.}, Toledo J., Pozo-Gilo L., Dona D., Corcho O. and Chaves-Fraga D. (2021) Knowledge Graph Construction with R2RML and RML: An ETL System-based Overview. In \textit{Proceedings of the Second International Workshop on Knowledge Graph Construction, co-located with the 18th Extended Semantic Web Conference}. June 6--10, online.
        
        \item \textbf{Iglesias-Molina, A.}, Chaves-Fraga, D., Priyatna, F. and Corcho, O. (2019) Towards the definition of a language-independent mapping template for knowledge graph creation. In \textit{Proceedings of the Third International Workshop on Capturing Scientific Knowledge co-located with the Eleventh International Conference on Knowledge Capture}. November 19--21, Marina del Rey.
    \end{itemize}
\end{itemize}

\begin{itemize}
    \item \textbf{Posters and demos}
    \begin{itemize}
        \item \textbf{Iglesias-Molina, A.} and Garijo D. (2023) Towards Assessing FAIR Research Software Best Practices in an Organization Using RDF-star. In \textit{Proceedings of the Semantics 2023 Posters and Demos Track}, September 19--22, Leipzig.
        
        \item Delva, T., Arenas-Guerrero, J., \textbf{Iglesias-Molina, A.}, Corcho, O., Chaves-Fraga, D., and Dimou, A. (2021) RML-star: A declarative mapping language for RDF-star generation. In \textit{Proceedings of the ISWC 2021 Posters, Demos and Industry Tracks}, October 24--28, online.
    
        \item \textbf{Iglesias-Molina, A.}, Pozo-Gilo, L., Dona, D., Ruckhaus, E., Chaves-Fraga, D. and Corcho, O. (2020) Mapeathor: Simplifying the Specification of Declarative Rules for Knowledge Graph Construction. In \textit{Proceedings of the ISWC 2020 Demos and Industry Tracks}, November 2--6, online.
    \end{itemize}
\end{itemize}


\section{Research stay}
\begin{itemize}
    \item 06/07/2022 -- 06/10/2022. Research stay at the \textbf{Information Sciences Institute of the University of Southern California}, supervised by Prof. Dr. Filip Ilievski. During this stay, we analyzed the differential impact of diverse knowledge graph representation over different consumption scenarios: knowledge exploration performed by users, systematic query performance and graph completion tasks. This stay was funded by a scholarship from Programa Propio I+D+i of UPM, oriented to research personnel in predoctoral formation for doing an international research stay equal to or superior to three months.
\end{itemize}

\section{Workshops and Tutorials}
\begin{itemize}
    \item Organization of the \textbf{Fourth and Fifth Editions of the Knowledge Graph Construction Workshop (KGCW)}\footnote{\url{https://w3id.org/kg-construct/workshop/2023}}\textsuperscript{,}\footnote{\url{https://w3id.org/kg-construct/workshop/2024}}, co-located in the Extended Semantic Web Conference (ESWC 2023 and 2024), celebrated both times in Hersonissos, Greece, organized together with Anastasia Dimou (KU Leuven), David Chaves-Fraga (Universidade de Santiago de Compostela), Umutcan Serles (STI Innsbruck) and Dylan Van Assche (Universiteit Gent). This workshop gathers in every edition the community of KG Construction to present relevant research, breakthroughs and resources in the field, complementing the activities of the Knowledge Graph Construction W3C Community Group. 

    \item Organization of the \textbf{Tutorial on Knowledge Graph Construction using Declarative Mapping Rules}\footnote{\url{https://oeg-dataintegration.github.io/kgc-tutorial-2020/}}, co-located in the 19th International Semantic Web Conference (ISWC2020), celebrated on November 2-6 2020 online, organized together with Oscar Corcho, David Chaves-Fraga and Andrea Cimmino (UPM); 
    and the \textbf{Tutorial on Declarative Construction and Validation of Knowledge Graphs}\footnote{\url{https://w3id.org/kg-construct/tutorials/kcap2023}}, co-located in the 12th International Conference on Knowledge Capture (K-CAP2023), celebrated on December 5-7 2023 in Pensacola, USA, organized together with Xuemin Duan (KU Leuven). 
    Presenter in the \textbf{Knowledge Graph Construction Tutorial}\footnote{\url{https://w3id.org/kg-construct/costdkg-eswc-tutorial}}, co-located in the 19th Extended Semantic Web Conference (ESWC2022), celebrated on May 29 – June 2 2022.
    All tutorials focused on explaining in detail, from a practical perspective, the process of constructing knowledge graphs, from writing mappings to their execution, publication and validation with suitable tools. 
\end{itemize}

\section{Projects}
During the development of this thesis, the author participated in the following research and innovation projects:

\begin{itemize}
    \item \textit{SOLARCHEM 5.0: Towards Digital Transition in Solar Chemistry (SolarChem 5.0): AI-assisted robotized platform for the development of efficient photoelectrodes}, reference TED2021-130173B-C41, within the State Program to Promote Scientific-Technical Research and its Transfer (Strategic Projects Oriented to Ecological Transition and Digital Transition).

    \item \textit{DRUGS4COVID++: Servicios de Inteligencia Artificial para la creación de un grafo de conocimientos sobre fármacos usados en el control clínico de la enfermedad, a partir de la explotación de grandes corpus de documentación científica sobre SARS-COV-2 y COV}, funded by BBVA grants for Teams of Scientific Research on SARS-CoV-2 and COVID-19.

    \item \textit{Ontology Extension about Insurances and Construction of Knowledge Graphs}, funded by REALE. 
    
    \item \textit{Governance Model, Best Practices and Standards for BASF Ontologies}, funded by BASF. 
\end{itemize}

\chapter{State of the Art \textcolor{red}{-- By 2/2}}
\label{chapter:sota}

In this chapter, we discuss the state of the art in declarative construction of knowledge graphs. We first present \textcolor{red}{the representation of knowledge on the web} (\cref{sec:chp2_semweb}). Then we focus on describing current techniques to construct knowledge graphs, focusing on the declarative approaches (\cref{sec:chp2_declarative_kgc}), and the user-friendly approaches developed to facilitate this task (\cref{sec:chp2_easy_kgc}). We also describe the current proposals and methodologies for supporting the knowledge graph life cycle \ana{queda algo suelto, encajar algo de construcción y porqué tiene esto relevancia con el resto} (\cref{sec:chp2_kg_lifecycle}).



\section{Knowledge Representation in the Semantic Web \textcolor{red}{-- By 19/1}}
\label{sec:chp2_semweb}

\textit{This section describes some background knowledge about concepts that play a relevant role in knowledge representation on the web. First we look into RDF, used for representing knowledge graphs on the web. Then we go over different approaches to reify knowledge in RDF, useful for representing additional information beyond triple paradigm. ??}

\subsection{Resource Description Framework (RDF)}

The Resource Description Framework (RDF)~\parencite{rdf} is a standard model for data interchange on the web. Its basic unit are triples: two concepts linked by a relationship. These triples are represented in the form of $<$ \textit{s,p,o} $>$, where \textit{s} is the subject, \textit{p} is the predicate and \textit{o} the object. This model relies on the use on the linking structure of the Web, using IRIs to identify uniquely every resource and relationship. 

\cref{lst:chp2_rdf-example} presents an example of an RDF graph composed of a set of triples about pole vault records, which is also visually depicted in \cref{fig:chp2_rdf-example}. In total there are four triples. They all share the same subject, which is a resource uniquely identified by the IRI $<$\texttt{http://example.com/athlete/1}$>$. Likewise, all predicates in the triples are also defined by an IRI, e.g. $<$\texttt{http://example.com/ns\#name}$>$. The first triple (Line 5) defines that the subject is an instance of the class \texttt{ex:Athlete} with the predicate \texttt{rdf:type}. The rest of the triples define attributes to this instance with name (\texttt{ex:name}) and position in the ranking (\texttt{ex:rank}). The objects of both triples are literals, that may be strings ("Yelena Isinbayeva", Line 6) or typed literals ("1"\scalebox{.8}{\textsuperscript{$\wedge\wedge$}}\texttt{xsd:integer}, Line 7). Type literals refer to data values attached with a tag that represents their data type.

\begin{minipage}{\textwidth}
\begin{captionedlisting}{lst:chp2_rdf-example}{Example of RDF graph.}
\centering
{\begin{lstlisting}[language=r2rml]
@prefix rdf: <http://www.w3.org/1999/02/22-rdf-syntax-ns#>.
@prefix xsd: <http://www.w3.org/2001/XMLSchema#>.
@prefix ex: <http://example.com/ns#>.

<http://example.com/athlete/1> rdf:type ex:Athlete .
<http://example.com/athlete/1> ex:name "Yelena Isinbayeva" .
<http://example.com/athlete/1> ex:rank "1"^^xsd:integer .
\end{lstlisting}}
\end{captionedlisting}
\end{minipage}

There are different syntaxes to serialize RDF graphs. RDF/XML\footnote{\url{https://www.w3.org/TR/rdf-syntax-grammar}} was the first to be proposed, and relies on XML. Notation3 (N3)\footnote{\url{https://www.w3.org/TeamSubmission/n3/}} was developed as a human readable syntax, but its use is not common. Instead, the N-Triples\footnote{\url{https://www.w3.org/TR/n-triples}} and Turtle\footnote{\url{https://www.w3.org/TR/turtle}} serializations, subsets of N3, are more widely used. JSON-LD\footnote{\url{https://www.w3.org/TR/json-ld11}} was developed for programming environments, as it comprised in JSON documents. Lastly, RDFa\footnote{\url{https://www.w3.org/TR/rdfa-primer}} extends HTML to markup structured content in webpages, with the objective of improving the results retrieved by search engines. 



\begin{figure*}[t]
\centering
\includegraphics[width=0.6\linewidth]{figures/chp2_rdf-example.pdf}
\caption[RDF graph example]{Visual representation of the RDF graph shown in \cref{lst:chp2_rdf-example}.}
\label{fig:chp2_rdf-example}
\end{figure*}

\subsection{Statements about statements in RDF}

%\ana{basic intro about what is this for and mot example about info that cannot be plainly represented in RDF. Then present all approaches used along the document: std reification, n-ary relationships, singleton properties, rdf-star, named graphs, i'd leave wikidata out of this}

Plain triples are not always able to represent the complexity of certain knowledge. This is the case of statement annotation, when it is required to add information to a triple that does not correspond to a single resource, but the whole. This situation has triggered the development different approaches to achieve triple annotation (or reification). \ana{figure } illustrates instances of these models for the main statement \textit{Yelena Isinbayeva obtained the first position in the rank}, annotated with the additional statement \textit{in the season of 2009}. 

\begin{figure*}[t]
\centering
\includegraphics[width=\linewidth]{figures/chp2_reifications.pdf}
\caption[Approaches for statement reification in RDF]{Approaches for statement reification in RDF: (a) Standard Reification, (b) N-Ary Relationships, (c) Singleton Properties, (d) RDF-star and (e) Named Graphs.}
\label{fig:chp2_reification}
\end{figure*}

\noindent\textbf{Standard Reification}~\cite{lassila1999rdf} explicitly declares a resource to denote an \texttt{rdf:Statement}.
This statement has \texttt{rdf:subject}, \texttt{rdf:predicate}, and \texttt{rdf:object} attached to it and can be further annotated with additional statements. The resource is typically a blank node, but an IRI can be used. \textcolor{red}{In \cref{fig:chp2_reification}a, the resource \texttt{:entity-stm/407} is an \texttt{rdf:Statement} with four associated triples, where the objects of the triples are the actual values of the triple (i.e. \texttt{:entity/407} for the subject, \texttt{:semanticType} for the predicate, and \texttt{"orga"} for the object). The property \texttt{:score} is used with its own value as object.}


\noindent\textbf{N-Ary Relationships}~\cite{naryw3c2006} converts a relationship into an instance that describes the relation, which can have attached both the main object and additional statements.
This representation is widely used in ontology engineering as an ontology design pattern~\cite{gangemi2013multi}.\textcolor{red}{ In \cref{fig:chp2_reification}b, the entity \texttt{:entity/407} points to an intermediate node (\texttt{:entity-semtype/ 407}) which holds the triples for both the assignment of the semantic type and the score.}

\noindent\textbf{Singleton Properties}~\cite{nguyen2014don}

\noindent\textbf{RDF-star}~\cite{hartig2017foundations,hartig2023rdf12} extends RDF to introduce a new syntax for compact triple reification. 
It introduces the notion of triple recursiveness with \texttt{Quoted Triples}, which can be used as subjects and/or objects of other triples. This is the only approach that extends the standard RDF features. 
%, having a potential impact on the development of supporting tools, triplestores, etc. 
This representation is currently being incorporated into the RDF 1.2 specification~\cite{hartig2023rdf12}, which is currently being developed under the RDF-star W3C Working Group\footnote{\url{https://www.w3.org/groups/wg/rdf-star/}}.
\textcolor{red}{In Figure \ref{fig:chp2_reification}d we observe the example as an RDF-star graph, and it is represented in RDF as \texttt{{<<:entity/407 :semanticType "orga">> :score 0.8}}.}


\noindent\textbf{Named Graphs}~\parencite{cyganiak2014rdf11} are a SPARQL 1.1 feature that allows the assignment of an IRI to one or several triples as a graph identifier. Hence, graph IRIs allow the unique identification of triples. These IRIs can be used as subjects to add additional statements. \textcolor{red}{In \cref{fig:chp2_reification}e, the triple indicating the semantic type of an entity is assigned the named graph \texttt{:entity-graph/407}. The graph IRI is subsequently used as subject in a triple that annotates the confidence score of the information within the graph. }

\section{Declarative Knowledge Graph Construction \textcolor{red}{By 12/1}}
\label{sec:chp2_declarative_kgc}

Knowledge graphs can be constructed in diverse manners. One way comprises collecting knowledge from contributions form the community, such as Wikidata. Other way involves transforming data from heterogeneous formats and sources, unstructured or (semi-)structured, into RDF. This section focuses on providing an overview of the latter, to declaratively construct knowledge graphs from heterogeneous data sources. \ana{decir que pasa en cada subsección?}

%\ana{overview de distintas formas y métodos en general de construir KGs. Debería ser algo larga, ampliando lo que hay en la intro. Luego terminar con enfocarse en los approaches declarativos, que se extienden en la siguiente subsección}

%\ana{RAW} In this section, the current scene of mapping languages is described first, regardless of the approach they follow, i.e., RDF materialization or virtualization. Then, previous works comparing mapping languages are surveyed. 



\subsection{Declarative Mapping Rules}

Constructing RDF knowledge graphs from heterogeneous data sources involve a schema transformation of the data to the desired graph structure. This transformation may be done in several different ways, usually involving mapping languages that allow expressing the transformation rules to create the target graphs. Hence, these mappings hold declaratively the relationships between the source and target data schemas. This comprises an agnostic approach that can (and has been) applied to multiple different use cases. \ana{refs}

The usual workflow in which declarative approaches are involved is depicted in \ana{figure}. They enable both materialization and virtualization of knowledge graphs. In materialization scenarios, data is transformed into the target graph, usually following the schema provided by an ontology. In virtualization scenarios, data is not transformed; instead, the original data source is accessed also following the schema of the target (virtual) graph. This process requires translating the original query into the language of the original data source. \ana{mention OBDA?}

Following, we present an overview of existing mapping languages, listed in \cref{tab:chp2_languages_summary}. We classify these languages in three categories, based on the schema they are based on or extend: (i)~RDF-based, (ii)~SPARQL-based, and (iii)~based on alternative schemas. An overview of the evolution, extensions and influences of these languages is depicted in \cref{fig:chp2_mapping_languages}.



\begin{figure*}[h]
\centering
\includegraphics[width=1\linewidth]{figures/chp2_mapping_languages}
\caption[Existing mapping languages and the relationships among them]{Existing mapping languages and the relationships among them.}
\label{fig:chp2_mapping_languages}
\end{figure*}

\begin{table}[t]
\caption[Mapping languages overview]{Analyzed mapping languages and their corresponding references. \ana{pensar en eliminar helio, la versión nueva igual no entra dentro de esto y puede que cuadre quitar SMS2 y ponerlo luego con el análisis de las sintaxis}}
\label{tab:chp2_languages_summary}
\begin{tabular}{c|c|c}
\hline
%\rowcolor[HTML]{EFEFEF} 
Classification                 & Language        & Reference(s) \\ \hline
\multirow{11}{*}{RDF-based}   & D2RQ            & \parencite{bizer2004d2rq,d2rq}\\ \cline{2-3} 
                              & R$_2$O          & \parencite{barrasa2004r2o}\\ \cline{2-3} 
                              & R2RML           & \parencite{das2012r2rml}\\ \cline{2-3} 
                              & xR2RML          & \parencite{michel2015xr2rml,xr2rml}\\ \cline{2-3} 
                              & RML             & \parencite{Dimou2014rml,rml}\\ \cline{2-3} 
                              & KR2RML          & \parencite{slepicka2015kr2rml}\\ \cline{2-3} 
                              & FunUL           & \parencite{junior2016funul}\\ \cline{2-3} 
                              & R2RML-f         & \parencite{debruyne2016r2rmlf}\\ \cline{2-3} 
                              & D2RML           & \parencite{chortaras2018d2rml}\\ \cline{2-3} 
                              & R2RML for collections & \parencite{debruyne2017R2RML-collections}\\ \cline{2-3}   
                              & XLWrap          & \parencite{langegger2009xlwrap,xlwrap}\\ \cline{2-3} 
                              & CSVW            & \parencite{Tennison2015csvw}\\ \hline
\multirow{4}{*}{SPARQL-based} & SPARQL-Generate &     
                              \parencite{Lefrancois2017sparqlgenerate,sparqlgenerate}\\ \cline{2-3} 
                              & XSPARQL         & \parencite{Bischof2012xsparql,xsparql}\\ \cline{2-3} 
                              & TARQL           & \parencite{tarql}\\ \cline{2-3}
                              & Facade-X        & \parencite{asprino2023sparql-anything,sparqlanything}\\ \cline{2-3}
                              & Sansa            & \parencite{stadler2023spark}\\ \hline
\multirow{3}{*}{Others}       & Helio mappings  & \parencite{cimmino2022helio}\\ \cline{2-3} 
                              & D-REPR          & \parencite{Vu2019d-repr}\\ \cline{2-3} 
                              & ShExML          & \parencite{Garcia-Gonzalez2020shexml,shexml}\\  \hline
\end{tabular}
\end{table}




\subsubsection{RDF-based mapping languages.} 

This group of languages are specified as ontologies or vocabularies able to describe the transformation rules of heterogeneous data into RDF. They are written in RDF documents, usually using the Turtle syntax~\parencite{turtle}. 



\noindent\textbf{D2RQ}~\parencite{bizer2004d2rq}

\noindent\textbf{XLWrap}~\parencite{langegger2009xlwrap}

\noindent\textbf{R2RML}~\parencite{das2012r2rml} 

\noindent\textbf{RML}~\parencite{Dimou2014rml} + extensions (RML-star, RML fields, RML target, RML+FnO)

\noindent\textbf{KR2RML}~\parencite{slepicka2015kr2rml}

\noindent\textbf{xR2RML}~\parencite{michel2015xr2rml}

\noindent\textbf{FunUL}~\parencite{junior2016funul}

\noindent\textbf{R2RML-F}~\parencite{debruyne2016r2rmlf}

\noindent\textbf{R2RML for collections and containers}~\parencite{debruyne2017R2RML-collections}

\noindent\textbf{D2RML}~\parencite{chortaras2018d2rml}

\noindent\textbf{CSVW}~\parencite{Tennison2015csvw}

\textit{The most well-known language in this category is R2RML~\parencite{das2012r2rml}, which allows mapping of data stored in relational databases to RDF. This language is heavily influenced by previous languages (R$_2$O~\parencite{barrasa2004r2o} and D2RQ~\parencite{bizer2004d2rq}). Some serializations (e.g. SML~\parencite{Stadler2015sml}, OBDA mappings from Ontop~\parencite{rodriguez2015efficient}) and several extensions of R2RML were developed in the following years after its release: R2RML-f~\parencite{debruyne2016r2rmlf} extends R2RML to include functions to be applied over the data; RML~\parencite{Dimou2014rml} and its user-friendly compact syntax YARRRML~\parencite{Heyvaert2018yarrrml} provide the possibility of covering additional data formats (CSV, XML and JSON); this language also considers the use of functions for data transformation (e.g. lowercase, replace, trim) by using the Function Ontology (FnO)\footnote{\url{https://fno.io/rml/}}~\parencite{DeMeester2017fno_dbpedia}; FunUL~\parencite{junior2016funul} proposes an extension to also incorporate functions, but focusing on the CSV format; KR2RML~\parencite{slepicka2015kr2rml} is also an extension for CSV, XML and JSON, with the addition of representing all sources with the Nested Relational Model as an intermediate model and the possibility of cleaning data with Python functions; xR2RML~\parencite{michel2015xr2rml} extends R2RML and RML to include NoSQL databases and incorporates more features to handle tree-like data; D2RML~\parencite{chortaras2018d2rml}, also based on R2RML and RML, is able to transform data from XML, JSON, CSVs and REST/SPARQL endpoints, and enables functions and conditions to create triples. }

\textit{In this category, we can also find more languages not related to R2RML. XLWrap~\parencite{langegger2009xlwrap} is focused on transforming spreadsheets into different formats. CSVW~\parencite{Tennison2015csvw} enables tabular data annotation on the Web with metadata, but also supports the generation of RDF. Finally, WoT Mappings~\parencite{cimmino2020ewot} are oriented to be used in the context of the Web of Things.}




\subsubsection{SPARQL-based mapping languages.} 

This group is integrated by languages that leverage the SPARQL query language, usually by extending its features to describe non-RDF data sources~\parencite{harris2013sparql}. 


\noindent\textbf{XSPARQL}~\parencite{Bischof2012xsparql}

\noindent\textbf{SPARQL-Generate}~\parencite{Lefrancois2017sparqlgenerate}

\noindent\textbf{TARQL}~\parencite{tarql}

\noindent\textbf{SPARQL-Anything}~\parencite{asprino2023sparql-anything}

\noindent\textbf{Sansa/whatever}~\parencite{stadler2023spark}




\textit{XSPARQL~\parencite{Bischof2012xsparql} merges SPARQL and XQuery to transform XML into RDF. TARQL~\parencite{tarql} uses the SPARQL syntax to generate RDF from CSV files. SPARQL-Generate~\parencite{Lefrancois2017sparqlgenerate} is capable of generating RDF and document streams from a wide variety of data formats and access protocols. Most recently, Facade-X has been developed, not as a new language, but as a "\textit{facade} to wrap the original resource and to make it queryable as if it was RDF"~\parencite{asprino2023sparql-anything}. It does not extend the SPARQL language, instead it overrides the SERVICE operator. Lastly, authors would like to highlight a loosely SPARQL-based language, Stardog Mapping Syntax 2 (SMS2)~\parencite{sms2}, which represents virtual Stardog graphs and is able to support sources such as JSON, CSV, RDB, MongoDB and Elasticsearch.}




\subsubsection{Based on other schemas.} 

Lastly, the languages of this group rely on schemas different from RDF or SPARQL, equally able and expressive enough to enable users to write transformation rules. 


\noindent\textbf{R$_2$O}~\parencite{barrasa2004r2o} XML-based~\footnote{https://pdfs.semanticscholar.org/4c47/0826aafc07fc6d37ca7e2474c1d3b290ade1.pdf}

\noindent\textbf{D-REPR}~\parencite{Vu2019d-repr}

\noindent\textbf{ShExML}~\parencite{Garcia-Gonzalez2020shexml,garcia2021shexml-challenges}

\noindent\textbf{Helio??}~\parencite{cimmino2022helio}

\textit{ShExML~\parencite{Garcia-Gonzalez2020shexml,garcia2021shexml-challenges} uses Shape Expressions (ShEx)~\parencite{prud2014shex} to map data sources in RDBs, CSV, JSON, XML and RDF using SPARQL queries. The Helio mapping language~\parencite{cimmino2022helio} is based on JSON and provides the capability of using functions for data transformation and data linking~\parencite{cimmino2018hybrid}. D-REPR~\parencite{Vu2019d-repr} focuses on describing heterogeneous data with JSONPath and allows the use of data transformation functions. XRM (Expressive RDF Mapper)~\parencite{xrm} is a commercial language that provides a unique user-friendly syntax to create mappings in R2RML, CSVW and RML.}










\subsection{Mapping Languages Comparison}
As the number of mapping languages increased and their adoption grew wider, comparisons between these languages inevitably occurred. This is the case of, for instance, SPARQL-Generate~\parencite{Lefrancois2017sparqlgenerate}, which is compared to RML in terms of query/mapping complexity; and ShExML~\parencite{Garcia-Gonzalez2020shexml}, which is compared to SPARQL-Generate and YARRRML from a usability perspective.

Some studies dig deeper, providing qualitative complex comparison frameworks. Hert et al.~\parencite{hert2011comparison} provide a comparison framework for mapping languages focused on transforming relational databases to RDF. The framework is composed of 15 features, and the languages are evaluated based on the presence or absence of these features.% (Logical table to class, M:N relationships, project attributes, select conditions, user-defined instance URIs, literal to URIs, vocabulary reuse, transformation functions, datatypes, named graphs, blank nodes, integrity constraints, static metadata, one table to \textit{n} classes, and write support). 
The results lead authors to divide the mappings into four categories (direct mapping, read-only general-purpose mapping, read-write general-purpose mapping, and special-purpose mapping), and ponder on the heavy reliance of most languages on SQL to implement the mapping, and the usefulness of read-write mappings (i.e., mappings able to write data in the database). De Meester et al.~\parencite{DeMeester2019comparison} show an initial analysis of 5 similar languages (RML+FnO, xR2RML, FunUL, SPARQL-Generate, YARRRML) discussing their characteristics, according to three categories: non-functional, functional and data source support. The study concludes by remarking on the need to build a more complete and precise comparative framework and asking for a more active participation from the community to build it. To the best of our knowledge, there is no comprehensive work in the literature comparing all existing languages. \ana{ojo con esto que con el survey de ghent ya no es del todo cierto :(} 


\section{User-Friendly Knowledge Graph Construction Approaches}
\label{sec:chp2_easy_kgc}
 
This section presents the different approaches developed for easing the writing of mapping rules for practitioners. We divide these approaches into two categories, (i) visual editors (\cref{sec:chp2_visual-editors}) that rely in an interactive application to graphically depict and edit mappings, and (ii) serializations (\cref{sec:chp2_serializations}) that rely on a simplified syntax for writing the mapping rules. \cref{tab:chp2_easy-mappings} lists the approaches described in the remaining of the section.



\subsection{Visual Editors}
\label{sec:chp2_visual-editors}

Visual editors were first developed for enhancing and easing the mapping writing process. We can subdivide the proposals released over the years based on how the mapping is graphically represented in the tool: either with a tree or graph layout, or using building blocks (\cref{fig:chp2_visual-editors}).


The tree layout (\cref{fig:chp2_visual-editors}a) succeeded in the first approaches developed. 
The first one was developed for the R$_2$O mapping language in \textbf{ODEMapster}~\parencite{barrasa2006odemapster}, providing a graphical interface to visualize and edit the mappings by linking the database elements with the ontology resources. 
After the release of R2RML, \textbf{Karma}~\parencite{gupta2012karma} and \textbf{RBA} (R2RML by assertion)~\parencite{neto2013rba} were developed for this language. Both work similarly to ODEMapster, and Karma additionally provides automatic mapping suggestions. 

\begin{table}[t]
\caption[Approaches for easy mapping creation]{Approaches for facilitating the mapping creation process for users. The proposals are divided into (i) visual editors and (ii) text-based serializations. For each proposal, the reference, compliant mapping language and subtype is provided.}
\label{tab:chp2_easy-mappings}
\resizebox{\columnwidth}{!}{
\centering
\begin{tabular}{cccc}
%\rowcolor[HTML]{EFEFEF} 
\textbf{Classification} & \textbf{Approach} & \textbf{Mapping Language} & \textbf{Type} \\ \midrule
\multirow{15}{*}{\textbf{Visual Editor}} & ODEMapster~\parencite{barrasa2006odemapster} & R2O & Tree \\
 & Karma~\parencite{gupta2012karma} & R2RML & Tree \\
 & RBA~\parencite{neto2013rba} & R2RML & Tree \\
 & \cite{sengupta2013editing} & R2RML & Building blocks \\
 & \cite{lembo2014visualization} & R2RML & Graph \\
 & RMLEditor~\parencite{heyvaert2016rmleditor} & R2RML/RML & Graph \\
 & SQuaRE~\parencite{blinkiewicz2016square} & R2RML & Graph \\
 & OntopPro~\parencite{calvanese2017ontop} & Proprietary & Building blocks \\
 & Juma~\parencite{junior2017juma} & R2RML & Building blocks \\
 & RMLx~\parencite{aryan2017rmlx} & RML & Building blocks \\
 & Map-On~\parencite{sicilia2017map} & R2RML & Graph \\
 & gra.fo\tablefootnote{\label{foot:gra.fo}\url{https://gra.fo/}} & R2RML & Graph \\
 & Stardog designer\tablefootnote{\label{foot:stardog-designer}\url{https://www.stardog.com/designer/}} & SMS2 & Graph \\
 & Ontopic Studio\tablefootnote{\label{foot:ontopic-studio}\url{https://ontopic.ai/en/ontopic-studio/}} & R2RML/Proprietary & Building blocks \\
 & Eccenca Corporate Memory\tablefootnote{\label{foot:eccenca}\url{https://documentation.eccenca.com/latest/build/lift-data-from-tabular-data-such-as-csv-xslx-or-database-tables}} & Proprietary & Building blocks \\ \midrule
\multirow{5}{*}{\textbf{Serialization}} & SML~\parencite{Stadler2015sml} & R2RML & SPARQL-based \\
 & Ontop proprietary~\parencite{calvanese2017ontop} & R2RML & TTL- and SQL-based \\
 & YARRRML~\parencite{Heyvaert2018yarrrml} & RML & YAML-based \\
 & SMS2\tablefootnote{\label{foot:sms2}\url{https://docs.stardog.com/archive/7.5.0/virtual-graphs/mapping-data-sources.html\#sms2-stardog-mapping-syntax-2}} & R2RML & SPARQL-based \\
 & XRM\tablefootnote{\label{foot:xrm}\url{https://zazuko.com/products/expressive-rdf-mapper/}} & R2RML/RML/CSVW & Proprietary syntax \\ \bottomrule
\end{tabular}
}
\end{table}


\begin{figure*}[t]
\centering
\includegraphics[width=0.9\linewidth]{figures/chp2_visual-editors.pdf}
\caption[Graphical representations approaches in visual mapping editors.]{Graphical representations approaches in visual mapping editors: (a) tree layout, (b) graph layout, and (c) building blocks.}
\label{fig:chp2_visual-editors}
\end{figure*}


The graph display (\cref{fig:chp2_visual-editors}b) was later adopted by multiple editors, such as in \cite{lembo2014visualization}, \textbf{SQuaRE}~\parencite{blinkiewicz2016square}, \textbf{RMLEditor}~\parencite{heyvaert2016rmleditor} and \textbf{Map-On}~\parencite{sicilia2017map}. These editors provide a graph overview of the mapping while constructing it, while also showing the data sources and ontology. All of them are able to create R2RML mappings, and in addition, the RMLEditor can also produce RML mappings. There are also non open-source editors developed by companies that use a graph layout, such as \textbf{\url{gra.fo}}\cref{foot:gra.fo} and the \textbf{Stardog designer}\cref{foot:stardog-designer}. The former works with R2RML, and the latter with the Stardog proprietary syntax, SMS2, described in the next section. 

%~\parencite{fu2013tree-vs-graph} 

Some tools were also developed that follow an alternative approach to the tree and graph layouts, broadly used in the semantic web supporting applications. This approach comprises the use of building blocks or templates (\cref{fig:chp2_visual-editors}c), that are the components to build a mapping between data source and ontology. The first editor following this approach was proposed in \cite{sengupta2013editing}, later refined in \cite{pinkel2014best}. \textbf{OntopPro}~\parencite{calvanese2017ontop} released a Protege plugin to create and edit mappings (in their proprietary language) with templates, allowing also the creation of RDF triples and running SPARQL queries. \textbf{Juma}~\parencite{junior2017juma} and \textbf{RMLx}~\parencite{aryan2017rmlx} allow building R2RML and RML mappings respectively with building blocks, correspondent to the different parts of the mappings. There are also a couple of examples that enable this kind of visualization for mapping construction but using a proprietary language, \textbf{Ontopic Studio}\cref{foot:ontopic-studio} and \textbf{Eccenca Corporate Memory}\cref{foot:eccenca}. However, despite providing a friendly interface for users with non-technical profiles, the uptake of visual editors is limited.  

\subsection{Serializations}
\label{sec:chp2_serializations}

As an alternative to visual approaches, text-based user-friendly serializations were developed. This approach suited most practitioners with technical profiles or with preferences for a text oriented environment. 
\textbf{SML}~\parencite{Stadler2015sml} was developed as user-friendly syntax for R2RML. It provides an SPARQL-based syntax, enhancing the simplicity for writing mappings but maintaining the same expressiveness. 
The virtual KG processor \textbf{Ontop}~\parencite{calvanese2017ontop} also provides a simplified serialization for R2RML, which combines the Turtle syntax for triple generation and SQL for data access. 
The Stardog triplestore developed another SPARQL-based proprietary serialization, the Stardog Mapping Syntax 2 (\textbf{SMS2})\cref{foot:sms2}. 

Later on, as more mapping languages emerged, new serializations with a broader language compliance range emerged. This is the case of \textbf{XRM}\cref{foot:xrm} (Expressive RDF Mapper), that provides a unique syntax for writing R2RML, RML and CSVW mappings. It can be used with a service plugin integrated into common text editors, Visual Studio Code and Eclipse. This service warns the users about errors in the mapping while writing, and translates the mapping rules into one of the aforementioned languages. 

\textbf{YARRRML}~\parencite{Heyvaert2018yarrrml} was developed as a YAML-based\footnote{\url{https://yaml.org/spec/1.2.2/}} user-friendly syntax for RML. 
\cref{lst:chp2_yarrrml-mapping} shows an example of a YARRRML mapping equivalent to the RML example shown in \cref{lst:chp2_rml-mapping}. 
YARRRML mappings need as well the declaration of the prefixes at the beginning of the document, which is done using the \texttt{prefixes} key (Lines 1-2). The mapping rules are defined within the \texttt{mappings} key (Lines 4-13), which are aggregated in a rule set with keys named by the user (e.g. \texttt{athletes} key, Line 5). Each mapping rule set describes the input data sources (\texttt{sources} key, Lines 6-7), subject (\texttt{s} key, Line 8) and predicate-object pairs (\texttt{po} key, Lines 9-11).
This serialization can be used with the Matey\footnote{\url{https://rml.io/yarrrml/matey/}}, a web service able to translate YARRRML into RML mapping files, or directly generate RDF triples. 

\begin{captionedlisting}{lst:chp2_yarrrml-mapping}{YARRRML mapping to generate the RDF graph in \cref{lst:chp2_r2rml-result-rdf} with data from the JSON file shown in \cref{lst:chp2_json-example}. This mapping translates into the RML mapping shown in \cref{lst:chp2_rml-mapping}.}
\centering
{\begin{lstlisting}[language=yarrrml]
prefixes:
 ex: "http://example.com/ns#"

mappings:
  athletes:
    sources:
      - ["data.json~jsonpath", "$\dollar$.*"]
    s: http://example.com/athlete/$\dollar$(RANK)
    po:
      - [a, ex:Athlete]
      - [ex:name, $\dollar$(NAME)]
      - [ex:rank, $\dollar$(RANK)]
      - [ex:mark, $\dollar$(MARK)]
\end{lstlisting}}
\end{captionedlisting}

These serializations are in general widely adopted as they have proven useful for facilitating the writing of mapping rules. Hence, as mapping languages change and incorporate new features, these serializations need to be updated as well.

\section{Knowledge Graph Lifecycle}
\label{sec:chp2_kg_lifecycle}

\section{Conclusions and limitations of the State of the Art \textcolor{red}{-- By 2/2}}

\ana{hay que reformular, pero estas serian las ideas} 

\begin{itemize}
    \item \textit{identification and characterization of expressivenes and features of current mapping languages to identify needs and challenges in KGC with declarative approaches}
    \item \textit{for mapping creation: visual approaches not really adopted, but serializations not really suitable for non-technical profies, find the middle ground there}
    \item \textit{to motivate the adoption of declarative KGC technologies, see in which parts or how can be involved in other parts of KG life cycle and if it really can improve the process}
\end{itemize}

\chapter{Objectives and Contributions}
\label{chapter:objectives}

This chapter presents the main objectives of this thesis and identifies the contributions to the state of the art. We enumerate the assumptions considered in this work, describe the main hypothesis and delimit the scope of the thesis describing the restrictions.

\section{Objectives}
\label{sec:chp3-objectives}

The general objective of this thesis is to \textit{improve the mapping languages regarding expressiveness and usage to comply with the current evolving needs in knowledge graph construction}. In order to achieve this foal, the following sub-objectives are defined:

\begin{enumerate}
    %\item[\textbf{O1.}] To analyse, understand and gather the capabilities of the mapping languages for KG construction from heterogeneous data sources.
    \item[\textbf{O1.}] To understand and gather the necessities for knowledge graph construction from heterogeneous data sources.
    \item[\textbf{O2.}] To help knowledge engineers and domain experts to build mapping documents providing the means for a user-friendly experience.
    \item[\textbf{O3.}] To assess the role of mapping technologies for improving the evolution of knowledge graphs. 
\end{enumerate}

In order to achieve the first objective, the following open research problem must be solved:

\section{Contributions}
\label{sec:chp3-contributions}

In this work, we describe the solutions corresponding to the objectives and open research problems described in \cref{sec:chp3-objectives}. We present as follows the contributions that support the advance of the current state of the art regarding the first objective (understand and gather the needs for KG construction):

\begin{enumerate}
    \item[\textbf{C1.1.}] Comparison framework analysing the sota mapping languages
    \item[\textbf{C1.2.}] Identification and definition of requirements for KGC 
    \item[\textbf{C1.3.}] Implementation in formal language these requirements
    \item[\textbf{C1.4.}] Uptake of requirements and new needs in KGC into RML, focus on RML-star
\end{enumerate}

Regarding the second objective (to help knowledge engineers and domain experts to build mappings), we present new advances with the following contributions:

\begin{enumerate}
    \item[\textbf{C2.1.}] Proposal of spec of mapping rules in spreadsheets
    \item[\textbf{C2.2.}] Update of YARRRML w.r.t. updates in RML 
\end{enumerate}

Finally, with regard to the third objective (to assess the role of mapping technologies in KG evolution), this work presents the following contribution:

\begin{enumerate}
    \item[\textbf{C3.1.}] Identification of situations where mapping-compliant technologies work well with regards to other usual technologies involved in the evolution of knowledge graphs.  where they fail so it can be improved?
\end{enumerate}


\section{Assumptions}
\label{sec:chp3-assumptions}
Our work is based on the assumptions listed below. These assumptions provide a background to facilitate the comprehension of the decisions taken during the development of this work. 


\begin{enumerate}
    \item[\textbf{A1}] Mapping languages have human readable documentation
    \item[\textbf{A2}] The target schema (ontology) for creating the mapping documents is available and implemented in OWL or RDFS. 
\end{enumerate}


\section{Hypothesis}
\label{sec:chp3-hypothesis}

After the identification of the assumptions, we can describe the research hypothesis  of this thesis. They cover the general characteristics of the contributions \ana{si?}

\begin{enumerate}
    \item[\textbf{H1}] The expressiveness of mapping languages can be gathered and represented in a formal language (i.e. an ontology).
    \item[\textbf{H2}] New needs in KG construction ¿and updates in the RDF schema?  can be implemented in current mapping languages
    \item[\textbf{H3}] Writing the mapping rules in spreadsheets can improve the user experience for practitioners of different backgrounds with respect to other approaches for writting mappings
    \item[\textbf{H4}] Mappings can help enhance in the evolution of KGs withing the KG lifecycle, not only in construction.
\end{enumerate}


\section{Restrictions}
\label{sec:chp3-restrictions}

Finally, there is a set of restrictions that describe the limitations and define the future work objectives.

\begin{enumerate}
    \item[\textbf{R1}] \ana{algo de la eleccion de lenguajes?}
    \item[\textbf{R2}] The requirements in KGC are considered until summer 2023, time of writing this document. The updates made between this time until its publication are not considered.
    \item[\textbf{R3}] The representation of mapping language features that provide a procedural language (loops, embedded clauses) is out of the scope
    \item[\textbf{R4}] The spreadsheet template consider the features of RML v2014 and the current RML-FNML spec
    \item[\textbf{R5}] THe YARRRML updates include all RMLv2 modules except for the RML-CC module.
    \item[\textbf{R6}] The changes considered in KG evolution are a result of the schema changes of a switch in reification approach, not all possible modifications in KGs are considered.
    \item[\textbf{R7}] The evolution considered involves only changes in the schema of the KG, the data represented in the KG is stable with no additions or deletions, just change in how it is structured.
\end{enumerate}



\chapter{Mappings Languages for Knowledge Graph Construction}
\label{chapter:mappings}

Mappings are the key element for the Knowledge Graph construction process to enhance maintainability, understandability and reproducibility. This chapter first presents an extensive analysis of current mapping languages in the form of a comparison framework. Based on this comparison, a set of requirements are extracted and used for building an ontology that aims at gathering the expressiveness of current mapping languages. Finally, we present how a widely adopted mapping language has adopted the requirements identified, focusing on the features toconstruct RDF-star graphs.

%\ana{al igual esta intro se puede extender más, explicando un poco más cada parte y situandola en su contexto, que no sea una enumeración de las subsecciones.}

\section{Comparison framework}
\label{sec:chp4_framework}


This section presents a comparison framework that collects and analyzes the main features included in mapping languages. The diversity of the languages that have been analyzed is crucial for understanding the needs for constructing knowledge graphs from heterogeneous data sources. Thus, we can extract relevant shared features and requirements along with the peculiarities of each language. 

%\ana{igual los siguientes tres párrafos pueden componer la 'metodología' de la que habla oscar?}

\subsection{Methodology}

The framework presented in this section analyzes languages from the three categories identified in \cref{sec:chp2_declarative_kgc}. 
The selected languages fulfill the following requirements: 
(i) are widely used, relevant and/or include novel or unique features; 
(ii) are currently maintained, and not deprecated; 
(iii) are not a serialization or a user-friendly representation of another language; and 
(iv) are not tailored for a specific use case. 
For instance, D2RQ~\parencite{bizer2004d2rq} and R$_2$O~\parencite{barrasa2004r2o} are not included since they were superseded by R2RML, which is analysed in the comparison. 
T2WML is specific for transforming data following only the schema of Wikidata, and WoT mappings for the IoT domain, reason for which they are also excluded.
And lastly, NORSE is not included either since it is conceived for being generated by translating from RML mappings. 

%% esto igual no, porque ya se dice antes que no a las serializaciones y se presentan como tal en el estado del arte
%YARRRML~\parencite{Heyvaert2018yarrrml} and XRM~\parencite{xrm} are not included either, due to the fact that they provide a syntax for other already included languages (RML and R2RML; CSVW also for XRM).


The following RDF-based languages are included: R2RML~\parencite{das2012r2rml}, RML~\parencite{Dimou2014rml}, KR2RML~\parencite{slepicka2015kr2rml}, xR2RML~\parencite{michel2015xr2rml}, R2RML-F~\parencite{debruyne2016r2rmlf}, FunUL~\parencite{junior2016funul},  XLWrap~\parencite{langegger2009xlwrap}, CSVW~\parencite{Tennison2015csvw}, D2\-RML~\parencite{chortaras2018d2rml} and R2RMLcc~\parencite{debruyne2017R2RML-collections}. The analyzed SPARQL-based languages are: XSPARQL~\parencite{Bischof2012xsparql}, TARQL~\parencite{tarql},  SPARQL-Gene\-rate~\parencite{Lefrancois2017sparqlgenerate} and Facade-X~\parencite{daga2021facade}. Finally, we selected the following languages based on other formats: ShExML~\parencite{Garcia-Gonzalez2020shexml}, Helio Mappings~\parencite{cimmino2022helio} and D-REPR~\parencite{Vu2019d-repr}.  

The framework has been built as a result of analyzing the common features of these mapping languages, and also the specific features that make them unique and suitable for some scenarios. It includes information on data sources, general features for the construction of RDF graphs, and features related to the creation of subjects, predicates, and objects. In the following subsections, the features of each part of the framework are explained in detail. The language comparison for data sources is provided in \cref{tab:chp4_sources}, for triples creation in \cref{tab:chp4_spo}, and for general features in \cref{tab:chp4_metarules}. 

These languages have been analyzed based on their official specification, documentation, or reference paper (listed in \cref{tab:chp2_languages_summary}). Specific implementations and extensions that are not included in the official documentation are not considered in our framework. The cells (i.e. language feature) marked "*"  in the framework tables indicate that there are non-official implementations or extensions that include the feature.

%Throughout the section, there are examples showing how different languages use the analyzed features. The example is built upon two input sources: an online JSON file, "coordinates.json", with geographical coordinates (\ref{fig:ex_json}); and a table from a MySQL database, "cities" (\ref{fig:ex_rdb}). The reference ontology is depicted in \ref{fig:ex_onto}. It represents information about cities and their locations. The expected RDF output of the data transformation is shown in \ref{lst:output}. Each mapping represents only the relevant rules that the subsection describes. The entire mapping can be found in the examples section of the ontology documentation\ref{foot:cmlink}. 


\begin{sidewaystable}[]
\centering
\caption[Comparison framework: Data source description]{Data retrieval and data source expression for the analysed mapping languages from the references stated in \cref{tab:chp2_languages_summary}. (*) indicates features not explicitly declared in the language, but that are implemented by compliant tools.}
\label{tab:chp4_sources}
\resizebox{1\textwidth}{!}{%
\def\arraystretch{2}
{\Huge
\begin{tabular}{l|l|c|c|c|c|c|c|c|c|c|c|c|c|c|c|c|c|c}
\toprule
\multicolumn{2}{c|}{\textbf{Feature}} & \textbf{ShExML} & \textbf{XSPARQL} & \textbf{TARQL} & \textbf{CSVW} & \textbf{R2RML} & \textbf{RML} & \textbf{KR2RML} & \textbf{xR2RML} & \textbf{\begin{tabular}[c]{@{}c@{}}SPARQL-\\ Generate\end{tabular}} & \textbf{R2RML-F} & \textbf{FunUL} & \textbf{Helio} &  \textbf{D-REPR} & \textbf{XLWrap} & \textbf{D2RML} & \textbf{Facade-X} & \textbf{R2RMLcc}  \\ \midrule
\multirow{5}{*}{\begin{tabular}[c]{@{}l@{}}\textbf{Retrieval} \\ \textbf{of data}\end{tabular}} & \textbf{Streams} & \xmark & \xmark & \xmark & \xmark & \xmark & \checkmark*\footnote{Implemented by RMLSreamer, available at \scriptsize\url{https://github.com/RMLio/RMLStreamer}.} & \xmark & \xmark & \checkmark & \xmark & \xmark & \xmark &  \xmark & \xmark & \xmark & \checkmark & \xmark \\ \cmidrule{2-19} 
 & \begin{tabular}[c]{@{}l@{}}\textbf{Synchronous} \\ \textbf{sources}\end{tabular} & \checkmark & \checkmark & \checkmark & \checkmark & \checkmark & \checkmark & \checkmark & \checkmark & \checkmark & \checkmark & \checkmark & \checkmark &  \checkmark & \checkmark & \checkmark & \checkmark & \checkmark \\ \cmidrule{2-19} 
 & \begin{tabular}[c]{@{}l@{}}\textbf{Asynchronous} \\ \textbf{sources}\end{tabular} & -- & -- & -- & -- & -- & -- & -- & -- & Events, Periodic & -- & -- & Periodic &  -- & -- & -- & -- & -- \\ \midrule
\multirow{11}{*}{\begin{tabular}[c]{@{}l@{}}\textbf{Expressing} \\ \textbf{data sources}\end{tabular}} & \textbf{Security terms} & -- & -- & -- & -- & \begin{tabular}[c]{@{}c@{}}Basic \\ (DB)\end{tabular} & \begin{tabular}[c]{@{}c@{}}Basic \\ (DB)\end{tabular} & \begin{tabular}[c]{@{}c@{}}Basic \\ (DB)\end{tabular} & \begin{tabular}[c]{@{}c@{}}Basic \\ (DB)\end{tabular} & -- & \begin{tabular}[c]{@{}c@{}}Basic \\ (DB)\end{tabular} & -- & \begin{tabular}[c]{@{}c@{}}API Key, OAuth2, \\ Bearer, Basic\end{tabular} &  -- & -- & \begin{tabular}[c]{@{}c@{}}Basic \\ (DB)\end{tabular} & --  & \begin{tabular}[c]{@{}c@{}}Basic \\ (DB)\end{tabular}\\ \cmidrule{2-19} 
 & \textbf{Encoding} & \xmark & \xmark & \checkmark*\footnote{Command line input option \texttt{---encoding}~\parencite{tarql}.} & \checkmark & \xmark & \checkmark & \xmark & \xmark & \xmark & \xmark & \xmark & \checkmark &  \xmark & \xmark & \xmark & \checkmark & \xmark \\ \cmidrule{2-19} 
 & \textbf{MIME Type} & \xmark & \xmark & \xmark & \xmark & \xmark & \xmark & \xmark & \xmark & \xmark & \xmark & \xmark & \checkmark &  \xmark & \xmark & \xmark & \checkmark  & \xmark \\ \cmidrule{2-19} 
 & \begin{tabular}[c]{@{}l@{}}\textbf{Features} \\ \textbf{describing}\\ \textbf{data}\end{tabular} & \begin{tabular}[c]{@{}l@{}}Iterator, \\ Queries\end{tabular} & -- & \begin{tabular}[c]{@{}c@{}}Delimiter, \\ Separator\end{tabular} & \begin{tabular}[c]{@{}c@{}}Delimiter, \\ Separator, Regex\end{tabular} & Queries & \begin{tabular}[c]{@{}c@{}}Delimiter, Regex, \\ Iterator, Queries, \\ Separator\end{tabular} & Queries & \begin{tabular}[c]{@{}c@{}}Regex, \\ Iterator, Queries\end{tabular} & \begin{tabular}[c]{@{}c@{}}Delimiter, Regex, \\ Iterator, Queries, \\ Separator\end{tabular} & \begin{tabular}[c]{@{}c@{}}Iterator, \\ Queries\end{tabular} & \begin{tabular}[c]{@{}c@{}}Iterator, \\ Queries\end{tabular} & \begin{tabular}[c]{@{}c@{}}Delimiter, Regex, \\ Iterator, Queries, \\ Separator\end{tabular} &  \begin{tabular}[c]{@{}c@{}}Delimiter, \\ Regex, Iterator\end{tabular} & Separator & \begin{tabular}[c]{@{}c@{}}Delimiter, Regex, \\ Iterator, Queries\end{tabular} & \begin{tabular}[c]{@{}c@{}}Delimiter, Regex, \\ Iterator, Queries, \\ Separator\end{tabular} & Queries \\ \cmidrule{2-19} 
 & \begin{tabular}[c]{@{}c@{}}\textbf{Retrieval} \\ \textbf{protocol}\end{tabular} & \begin{tabular}[c]{@{}c@{}}file, http(s), \\ odbc/jdbc\end{tabular} & file & file & file, http(s) & \begin{tabular}[c]{@{}c@{}}file, http(s), \\ odbc/jdbc\end{tabular} & \begin{tabular}[c]{@{}c@{}}file, http(s), \\ odbc/jdbc\end{tabular} & \begin{tabular}[c]{@{}c@{}}file, \\ odbc/jdbc\end{tabular} & \begin{tabular}[c]{@{}c@{}}file, \\ odbc/jdbc\end{tabular} & \begin{tabular}[c]{@{}c@{}}file, http(s), \\ odbc/jdbc \\ WebSocket, MQTT\end{tabular} & \begin{tabular}[c]{@{}c@{}}file, http(s), \\ odbc/jdbc\end{tabular} & file, http(s) & \begin{tabular}[c]{@{}c@{}}file, \\ any URI-based\end{tabular} &  file & file & \begin{tabular}[c]{@{}c@{}}file, http(s), \\ odbc/jdbc\end{tabular} & file, http(s) & \begin{tabular}[c]{@{}c@{}}file, http(s), \\ odbc/jdbc\end{tabular} \\ \cmidrule{2-19} 
 & \textbf{Data formats} & \begin{tabular}[c]{@{}c@{}}Tabular, \\ Tree, Graph\end{tabular} & \begin{tabular}[c]{@{}c@{}}Tree \\ (XML)\end{tabular} & \begin{tabular}[c]{@{}c@{}}Tabular \\ (CSV)\end{tabular} & Tabular & Tabular & \begin{tabular}[c]{@{}c@{}}Tabular, \\ Tree, Graph\end{tabular} & \begin{tabular}[c]{@{}c@{}}Tabular, \\ Tree\end{tabular} & \begin{tabular}[c]{@{}c@{}}Tabular, \\ Tree\end{tabular} & \begin{tabular}[c]{@{}c@{}}Tabular, Tree, \\ Plain Text, Graph\end{tabular} & Tabular & \begin{tabular}[c]{@{}c@{}}Tabular, \\ Graph\end{tabular} & \begin{tabular}[c]{@{}c@{}}Tabular, Tree, \\ Plain Text, Graph\end{tabular} &  \begin{tabular}[c]{@{}c@{}}Tabular (CSV), \\ Tree\end{tabular} & \begin{tabular}[c]{@{}c@{}}Tabular \\ (CSV, Excel)\end{tabular} & \begin{tabular}[c]{@{}c@{}}Tabular, Tree, \\ Plain Text, Graph\end{tabular} & \begin{tabular}[c]{@{}c@{}}Tabular, Tree, \\ Plain Text, Graph\end{tabular} & Tabular \\ \bottomrule
\end{tabular}%
}
}
\end{sidewaystable}

\subsection{Data Sources Description}


\cref{tab:chp4_sources} shows the ability of each mapping language to describe a data source in terms of retrieval, description, security, data format and protocol. 


\noindent\paragraph{\textbf{Data Retrieval.}} Data from data sources may be retrieved in a continuous manner (e.g., \textit{Streams}),  periodically (e.g., \textit{Asynchronous sources}), or just once, when the mapping is executed (e.g., \textit{Synchronous sources}). As shown in \cref{tab:chp4_sources}, all mapping languages are able to represent synchronous data sources. Additionally, SPARQL-Generate and Helio are able to represent periodical data sources, and SPARQL-Generate also represents continuous data sources (e.g. \texttt{it:WebSocket()} in SPARQL-Generate). Other languages do not explicitly express that feature in the language, but a compliant engine may implement it.

\noindent\paragraph{\textbf{Representing Data Sources.}} Extracting and retrieving heterogeneous data involves several elements that mapping languages need to consider: \textit{Security terms} to describe access (e.g., relational databases (RDB), API Key, OAuth2, etc); \textit{Retrieval protocol} such as local files, HTTP(S), JDBC, etc; \textit{Features that describe the data} to define particular characteristics of the source data (e.g. queries, regex, iterator, delimiter, etc); \textit{Data formats} such as CSV, RDB, and JSON; \textit{Encoding} and content negotiation (i.e. \textit{MIME Type}). 

Half of the languages do not allow the definition of security terms. Some languages are specific for RDB terms (R2RML and extensions, with \texttt{rr:logical\-Table}), and only Helio can define security terms. These two languages are also the only ones that allow the specification of MIME Types, and can also specify the encoding along with TARQL and CSVW (e.g. \texttt{csvw:encoding} attribute of \texttt{csvw:Dialect} in CSVW). 

Regarding protocols, all languages consider the use of local files. It is highly usual to consider HTTP(s) and database access (especially with the ODBC and JDBC protocols). Only XSPARQL, TARQL, D-REPR, and XLWrap describe exclusively local files. 

The features provided by each language are closely related to the data formats that are covered. Queries are usual for relational databases and NoSQL document stores and iterators for tree-like formats. Some languages also enable the description of delimiters and separators for tabular formats (e.g., CSVW defines the class \texttt{Dialect} to describe these features; this class is reused by RML), and finally, less common Regular Expressions can be defined to match specific parts of the data in languages such as CSVW, SPARQL-Generate, Helio, D-REPR, and D2RML (e.g., \texttt{RegexHandler} in Helio, \texttt{format} in CSVW). 

The most used format is tabular (RDB and CSV). Some languages can also process RDF graphs such as ShExML, RML, SPARQL-Generate, Helio, and D2RML (e.g. \texttt{QUERY} in ShExML,  SPARQL service description\footnote{\url{http://www.w3.org/ns/sparql-service-description\#}} in RML), and the last three languages can also process plain text.


%\noindent\paragraph{\textbf{Data Sources Example.}} This example shows how ShExML and R2RML describe heterogeneous data sources. The sources are a table called "cities" (\ref{fig:ex_rdb}) that belongs to a relational database that stores information about cities: name, population, zipcode and year in which the data was updated; and a JSON file "coordinates.json" (\ref{fig:ex_json}) available online that contains the latitude and longitude of the central point of each city. R2RML is only able to describe the database table (\ref{lst:shexml_source}); instead ShExML is able to describe both the RDB and the online JSON file (\ref{lst:shexml_source}).






\begin{sidewaystable}[]
\centering
\caption[Comparison framework: Triple generation]{Features for subject, predicate, and object generation of the studied mapping languages from the references stated in \cref{tab:chp2_languages_summary}.}
\label{tab:chp4_spo}
\resizebox{\textwidth}{!}{%
\def\arraystretch{2}
\begin{tabular}{l|l|l|c|c|c|c|c|c|c|c|c|c|c|c|c|c|c|c|c}
\toprule
\multicolumn{3}{c|}{\textbf{Feature}} & \textbf{ShExML} & \textbf{XSPARQL} & \textbf{TARQL} & \textbf{CSVW} & \textbf{R2RML} & \textbf{RML} & \textbf{KR2RML} & \textbf{xR2RML} & \textbf{\begin{tabular}[c]{@{}c@{}}SPARQL-\\Generate\end{tabular}} & \textbf{R2RML-F} & \textbf{FunUL} & \textbf{Helio} &  \textbf{D-REPR} &\textbf{ XLWrap} & \textbf{D2RML} & \textbf{Facade-X} & \textbf{R2RMLcc} \\ \midrule
\multicolumn{1}{c|}{\multirow{8}{*}{\textbf{Subject}}} & \multicolumn{2}{l|}{\textbf{Constant}} & IRI & BN, IRI & BN, IRI & IRI & BN, IRI & BN, IRI & -- & BN, IRI & BN, IRI & BN, IRI & BN, IRI & IRI &  BN, IRI & BN, IRI & BN, IRI & BN, IRI & BN, IRI \\ \cmidrule{2-20} 
\multicolumn{1}{c|}{} & \multirow{6}{*}{\textbf{Dynamic}} & \textbf{RDF Resource} & IRI & BN, IRI & BN, IRI & IRI & BN, IRI & BN, IRI & IRI & BN, IRI & IRI & BN, IRI & BN, IRI & IRI &  BN, IRI & BN, IRI & BN, IRI & IRI  & BN, IRI \\ \cmidrule{3-20} 
\multicolumn{1}{c|}{} &  & \textbf{Data Reference} & \begin{tabular}[c]{@{}c@{}}1..* Ref\\ 1..* Format\end{tabular} & \begin{tabular}[c]{@{}c@{}}1..* Ref\\ 1..1 Format\end{tabular} & \begin{tabular}[c]{@{}c@{}}1..* Ref\\ 1..1 Format\end{tabular} & \begin{tabular}[c]{@{}c@{}}1..* Ref\\ 1..1 Format\end{tabular} & \begin{tabular}[c]{@{}c@{}}1..* Ref\\ 1..1 Format\end{tabular} & \begin{tabular}[c]{@{}c@{}}1..* Ref\\ 1..1 Format\end{tabular} & \begin{tabular}[c]{@{}c@{}}1..* Ref\\ 1..* Format\end{tabular} & \begin{tabular}[c]{@{}c@{}}1..* Ref\\ 1..* Format\end{tabular} & \begin{tabular}[c]{@{}c@{}}1..* Ref\\ 1..1 Format\end{tabular} & \begin{tabular}[c]{@{}c@{}}1..* Ref\\ 1..1 Format\end{tabular} & \begin{tabular}[c]{@{}c@{}}1..* Ref\\ 1..1 Format\end{tabular} & \begin{tabular}[c]{@{}c@{}}1..* Ref\\ 1..* Format\end{tabular} &  \begin{tabular}[c]{@{}c@{}}1..* Ref\\ 1..1 Format\end{tabular} & \begin{tabular}[c]{@{}c@{}}1..* Ref\\ 1..1 Format\end{tabular} & \begin{tabular}[c]{@{}c@{}}1..* Ref\\ 1..1 Format\end{tabular} & \begin{tabular}[c]{@{}c@{}}1..* Ref\\ 1..* Format\end{tabular} & \begin{tabular}[c]{@{}c@{}}1..* Ref\\ 1..1 Format\end{tabular} \\ \cmidrule{3-20} 
\multicolumn{1}{c|}{} &  & \textbf{Data Sources} & 1..* & 1..* & 1..1 & 1..1 & 1..1 & 1..1 & 1..* & 1..* & 1..* & 1..1 & 1..1 & 1..* &  1..1 & 1..1 & 1..1  & 1..* & 1..1 \\ \cmidrule{3-20} 
\multicolumn{1}{c|}{} &  & \textbf{Hierarchy Iteration} & \checkmark & \checkmark & \xmark & \xmark & \xmark & \checkmark & \checkmark & \xmark & \checkmark & \xmark & \xmark & \xmark &  \checkmark & \xmark & \checkmark & \checkmark & \xmark \\ \cmidrule{3-20} 
\multicolumn{1}{c|}{} &  & \textbf{Functions} & -- & 1..* & 1..* & -- & -- & 1..* & 1..* & -- & 1..* & 1..* & 1..* & 1..* &  1..* & 1..* & 1..* & 1..* & -- \\ \midrule
\multirow{8}{*}{\textbf{Predicate}} & \multicolumn{2}{l|}{\textbf{Constant}} & IRI & IRI & IRI & IRI & IRI & IRI & IRI & IRI & IRI & IRI & IRI & IRI &  IRI & IRI & IRI & IRI & IRI  \\ \cmidrule{2-20} 
 & \multirow{6}{*}{\textbf{Dynamic}} & \textbf{RDF Resource} & -- & IRI & IRI & IRI & IRI & IRI & IRI & IRI & IRI & IRI & IRI & IRI &  IRI & -- & IRI & IRI & IRI \\ \cmidrule{3-20} 
 &  & \textbf{Data Reference} & -- & \begin{tabular}[c]{@{}c@{}}1..* Ref\\ 1..1 Format\end{tabular} & \begin{tabular}[c]{@{}c@{}}1..* Ref\\ 1..1 Format\end{tabular} & \begin{tabular}[c]{@{}c@{}}1..* Ref\\ 1..1 Format\end{tabular} & \begin{tabular}[c]{@{}c@{}}1..* Ref\\ 1..1 Format\end{tabular} & \begin{tabular}[c]{@{}c@{}}1..* Ref\\ 1..1 Format\end{tabular} & \begin{tabular}[c]{@{}c@{}}1..* Ref\\ 1..* Format\end{tabular} & \begin{tabular}[c]{@{}c@{}}1..* Ref\\ 1..1 Format\end{tabular} & \begin{tabular}[c]{@{}c@{}}1..* Ref\\ 1..1 Format\end{tabular} & \begin{tabular}[c]{@{}c@{}}1..* Ref\\ 1..1 Format\end{tabular} & \begin{tabular}[c]{@{}c@{}}1..* Ref\\ 1..1 Format\end{tabular} & \begin{tabular}[c]{@{}c@{}}1..* Ref\\ 1..* Format\end{tabular} &  \begin{tabular}[c]{@{}c@{}}1..* Ref\\ 1..1 Format\end{tabular} & -- & \begin{tabular}[c]{@{}c@{}}1..* Ref\\ 1..1 Format\end{tabular} & \begin{tabular}[c]{@{}c@{}}1..* Ref\\ 1..* Format\end{tabular} & \begin{tabular}[c]{@{}c@{}}1..* Ref\\ 1..1 Format\end{tabular} \\ \cmidrule{3-20} 
 &  & \textbf{Data Sources} & -- & 1..1 & 1..1 & 1..1 & 1..1 & 1..1 & 1..* & 1..1 & 1..* & 1..1 & 1..1 & 1..* &  1..1 & -- & 1..1 & 1..* & 1..1  \\ \cmidrule{3-20} 
 &  & \textbf{Hierarchy Iteration} & \xmark & \xmark & \xmark & \xmark & \xmark & \checkmark & \checkmark & \xmark & \xmark & \xmark & \xmark & \xmark &  \checkmark & \xmark & \checkmark & \checkmark & \xmark \\ \cmidrule{3-20} 
 &  & \textbf{Functions} & -- & 1..* & 1..* & -- & -- & 1..* & 1..* & -- & 1..* & 1..* & 1..* & 1..* &  1..* & -- & 1..* & 1..* & -- \\ \midrule
\multirow{11}{*}{\textbf{Object}} & \multicolumn{2}{l|}{\textbf{Constant}} & IRI, Literal & \begin{tabular}[c]{@{}c@{}}BN, IRI, \\ Literal\end{tabular} & \begin{tabular}[c]{@{}c@{}}BN, IRI, \\ Literal\end{tabular} & IRI, Literal & IRI, Literal & IRI, Literal & IRI, Literal & \begin{tabular}[c]{@{}c@{}}BN, IRI, Literal,\\ List, Container\end{tabular} & \begin{tabular}[c]{@{}c@{}}BN, IRI, \\ Literal, List\end{tabular} & IRI, Literal & IRI, Literal & IRI, Literal &  \begin{tabular}[c]{@{}c@{}}BN, IRI, \\ Literal\end{tabular} & IRI, Literal & \begin{tabular}[c]{@{}c@{}}BN, IRI, \\ Literal\end{tabular} & \begin{tabular}[c]{@{}c@{}}BN, IRI, \\ Literal, List\end{tabular} & \begin{tabular}[c]{@{}c@{}}BN, IRI, Literal,\\ List, Container\end{tabular} \\ \cmidrule{2-20} 
 & \multirow{7}{*}{\textbf{Dynamic}} & \textbf{RDF Resource} & \begin{tabular}[c]{@{}c@{}}IRI, Literal,\\ Lists\end{tabular} & \begin{tabular}[c]{@{}c@{}}BN, IRI, \\ Literal\end{tabular} & \begin{tabular}[c]{@{}c@{}}BN, IRI, \\ Literal\end{tabular} & IRI, Literal & \begin{tabular}[c]{@{}c@{}}BN, IRI, \\ Literal\end{tabular} & \begin{tabular}[c]{@{}c@{}}BN, IRI, \\ Literal\end{tabular} & \begin{tabular}[c]{@{}c@{}}IRI, Literal,\\ List\end{tabular} & \begin{tabular}[c]{@{}c@{}}BN, IRI, Literal,\\ List, Container\end{tabular} & \begin{tabular}[c]{@{}c@{}}BN, IRI, \\ Literal, List\end{tabular} & \begin{tabular}[c]{@{}c@{}}BN, IRI, \\ Literal\end{tabular} & \begin{tabular}[c]{@{}c@{}}BN, IRI, \\ Literal\end{tabular} & IRI, Literal &  \begin{tabular}[c]{@{}c@{}}BN, IRI, \\ Literal\end{tabular} & IRI, Literal & \begin{tabular}[c]{@{}c@{}}BN, IRI, \\ Literal\end{tabular} & \begin{tabular}[c]{@{}c@{}}BN, IRI, \\ Literal, List\end{tabular} & \begin{tabular}[c]{@{}c@{}}BN, IRI, Literal,\\ List, Container\end{tabular} \\ \cmidrule{3-20} 
 &  & \textbf{Data Reference} & \begin{tabular}[c]{@{}c@{}}1..* Ref\\ 1..* Format\end{tabular} & \begin{tabular}[c]{@{}c@{}}1..* Ref\\ 1..1 Format\end{tabular} & \begin{tabular}[c]{@{}c@{}}1..* Ref\\ 1..1 Format\end{tabular} & \begin{tabular}[c]{@{}c@{}}1..* Ref\\ 1..1 Format\end{tabular} & \begin{tabular}[c]{@{}c@{}}1..* Ref\\ 1..1 Format\end{tabular} & \begin{tabular}[c]{@{}c@{}}1..* Ref\\ 1..1 Format\end{tabular} & \begin{tabular}[c]{@{}c@{}}1..* Ref\\ 1..* Format\end{tabular} & \begin{tabular}[c]{@{}c@{}}1..* Ref\\ 1..* Format\end{tabular} & \begin{tabular}[c]{@{}c@{}}1..* Ref\\ 1..1 Format\end{tabular} & \begin{tabular}[c]{@{}c@{}}1..* Ref\\ 1..1 Format\end{tabular} & \begin{tabular}[c]{@{}c@{}}1..* Ref\\ 1..1 Format\end{tabular} & \begin{tabular}[c]{@{}c@{}}1..* Ref\\ 1..* Format\end{tabular} &  \begin{tabular}[c]{@{}c@{}}1..* Ref\\ 1..1 Format\end{tabular} & \begin{tabular}[c]{@{}c@{}}1..* Ref\\ 1..1 Format\end{tabular} & \begin{tabular}[c]{@{}c@{}}1..* Ref\\ 1..1 Format\end{tabular} & \begin{tabular}[c]{@{}c@{}}1..* Ref\\ 1..* Format\end{tabular} & \begin{tabular}[c]{@{}c@{}}1..* Ref\\ 1..1 Format\end{tabular} \\ \cmidrule{3-20} 
 &  & \textbf{Data Sources} & 1..* & 1..* & 1..1 & 1..1 & 1..1 & 1..1 & 1..* & 1..* & 1..* & 1..1 & 1..1 & 1..* &  1..1 & 1..1 & 1..1 & 1..* & 1..1 \\ \cmidrule{3-20} 
 &  & \textbf{Hierarchy Iteration} & \checkmark & \checkmark & \xmark & \xmark & \xmark & \checkmark & \checkmark & \xmark & \checkmark & \xmark & \xmark & \xmark &  \checkmark & \xmark & \checkmark & \checkmark & \xmark \\ \cmidrule{3-20} 
 &  & \textbf{Functions} & 1 & 1..* & 1..* & -- & -- & 1..* & 1..* & -- & 1..* & 1..* & 1..* & 1..* &  1..* & 1..* & 1..* & 1..* & -- \\ \cmidrule{2-20} 
 & \multicolumn{2}{l|}{\textbf{Datatype and Language}} & \begin{tabular}[c]{@{}c@{}}static,\\ dynamic\end{tabular} & \begin{tabular}[c]{@{}c@{}}static,\\ dynamic\end{tabular} & \begin{tabular}[c]{@{}c@{}}static,\\ dynamic\end{tabular} & static & static & \begin{tabular}[c]{@{}c@{}}static,\\ dynamic\end{tabular} & -- & static & static & static & static & \begin{tabular}[c]{@{}c@{}}static,\\ dynamic\end{tabular} & static & -- & static & static & static \\ \bottomrule
\end{tabular}%
}
\end{sidewaystable}


\subsection{Triples Generation}
\cref{tab:chp4_spo} represents how different languages describe the generation of triples. We assess whether they generate the \textit{Subject}, \textit{Predicate}, and \textit{Object}: in (1) a \textit{Constant} manner, i.e. non-dependant on the data field to be created; or in (2) a \textit{Dynamic} manner, i.e. changing its value with each data field iteration. For \textit{Objects}, the possibility of adding \textit{Datatype and Language} tags is also considered; this feature assesses whether they can be added, and if they are added in a dynamic (changes with the data) or static (constant) manner. This table also analyzes the use and cardinality of transformation functions and the possibility of iterating over different nested level arrays (i.e., in tree-like formats).

The categories \textit{Constant} and \textit{RDF Resource} (the latter within \textit{Dynamic}) show which kind of resources can be generated by the language (i.e., IRI, Blank Node, Literal, List and/or Container). The \textit{Dynamic} category also considers: the \textit{Data References} (i.e. fields from the data source) that can appear with single of mixed formats; from how many \textit{Data Sources} (e.g. ```1:1" when only data from one file can be used) the term is generated; if \textit{Hierarchy Iteration} over different nested levels in tree-like formats is allowed; and if \textit{Functions} can be used to perform transformations on the data to create the term (e.g. \texttt{lowercase}, \texttt{toDate}, etc.).

\noindent\paragraph{\textbf{Subject Generation.}} Subjects can be IRIs or Blank Nodes (BN). This is well reflected in the languages, since, with a few exceptions that do not consider Blank Nodes, all languages are able to generate these two types of RDF resources, both constant and dynamically. All can generate a subject with one or more data references (e.g., in RML \texttt{rr:template "http://ex.org/\{id\}\-\{name\}"}), ShExML, xR2R\-ML, SPARQL-Generate, Facade-X, and Helio with different formats. For example, in xR2RML a CSV field that contains an array can be expressed as: \texttt{xrr:reference "Column(Mo\-vies)/JSONPath(\$.*)}. Part of the languages even allow generating subjects with more than one data source, this is the case of ShExML, XSPARQL, KR2RML, SPARQL-Generate, Facade-X, Helio and xR2RML. About a third of the languages allow hierarchy iterations (ShExML, XSPARQL, KR2RML, SPARQL-Generate, D-REPR, Facade-X and D2RML), and more than a half use functions with N:1 cardinality. Additionally, some of them even allow functions that can output more than one parameter (i.e., 1:N or N:M), but it is less usual.



\noindent\paragraph{\textbf{Predicate Generation.}} All languages can generate constant predicates as IRIs. Only four languages do not allow dynamic predicates (ShExML, and XLWrap). For those that do, they also allow more than one data reference. The languages that allow subject generation using multiple formats, data sources, functions, and hierarchy iterations, provide the same features for predicate generation.

\noindent\paragraph{\textbf{Object Generation.}} %There is a wider variety of RDF resources that the considered languages can generate,
Generally, languages can generate a wider range of resources for objects, since they can be IRIs, blank nodes, literals, lists, or containers. All of them can generate constant and dynamic literals and IRIs. Those languages that allow blank nodes in the subject also allow them in the object. Additionally, ShExML, KR2RML, SPARQL-Generate, Facade-X, R2RMLcc and xR2RML consider lists, and the last two languages also consider containers (e.g. \texttt{rr:termType xrr:RdfBag} in xR2RML). Data references, sources, hierarchy iterations, and functions remain the same as in subject generation. Lastly, datatype and language tags are not allowed in KR2RML and XLWrap; they are defined as constants in the rest of the languages, and dynamically in ShExML, XSPARQL, TARQL, RML, and Helio (e.g., \texttt{rml:languageMap} for dynamic language tags in RML).

%







\begin{sidewaystable}[]
\centering
%\rotatebox{180}{
\begin{minipage}{21.5cm}
\caption[Comparison Framework: General features]{Statements, linking rules, and function properties of the studied mapping languages from the references stated in \cref{tab:chp2_languages_summary}.}
\label{tab:chp4_metarules}
\resizebox{\textwidth}{!}{%
\def\arraystretch{1.5}
\begin{tabular}{c|l|c|c|c|c|c|c|c|c|c|c|c|c|c|c|c|c|c}
\toprule
\multicolumn{2}{c|}{\textbf{Feature}} & \textbf{ShExML} & \textbf{XSPARQL} & \textbf{TARQL} & \textbf{CSVW} & \textbf{R2RML} & \textbf{RML} & \textbf{KR2RML} & \textbf{xR2RML} & \textbf{\begin{tabular}[c]{@{}c@{}}SPARQL-\\Generate\end{tabular}} & \textbf{R2RML-F} & \textbf{FunUL} & \textbf{Helio} &  \textbf{D-REPR} & \textbf{XLWrap} & \textbf{D2RML} & \textbf{Facade-X} & \textbf{R2RMLcc} \\ \midrule
\multirow{8}{*}{\textbf{Statements}} & \textbf{Assign to named graphs} & static & -- & -- & -- & \begin{tabular}[c]{@{}c@{}}static, \\ dynamic\end{tabular} & \begin{tabular}[c]{@{}c@{}}static, \\ dynamic\end{tabular} & static & static & -- & \begin{tabular}[c]{@{}c@{}}static, \\ dynamic\end{tabular} & \begin{tabular}[c]{@{}c@{}}static, \\ dynamic\end{tabular} & -- &  -- & static & \begin{tabular}[c]{@{}c@{}}static, \\ dynamic\end{tabular} & --  & \begin{tabular}[c]{@{}c@{}}static, \\ dynamic\end{tabular} \\ \cmidrule{2-19} 
 & \begin{tabular}[c]{@{}l@{}}\textbf{Retrieve data from} \\  \textbf{one source}\end{tabular} & \checkmark & \checkmark & \checkmark & \checkmark & \checkmark & \checkmark & \checkmark & \checkmark & \checkmark & \checkmark & \checkmark & \checkmark &  \checkmark & \checkmark & \checkmark & \checkmark  & \checkmark \\ \cmidrule{2-19} 
 & \begin{tabular}[c]{@{}l@{}}\textbf{Retrieve data from} \\ \textbf{one or more sources}\end{tabular} & \checkmark & \checkmark & \xmark & \checkmark & \xmark & \xmark & \xmark & \xmark & \checkmark & \xmark & \xmark & \checkmark &  \checkmark & \xmark & \checkmark & \checkmark  & \xmark \\ \cmidrule{2-19} 
 & \begin{tabular}[c]{@{}l@{}}\textbf{Allow conditions} \\ \textbf{to form statements}\end{tabular} & \checkmark & \checkmark & \checkmark & \xmark & \xmark & \xmark & \xmark & \xmark & \checkmark & \xmark & \xmark & \xmark &  \xmark & \checkmark & \checkmark & \checkmark  & \xmark \\ \midrule
\multirow{12}{*}{\textbf{Linking rules}} & \textbf{Use one data reference} & \checkmark & \checkmark & \xmark & \checkmark & \checkmark & \checkmark & \xmark & \checkmark & \checkmark & \checkmark & \checkmark & \checkmark &  \checkmark & \xmark & \checkmark & \checkmark  & \checkmark \\ \cmidrule{2-19} 
 & \begin{tabular}[c]{@{}l@{}}\textbf{Use one or more} \\ \textbf{data references}\end{tabular} & \checkmark & \xmark & \xmark & \xmark & \xmark & \checkmark & \xmark & \checkmark & \checkmark & \xmark & \xmark & \checkmark &  \checkmark & \xmark & \xmark & \checkmark  & \xmark \\ \cmidrule{2-19} 
 & \textbf{No condition to link} & \checkmark & \checkmark & \xmark & \xmark & \checkmark & \checkmark & \xmark & \checkmark & \checkmark & \checkmark & \checkmark & \checkmark &  \xmark & \xmark & \checkmark & \checkmark  & \checkmark \\ \cmidrule{2-19} 
 & \textbf{Link with one condition} & \checkmark & \checkmark & \xmark & \xmark & \checkmark & \checkmark & \xmark & \checkmark & \checkmark & \checkmark & \checkmark & \checkmark &  \checkmark & \xmark & \checkmark & \checkmark  & \checkmark \\ \cmidrule{2-19} 
 & \begin{tabular}[c]{@{}l@{}}\textbf{Link with one or} \\ \textbf{more conditions}\end{tabular} & \xmark & \checkmark & \xmark & \xmark & \checkmark & \checkmark & \xmark & \checkmark & \checkmark & \checkmark & \checkmark & \checkmark &  \checkmark & \xmark & \checkmark & \checkmark  & \checkmark \\ \cmidrule{2-19} 
 & \begin{tabular}[c]{@{}l@{}}\textbf{Use only equal} \\ \textbf{function in condition}\end{tabular} & \checkmark & \checkmark & \xmark & \xmark & \checkmark & \checkmark & \xmark & \checkmark & \checkmark & \checkmark & \checkmark & \checkmark &  \checkmark & \xmark & \checkmark & \checkmark  & \checkmark \\ \cmidrule{2-19} 
 & \begin{tabular}[c]{@{}l@{}}\textbf{Use any similarity} \\ \textbf{function in condition}\end{tabular} & \xmark & \checkmark & \xmark & \xmark & \xmark & \checkmark & \xmark & \xmark & \xmark & \xmark & \xmark & \checkmark &  \xmark & \xmark & \checkmark & \checkmark  & \xmark \\ \midrule
\multirow{7}{*}{\textbf{Functions}} & \textbf{Cardinality} & 1:1, N:1 & \begin{tabular}[c]{@{}c@{}}1:1, N:1, \\ 1:N, N:M\end{tabular} & 1:1, N:1 & -- & -- & 1:1, N:1*\footnote{With the Function Ontology (FnO)~\parencite{DeMeester2017fno_dbpedia}}  & \begin{tabular}[c]{@{}c@{}}1:1, N:1, \\ 1:N, N:M\end{tabular} & -- & \begin{tabular}[c]{@{}c@{}}1:1, N:1, \\ 1:N, N:M\end{tabular} & 1:1, N:1 & 1:1, N:1 & 1:1, N:1 &  \begin{tabular}[c]{@{}c@{}}1:1, N:1, \\ 1:N, N:M\end{tabular} & \begin{tabular}[c]{@{}c@{}}1:1, N:1, \\ 1:N, N:M\end{tabular} & \begin{tabular}[c]{@{}c@{}}1:1, N:1, \\ 1:N, N:M\end{tabular} & \begin{tabular}[c]{@{}c@{}}1:1, N:1, \\ 1:N, N:M\end{tabular} & -- \\ \cmidrule{2-19} 
 & \textbf{Nested functions} & \xmark & \checkmark & \checkmark & \xmark & \xmark & \checkmark*$^{a}$ & \xmark & \xmark & \checkmark & \checkmark & \checkmark & \checkmark &  \checkmark & \checkmark & \checkmark & \checkmark  & \xmark \\ \cmidrule{2-19} 
 & \begin{tabular}[c]{@{}l@{}}\textbf{Functions belong} \\ \textbf{to a specification}\end{tabular} & \checkmark & \xmark & \checkmark & \xmark & \xmark & \checkmark*$^{a}$ & \xmark & \xmark & \checkmark & \xmark & \xmark & \checkmark &  \xmark & \checkmark & \xmark & \checkmark  & \xmark \\ \cmidrule{2-19} 
 & \begin{tabular}[c]{@{}l@{}}\textbf{Declare own} \\ \textbf{functions}\end{tabular} & \checkmark & \checkmark & \xmark & \xmark & \xmark & \checkmark*$^{a}$ & \xmark & \xmark & \checkmark & \checkmark & \checkmark & \xmark &  \xmark & \checkmark & \checkmark & \checkmark  & \xmark \\ \bottomrule
\end{tabular} 
}
\end{minipage}%}
\end{sidewaystable}



\subsection{General Features for Graph Construction}

\cref{tab:chp4_metarules} shows the features of mapping languages regarding the %generation of triples  
construction of RDF graphs such as \textit{linking rules}, \textit{metadata} or \textit{conditions}, assignment to \textit{named graphs}, and declaration of \textit{transformation functions} within the mapping. 

\noindent\paragraph{\textbf{Statements.}} General features that apply to statements are described in this section: the capability of a language to assign statements to \textit{named graphs}, to \textit{retrieve data from only one source} or \textit{more than one source}, and to apply \textit{conditions} that have to be met in order to create the statement (e.g. if the value of a field called "required" is \texttt{TRUE}, the triple is generated).

Most RDF-based languages allow static assignment to named graphs. R2RML, RML, R2RML-F, FunUL, and D2RML enable also dynamic definitions (e.g., \texttt{rr:graph\-Map} in R2RML and in its extensions). Theoretically, the rest of R2RML extensions should also implement this feature; however, to the best of our knowledge, it is not mentioned in their respective specifications. 

Allowing conditional statements is not usual; it is only considered in the SPARQL-based languages, XLWrap and D2RML (e.g. \texttt{xl:breakCo\-ndition} in XLWrap). Regarding data sources, all languages allow data retrieval from at least one source; ShExML, XSPARQL, CSVW, SPARQL-Generate, Facade-X, Helio, D-REPR and D2RML enable more sources. That is, using data in the same statement from, e.g., one CSV file and one JSON file.


\noindent\paragraph{\textbf{Linking Rules.}} Linking rules refer to linking resources that are being created in the mapping. For instance, having as object of a statement a resource that is the subject of another statement. These links are implemented in most languages by joining one or more data fields. Six languages do not allow these links: TARQL, CSVW, KR2RML and XLWrap. The rest is able to perform linking with at least one data reference and one or no condition. Fewer enable more data references and more conditions (e.g. in R2RML and most extensions allow the application of a \texttt{rr:joinCondition} over several fields). 

Linking rules using join conditions imply evaluating if the fields selected are equal. Since the join condition is the most common, applying the equal logical operator is the preferred choice. Only a few languages consider other similarity functions to perform link discovery, such as the Levenshtein distance and Jaro-Winkler, e.g., Helio. %Finally, it may be highlighted the ability for some languages to perform other logical operators appart from join, such as union (e.g. ShExML, KR2RML).




\noindent\paragraph{\textbf{Transformation functions.}} Applying functions in mappings allows practitioners transforming data before it is translated. For instance, to generate a label with an initial capital letter (\texttt{ex:ID001 rdfs:label "Emily"}) that was originally in lower case (```emily"), a function may be applied (e.g. GREL function \texttt{toTi\-tleCase()}). Only four of the analyzed languages do not allow the use of these functions: CSVW, R2RML and xR2RML. Of those that do, some use functions that belong to a specification (e.g. RML+FnO uses GREL functions\footnote{\url{https://docs.openrefine.org/manual/grelfunctions}}). All of them consider functions with cardinalities 1:1 and N:1; and half of them also include 1:N and N:M (i.e., output more than one value), for instance, a regular expression that matches and returns more than one value.  Nesting functions (i.e. calling a function inside another function) is not unusual; this is the case of SPARQL-based languages, the R2RML extensions that implement functions (except K2RML), Helio, D-REPR, and XLWrap. Finally, some languages even enable extending functions depending on specific user needs, such as XSPARQL, RML+FnO, SPARQL-Generate, Facade-X, R2RML-F, FunUL, XLWrap and D2RML.

%\noindent\paragraph{\textbf{Graph Construction Example.}} Assuming the description of data sources shown in \ref{fig:ex_json} and \ref{fig:ex_rdb} and the regular triples, this example shows how Helio and SPARQL-Generate describe conditional statements and linking rules. To generate the \texttt{eg:pop\-ulation} attribute (\ref{fig:ex_onto}), the record must have been updated after 2020. In addition, instances of the classes \texttt{eg:City} and \texttt{eg:Location} can be joined using the city name, present in both data sources. However, the names do not exactly match ("Almería" and "Almeria"; "A Coruña" and "La Coru\-ña"), which is why a distance metric is required to match the cities with a threshold of 0.75. The Helio mapping is not capable of describing the condition of the population, but instead it is able to use the Levenshtein distance function and link the sources (\ref{lst:helio_general}). SPARQL-Generate can describe the condition statement thanks to the SPARQL construct \texttt{FILTER}, but does not implement the distance metric function (\ref{lst:sg_general}). However, both Helio and SPARQL-Generate allow the removal of spaces in the subject URIs. 


\section{Mapping Rules for RDF: Conceptual Mapping}
\label{sec:chp4_cm_ontology}

Based on the analysis performed in \cref{sec:chp4_framework}, we abstract the features and limitations present in current mapping languages to represent them in an ontology. This ontology aims at gathering the expressiveness of current mapping languages, and is called the Conceptual Mapping\footnote{\label{foot:cmportal}\url{https://w3id.org/conceptual-mapping/portal}}. The methodology followed to build this ontology is first presented, and then each step of its construction is described in detail.

\subsection{Methodology}
The Conceptual Mapping ontology was developed following the guidelines provided by the Linked Open Terms (LOT) methodology. LOT is a well-known and mature lightweight methodology for the development of ontologies and vocabularies that has been widely adopted in academic and industrial projects~\citep{poveda2022lot}. It is based on the previous NeOn methodology~\citep{suarez2015neon} and includes four major stages: Requirements Specification, Implementation, Publication, and Maintenance~\cref{fig:chp4-2_lot}. In this section, we describe these stages and how they have been applied and adapted to the development of the Conceptual Mapping ontology.

\noindent\textbf{Requirements specification}
This stage refers to the activities carried out for defining the requirements that the ontology must meet. At the beginning of the requirements identification stage, the goal and scope of the ontology are defined. Following, the domain is analyzed in more detail by looking at the documentation, data that has been published, standards, formats, etc. In addition, use cases and user stories are identified. Then, the requirements are specified in the form of competency questions and statements. 

In this case, the specification of requirements includes purpose, scope, and requirements. The requirements are specified as facts rather than competency questions and validated with Themis~\citep{fernandez2021themis}, an ontology evaluation tool that allows validating requirements expressed as tests rather than SPARQL queries. We consider this approach to be adequate in this case since (1) there are no use cases as this ontology is a mechanism of representation of  mapping language's features; and (2) there are no SPARQL queries because they result from Competency Questions which are in turn extracted from use cases and user stories. Further details are shown in \cref{sec:chp4_requirements}.

\noindent\textbf{Implementation}
The goal of the Implementation stage is to build the ontology using a formal language, based on the ontological requirements identified in the previous stage. From the set of requirements a first version of the model is conceptualized. The model is subsequently refined by running the corresponding evaluations. Thus, the implementation process follows iterative sprints; once it passes all evaluations and meets the requirements, it is considered ready for publication.

The conceptualization is carried out representing the ontology in a graphical language using the  Chowlk notation~\citep{feria2022chowlk} (as shown in \cref{fig:chp4-2_cm_diagram}). The ontology is implemented in OWL 2 using Protégé. The evaluation checks different aspects of the ontology: (1)  requirements are validated using  Themis~\citep{fernandez2021themis}, (2)  inconsistencies are found with the Pellet reasoner, (3)  OOPS!~\citep{poveda2014oops} is used to identify modeling pitfalls, and (4) FOOPS!~\citep{garijo2021foops} is run to check the FAIRness of the ontology. Further details are described in \cref{sec:chp4_implementation}.

\noindent\textbf{Publication}
The publication stage addresses the tasks related to making the ontology and its documentation available. The ontology documentation was generated with Widoco~\citep{garijo2017widoco}, a built-in documentation generator in OnToology~\citep{alobaid2019automating}, and it is published with a W3ID URL\footnote{\label{foot:cmlink}\url{https://w3id.org/conceptual-mapping}}. The ontology and related resources can be accessed in the ontology portal. Further details are presented in \cref{sec:chp4_pub-main}.
%The ontology is published using a persistent URL \footnote{\url{https://w3id.org/conceptual-mapping}}.  and HTML documentation was generated with Widoco~\citep{garijo2017widoco}, using OnToology~\citep{alobaid2019automating}. As for maintenance, the ontology is available in a GitHub repository \footnote{https://github.com/oeg-upm/Conceptual-Mapping}.

\begin{figure}[!t]
\centering
\includegraphics[width=0.6\linewidth]{figures/chp4-2_lot.png}
\caption[LOT Methodology]{Workflow proposed by the LOT Methodology~\citep{poveda2022lot}.}
\label{fig:chp4-2_lot}
\end{figure}

\noindent\textbf{Maintenance}

Finally, the last stage of the development process, maintenance, refers to ontology updates as new requirements are found and/or errors are fixed. The ontology presented in this work promotes the gathering of issues or new requirements through the use of issues in the ontology GitHub repository. Additionally, it provides control of changes, and the documentation enables access to previous versions. Further details are shown in \cref{sec:chp4_pub-main}.



\subsection{Requirements}
\label{sec:chp4_requirements}
\ana{añadir requisitos en anexo?}
This section presents the purpose, scope, and requirements of the Conceptual Mapping Ontology. In addition, it also describes from where and how the requirements are extracted: analysing the mapping languages (presented as a comparative framework in \cref{sec:chp4_framework}) and the Mapping Challenges proposed by the community.

\subsubsection{Purpose and scope}

The Conceptual Mapping ontology aims at gathering the expressiveness of declarative mapping languages that describe the transformation of heterogeneous data sources into RDF. This ontology-based language settles on the assumption that all mapping languages used for the same basic purpose of describing data sources in terms of an ontology to create RDF, must share some basic patterns and inherent characteristics. Inevitably, not all features are common. As described in previous sections, some languages were developed for specific purposes, others extend existing languages to cover additional use cases, and others are in turn based in languages that already provide them with certain capabilities. The Conceptual Mapping ontology is designed to represent and articulate these core features, which are extracted from two sources: (1) the analysis of current mapping languages, and (2) the limitations of current languages identified by the community. These limitations, proposed by the W3C Knowledge Graph Construction Community Group\footnote{\label{foot:kgc}\url{https://www.w3.org/community/kg-construct/}}, are referred to as Mapping Challenges\footnote{\label{foot:challenges}\url{https://w3id.org/kg-construct/workshop/2021/challenges.html}} and have been partially implemented by some languages. %Both sources are described throughout this section.

This ontology also presents some limitations. As shown in \cref{sec:chp2_mappings}, mapping languages can be classified into three categories according to the schema in which they are based: RDF-based, SPARQL-based and based on other schemes. The Conceptual Mapping is included in the first category and, as such, has the same inherent capabilities and limitations as RDF-based languages regarding the representation of the language as an ontology. This implies that it is feasible to represent their expressiveness, whereas reusing classes and/or properties or creating equivalent constructs. Languages based on other approaches usually follow schemas that make them relatable to ontologies. This can be seen in the correspondence between YARRRML and RML: RML is written in Turtle syntax. YARRRML~\citep{Heyvaert2018yarrrml} is mainly used as a user-friendly syntax to facilitate the writing of RML rules. It is based on YAML, and can easily be translated into RML\footnote{\url{https://rml.io/yarrrml/matey/}}. 

Lastly, SPARQL-based languages pose a challenge. SPARQL is a rich and powerful query language~\citep{perez2009semantics} to which these mapping languages add more capabilities (e.g., SPARQL-Generate, SPARQL-Anything). It has an innate flexibility and capabilities sometimes not comparable to the other languages. For this reason, representing every single capability and feature of SPARQL-based languages is out of the scope of this work. Given the differences of representation paradigm between RDF and SPARQL for creating mappings, it cannot be ensured that the Conceptual Mapping covers all possibilities that a SPARQL-based language can, and as such, is considered a limitation.


\subsubsection{Mapping Challenges}
\label{sec:chp4_mapping_challenges}

Following its inception, the W3C Knowledge Graph Construction Community Group\cref{foot:kgc} defined a series of challenges for mapping languages based on the experience of members in using declarative mappings\cref{foot:challenges}. These challenges are a summary of the limitations of current languages. They have been partially addressed independently in some of the analyzed languages, such as RML~\citep{delva2021rml-fields,iglesias2023rml} and ShExML~\citep{garcia2021shexml-challenges}. These challenges are summarized as follows:

\begin{itemize}
    \item \textbf{[C1] Language Tags and Datatype.} It refers to dynamically building language tags [C1a] and datatypes [C1b], that is, from data rather than as constant values.
    \item \textbf{[C2] Iterators.} This challenge addresses the need to access data values 'outside' the iteration pattern [C2a], especially in \textcolor{black}{some tree-like data sources such as JSON}; and iterating over multi-value references [C2b].
    \item \textbf{[C3] Multi-value References.} It discusses how languages  handle data fields that contain multiple values [C3a], their datatypes and associated language tags [C3b].
    \item \textbf{[C4] RDF Collections and Containers.} This challenge addresses the need to handle RDF collections and containers.
    \item \textbf{[C5] Joins.} It refers to joining resources with zero join conditions [C5a] and joining literals instead of IRIs [C5b].
\end{itemize} 


\subsubsection{Conceptual Mapping Requirements}

In order to extract the requirements that serve as the basis for the development of the Conceptual Mapping ontology, we take as input the analysis from the comparison framework (\cref{sec:chp4_framework}) and the Mapping Challenges (\cref{sec:chp4_mapping_challenges}) described previously and the expertise of the authors. From a combination of their features, we extract 30 requirements. These requirements are expressed as facts, and are available in the ontology repository and portal\footnote{\url{https://oeg-upm.github.io/Conceptual-Mapping/requirements/requirements-core.html}}. Each requirement has a unique identifier, its provenance (comparison framework or mapping challenge id) and the corresponding constructs in the ontology. The constructs are written in Turtle, and lack cardinality restrictions for the sake of understandability. These requirements are tested with Themis, and its corresponding tests include these restrictions. More details on the evaluation of the requirements are provided in \cref{sec:eval}. 

The requirements gathered range from general-purpose to fine-grained details. The general-purpose requirements refer to the basic fundamental capabilities of mappings, e.g., to create the rules to generate RDF triples (cm-r8) from reference data sources (cm-r7). The requirements with the next level of detail involve some specific restrictions and functionalities, e.g. to indicate the specific type (whether they are IRIs, Blank nodes, etc.) of subjects (cm-r16), predicates (cm-r17), objects (cm-r18), named graphs (cm-r19), datatypes (cm-r20) and language tags (cm-r21); the possibility of using linking conditions (cm-r23) and functions (cm-r15). Finally, some requirements refer to specific details or features regarding the description of data sources (e.g. cm-r4, cm-r6) and transformation rules (e.g. cm-r14, cm-r22, cm-r25).

Not all the observed features in the comparison framework have been added to the set of requirements. Some features are really specific, and supported by a minority of languages, sometimes only one language. As a result, we selected the (really) detailed features in these requirements to build the core specification of the Conceptual Mapping when they tackled the basic functionalities of the language. The rest of the details are left to be included as extensions. This differentiation and the modeling criteria is explained further in \cref{sec:chp4_implementation}.










\subsection{Implementation}
\label{sec:chp4_implementation}

This section describes in detail the activities and tasks carried out to implement the ontology, that consists in the conceptualization of the model, the encoding in a formal language, and the evaluation to fix errors, inconsistencies, and ensure that it meets the requirements. Additionally, an example of the ontology's use is presented at the end of the section.



\subsubsection{Ontology Conceptualization}




\begin{sidewaysfigure*}[]
    \centering
    \includegraphics[width=1\linewidth]{figures/chp4-2_cm_diagram.pdf}
    \caption[Conceptual Mapping ontology overview]{Visual representation of the Conceptual Mapping ontology created using the Chowlk visual notation~\citep{feria2022chowlk}.}
    \label{fig:chp4-2_cm_diagram}
\end{sidewaysfigure*}

\textcolor{black}{The ontology's conceptualization is built upon the requirements extracted from experts experience, a thorough analysis of the features and capabilities of current mapping languages presented as a comparative framework; and the languages' limitations discussed by the community and denoted as Mapping Challenges. The resulting ontology model is depicted in \cref{fig:chp4-2_cm_diagram}. This model represents the core specification of the Conceptual Mapping ontology that contains the essential features to cover the requirements. Some detailed features are also included when considered important to the language expressiveness, or needed for the language main functionality. Other detailed features are considered as extensions, as explained further in this section \ana{ojo las extensiones}. For description purposes, we divide the ontology into two parts, \textit{Statements} and \textit{Data Sources}, that compose the core model. These two parts, when not used in combination, cannot describe a complete mapping. For that reason they are not separated into single modules. } 

\noindent\paragraph{\textbf{Data sources.}} A data source (\texttt{DataSource}) describes the source data that will be translated. For this section, the Data Catalog (DCAT) vocabulary~\citep{albertoni2020dcat2} has been reused. \texttt{DataSource} is a subclass of \texttt{dcat:Distribution}, which is a specific representation of a dataset (\texttt{dcat:Dataset}), defined as ``data encoded in a certain structure such as lists, tables and databases''. A source can be a streaming source (\texttt{StreamSource}) that continuously generates data, a synchronous source (\texttt{SynchronousSource}) or an asynchronous source (\texttt{AsynchronousSource}). Asynchronous sources, in turn, can be event sources (\texttt{EventSource}) or periodic sources (\texttt{Periodic Source}). The details of the data source access are represented with the data access service class (\texttt{Data AccessService}), which in turn is a subclass of \texttt{dcat:DataService}. This class represents a collection of operations that provides access to one or more datasets or data processing functions, i.e., a description of how the data is accessed and retrieved. The data access service optionally has a security scheme (e.g., OAuth2, API Key, etc.) and an access protocol (e.g., HTTP(s), FTP, etc.).

Data properties in the \texttt{dcat:Dataset}, \texttt{dcat:Distribution} and \texttt{dcat:DataService} classes may be reused according to the features that may be represented in each mapping language, e.g. \texttt{dcat:endpointURL},  \texttt{dcat:accessURL} and \texttt{dcat:endpointDescrip\-tion}. A data access service is related to a security scheme. The class \texttt{wot:Securi\-tyScheme} (from the Web of Things (WoT) Security ontology\footnote{\label{foot:wotsec}\url{https://www.w3.org/2019/wot/security}}) has been reused. This class has different types of security schemes as subclasses and includes properties to specify the information on the scheme (e.g. the encryption algorithm, the format of the authentication information, the location of the authentication information). The security protocol \texttt{hasProtocol} has as set of predefined values that have been organized as a SKOS concept scheme. It contains almost 200 security protocols, e.g., HTTP(s), JDBC, FTP, GEO, among others. This SKOS list can be extended according to the users' needs by adding new concepts. 

In order to represent the fragments of data that are referenced in a statement map, the class \texttt{Frame} has been defined. They are connected with the property \texttt{hasFrame}. A frame can be a \texttt{SourceFrame} (base case) or a \texttt{CombinedFrame}, the latter representing two source frames or combined frames that are combined by means of a join (\texttt{JoinCombination}), a union (\texttt{UnionCombination}) or a cartessian product (\texttt{Cartessi\-anProductCombination}). 

A source frame corresponds to a data source (with \texttt{hasDataSource}) and defines which data is retrieved from the source and how it is fragmented (with \texttt{expression}). Among others, JSONPaths, XPaths, queries, or regular expressions can be expressed with this feature. \textcolor{black}{The language of the expression is defined with \texttt{language}, which domain is the reused class from RML \texttt{rml:ReferenceFormulation}}. A source frame may be related to another source frame with  \texttt{hasNest\-edFrame}, e.g. a frame is accessed firstly with a SPARQL query, and their results as a CSV file with this property. A source fragment may refer to many data fields (with \texttt{hasField}, which is the inverse property of \texttt{belongsToFrame}).

The desired output of the statements that are to be generated can be described with the imported \texttt{rml:LogicalTarget}, referenced by the optional property \texttt{hasLogicalTarget}. A \textit{Logical Target} specifies which RDF serialisation the output should be encoded (\texttt{rml:serialization}), and the referred \texttt{rml:Target}. 
The \texttt{rml:Target} indicates how the output target can be accessed,
and can describe: (i) the compression format for the RDF output (\texttt{rml:compression}) and, if so, how e.g., GZip,
and (ii) the encoding (\texttt{rml:encoding}), e.g., UTF-8.

\noindent\paragraph{\textbf{Statements.}} The central class of this section is the \texttt{StatementMap}, which represents a rule that defines for a triple its subject (\texttt{hasSubject}), predicate (\texttt{hasPredicate}), and object (\texttt{hasObject}). Optionally, it can also specify the object datatype (\texttt{hasDatatype}), language (\texttt{hasLanguage}) and assigned named graph (\texttt{hasNamedGraph}). Therefore, statement maps are similar to RDF statements as both of them are comprised by a subject, predicate and object. In statement maps, objects are resources (\texttt{ResourceMap}), and subjects and predicates are more specific, certain subclasses of the resource map: predicates are reference node maps (\texttt{ReferenceNodeMap}) that represent resources with an IRI, i.e., ontology properties. Subjects are node maps (\texttt{NodeMap}) that may be blank nodes (\texttt{Blank Node}) or also reference node maps. An object may be a literal (\texttt{LiteralMap}), a blank node, a container (\texttt{ContainerMap}) or a collection that defines a list (\texttt{ListMap}). The language is expressed as a literal, and the datatype is also a resource with an IRI, i.e. a reference node map.

Resource maps are expressed with an evaluable expression (\texttt{EvaluableExpression}) that may be a constant value (\texttt{Constant}), a function expression (\texttt{FunctionExpression}), or a data field (\texttt{DataField}) that belongs to some data source fragment (\texttt{belongsToFrame}). For function expressions, the function name (\texttt{hasFuntionName}) is taken from a set of predefined names organized in a SKOS concept scheme. This SKOS list can be extended according to the users' needs by adding new concepts for functions that have not been defined. Recursion in this function expression is represented through its input (\texttt{hasInput}) as an expression list (\texttt{ExpressionList}). Expression lists have been represented as a subclass of RDF lists (\texttt{rdf:List}), and the properties (\texttt{rdf:first}) and (\texttt{rdf:rest}) have been reused. Expression lists may have nested expression lists inside.


A special case of a statement map is a conditional statement map (\texttt{ConditionalSta\-tementMap}), a statement map that must satisfy a condition for the triples to be generated. The condition (\texttt{hasBooleanCondition}) is a function expression (e.g. if a value from a field called ``present'' is set to ``False'', the statement is not generated). Another relevant class is the linking map (\texttt{LinkingMap}), that enables linking subjects from a source (\texttt{source}) and a target (\texttt{target}) statement maps, i.e., two resources are linked and triples are generated if a linking condition is satisfied. Similarly to the conditional statement map, this condition is represented as a function expression.

\subsubsection{Ontology Design Patterns}
The following ontology design patterns \textcolor{black}{have been applied in the conceptualization as they are common solutions to the problem of representing taxonomies and linked lists}:
\begin{itemize}
    \item The SKOS vocabulary has been reused to represent some coding schemes such as the protocol taxonomy and the function taxonomy. \textcolor{black}{The design pattern consists on having an instance of} \texttt{skos:ConceptScheme} for each taxonomy, then each concept or term in the taxonomy, \texttt{skos:Concept}, is related to the corresponding concept scheme through the property \texttt{skos:inScheme}. The class that uses the taxonomy is then related to \texttt{skos:Concept} through an object property, e.g., class \texttt{DataAccessSer\-vice} and object property \texttt{hasProtocol}.
    \item The class \texttt{ExpressionList} uses the \textcolor{black}{design} pattern for lists developed in RDF where the properties \texttt{rdf:first} and \texttt{rdf:rest} are used to represent a linked list. The base case (first) is an evaluable expression whereas the rest of the list is (recursively)  an \texttt{ExpressionList}.
\end{itemize}

\subsubsection{Ontology evaluation}\label{sec:eval}


The ontology, implemented in OWL with Protégé, has been evaluated in different ways to ensure that it is correctly implemented, has no errors, inconsistencies or pitfalls, and meets the requirements.

\noindent\paragraph{\textbf{Reasoner.}} We used the reasoner Pellet in Protégé to look for inconsistencies in the model, and the results showed no errors.

\noindent\paragraph{\textbf{OOPS!.}}~\citep{poveda2014oops} This tool was used to identify modeling pitfalls in the ontology. We executed the tool several times to fix the pitfalls, until there were no important ones. Currently, the results of OOPS! show pitfalls from the reused ontologies, but none important for the newly created terms and axioms. One minor pitfall is returned, P13, regarding the lack of inverse relationships, which we consider that are not  needed in the ontology. The rest of the pitfalls are as follows: P08 (missing annotations) from DCTERMS; P11 (missing domain or range in properties) for DCTERMS, DCAT and SKOS; and P20 (misusing ontology annotations) for DCAT.

\noindent\paragraph{\textbf{Themis.}}~\citep{fernandez2021themis} Themis is a tool able to evaluate whether the requirements are implemented in the ontology. To that end, the requirements must be provided in a specific syntax or described with the Verification Test Case (VTC) ontology\footnote{\url{https://albaizq.github.io/test-verification-ontology/OnToology/ontology/verification-test-description.ttl/documentation/index-en.html}}. The requirements of the Conceptual Mapping were translated to create the corresponding tests, and were tested in the tool with success. The requirements and associated test along with the complete set of tests annotated with the VTC ontology are available in the GitHub repository\footnote{\url{https://github.com/oeg-upm/Conceptual-Mapping/tree/main/requirements}}.

\noindent\paragraph{\textbf{FOOPS!.}}~\citep{garijo2021foops} Additionally, we tried running FOOPS! to check the FAIRness of the ontology, resulting in 73\%, which is acceptable. To improve the score, the ontology should be added to a registry and have more metadata describing it, and use a persistent base IRI. 

With these evaluations, we can conclude that the ontology is correctly encoded and implemented, and that it meets the requirements specified in \cref{sec:chp4_requirements}. 



\subsection{Publication and Maintenance}
\label{sec:chp4_pub-main}

In order to publish the ontology, the first step required is to create the ontology documentation. We used Widoco~\citep{garijo2017widoco}, integrated inside the OnToology~\citep{alobaid2019automating} system, to automatically generate and update the HTML documentation every time there is a commit in the GitHub repository where the ontology is stored. This documentation contains the ontology metadata, links to the previous version, a description of the ontology, the diagram, and detailed examples of the capabilities of the language. It is published using a W3ID URL\cref{foot:cmlink} and under the CC BY-SA 4.0 license.

The HTML documentation is not the only documentation resource provided. An overview of all resources is provided in the ontology portal\cref{foot:cmportal}. This portal shows in a table the ontologies associated with the Conceptual Mapping ontology. For now, the core (Conceptual Mapping) and an extension to describe CSV files in detail (Conceptual Mapping - CSV Description) are available. For each ontology, links to the HTML documentation, the requirements, the GitHub repository, the Issue Tracker, and the releases are provided. 

The maintenance is supported by the Issue Tracker\footnote{\url{https://github.com/oeg-upm/Conceptual-Mapping/issues}}, where proposals for new requirements, additions, deletions or modifications can be added as GitHub issues. This approach allows authors to review the proposals and discuss their possible implementation.



\subsection{Extensions}

The Conceptual Mapping ontology has been designed as a core ontology. However, as time passes, new requirements may emerge. In order to include these new requirements, new modules of the Conceptual Mapping ontology shall be developed. It is worth mentioning that this is a common practice for ontologies, which is highly suitable for adapting an existing ontology to new scenarios, by ontology modules specialized for a specific set of requirements. A clear example of this is the SAREF ontology\footnote{\url{https://saref.etsi.org/}}, that has a core module\footnote{\url{https://saref.etsi.org/core/v3.1.1/}} and then specific extensions\footnote{\url{https://saref.etsi.org/extensions.html}} for certain domains, such as energy (SAREF4ENER) or buildings (SAREF4BLDG) among others. For the Conceptual Mapping we present two extensions: for allowing a more detailed description of CSV files, and for generating RDF-star graphs.


\begin{figure}[!t]
\centering
\includegraphics[width=0.8\linewidth]{figures/chp4-2_cm-csv.pdf}
\caption[CM-CSV module]{Conceptualization of the CM-CSV module following the Chowlk visual notation~\citep{feria2022chowlk}.}
\label{fig:chp4-2_cm-csv}
\end{figure}


\subsubsection{CSV description: CM-CSV}
The core module of the Conceptual Mapping includes limited possibilities for describing CSV files in depth. The extension CM-CSV\footnote{\url{https://w3id.org/conceptual-mapping/csv}} provides extended features to this end. Thus, the CSVW~\citep{Tennison2015csvw} proposal has been blended as an ontology module linked to the core Conceptual Mapping ontology. The class \texttt{CSVSourceFrame} is created as a subclass of \texttt{SourceFrame} and \texttt{csvw:Dialect} to inherit their characteristics. This module is depicted in \cref{fig:chp4-2_cm-csv}. Thus, this extension allows to describe CSV characteristics such as if the file contains headers (\texttt{csvw:header}), its delimiter (\texttt{csvw:delimiter}), separator (\texttt{csvw:separator}), among many others. 

\subsubsection{RDF-star generation: CM-star}
RDF-star~\citep{hartig2017foundations} was recently proposed as a compact alternative for reification in RDF. It extends the RDF syntax introducing the notion of \textit{Quoted triples}, i.e. triples that can be placed in the position of objects and/or subjects. These triples can be inserted in the graph outside the quoted triple (i.e. they are \textit{asserted}), or appear only inside another triples (i.e. they are \textit{non-asserted}).

RDF-star has quickly gained popularity, leading
to its adoption by a wide range of systems~\footnote{\url{https://w3c.github.io/rdf-star/implementations.html}} (e.g., Apache Jena~\footnote{\url{https://jena.apache.org}}, Oxigraph~\citep{oxigraph}) and the formation of the RDF-star Working Group\footnote{\url{https://www.w3.org/groups/wg/rdf-star}}. \ana{ver si es too much después del SOTA}


\begin{figure}[!t]
\centering
\includegraphics[width=1\linewidth]{figures/chp4-2_cm-star.pdf}
\caption[CM-star module]{Conceptualization of the CM-star module following the Chowlk visual notation~\citep{feria2022chowlk}.}
\label{fig:chp4-2_cm-star}
\end{figure}


Regarding mapping languages, SPARQL-Anything is able to generate RDF-graphs as it natively implements the updates from Apache Jena; RML was extended for this purpose with the RML-star module~\citep{iglesias2022rmlstar,delva2021rml-star,iglesias2023rml} (further details in \cref{sec:chp4_rml_star}), as well as R2RML-star~\citep{sundqvist2022extending}.

The Conceptual Mapping implements the RDF-star graph generation adopting the recursive nature of quoted triples in the CM-star module\footnote{\url{https://w3id.org/conceptual-mapping/star}}. \cref{fig:chp4-2_cm-star} represents the overview of this module. Two subclasses of \texttt{StatementMap} are created to denote asserted (\texttt{AssertedStatementMap}) and non-asserted (\texttt{NonAssertedStatementMap}) statement maps. These statements can be quoted as subjects and objects with the properties \texttt{hasQuotedSubject} and \texttt{hasQuotedObject} respectively. The range of these properties is the union of the introduced statement subclasses. This extension enables potentially an infinite number of quoted statements, as the RDF-star specification indicates~\cite{hartig2023rdf}. 



\subsection{Ontology usage example}\label{sec:cm_example} 



\begin{figure}[t!]
    \centering
    \begin{subfigure}[b]{0.45\linewidth}
        \centering
    	\includegraphics[width=1\linewidth]{figures/chp4-2_example_ont}
    	\caption{Example reference ontology that represents the classes \texttt{City} and \texttt{Location}, linked by the property \texttt{eg:location}.}
    	\label{fig:chp4-2_chp4_ex_onto}
    \end{subfigure}
    \begin{subfigure}[b]{0.28\linewidth}
        \centering
    	\includegraphics[width=1\linewidth]{figures/chp4-2_example_json}
    	\caption{Example input JSON file ```coordinates.json".}
    	\label{fig:chp4-2_chp4_ex_json}
    \end{subfigure}
    \begin{subfigure}[b]{0.7\linewidth}
        \centering
    	\includegraphics[width=1\linewidth]{figures/chp4-2_example_rdb}
    	\caption{Example input MySQL table ```cities".}
    	\label{fig:chp4-2_chp4_ex_rdb}
    \end{subfigure}
    \caption[Example data and ontology about cities for CM mapping]{Input source data and reference ontology that represents information on cities and their location.}
    \label{fig:chp4-2_chp4_ex_input}
\end{figure}



%\ana{no sé si dejar esto o rehacer y poner de drugs4covid, para que no sean datos de juguete. O si acaso poner que sea un use case, eso requeriría de hacer el mapping también en otros lenguajes. Por ahora dejar como está}

This section builds a mapping in three steps (data sources in \cref{lst:chp4_cm_sources}, triples in \cref{lst:chp4_cm_spo} and special statements in \cref{lst:chp4_cm_general}) to represent how the proposed language can describe data with different features. The mapping uses the data sources ``coordinates.json" (\cref{fig:chp4-2_chp4_ex_json}) and ``cities" (\cref{fig:chp4-2_chp4_ex_rdb}) as input and the ontology depicted in  \cref{fig:chp4-2_chp4_ex_onto} as reference, to create the output RDF shown in \cref{lst:chp4_output}. 
%Additionally, \cref{appendix2}\ana{APENDIX!!} contains a second example to illustrate different features than the ones represented in the example of this section, to provide more insights about the expressiveness of this language.

\begin{captionedlisting}{lst:chp4_output}{Expected RDF output for the data sources and the ontology in \cref{fig:chp4-2_chp4_ex_input}.}
\centering
{\begin{lstlisting}[]
<http://ex.com/loc/40.4189--3.6919> a eg:Location ;
	eg:lat "40.4189"^^xsd:decimal ;
	eg:long "-3.6919"^^xsd:decimal .
<http://ex.com/loc/43.3713--8.4188> a eg:Location ;
	eg:lat "43.3713"^^xsd:decimal ;
	eg:long "-8.4188"^^xsd:decimal .
<http://ex.com/loc/36.8333--2.45> a eg:Location ;
	eg:lat "36.8333"^^xsd:decimal ;
	eg:long "-2.45"^^xsd:decimal .
<http://ex.com/city/ACoruña> a eg:City ;
	eg:zipcode 15001, 15002, 15003, 15004 ;
	eg:location <http://ex.com/loc/43.3713--8.4188> .
<http://ex.com/city/Almería> a eg:City ;
	eg:zipcode 04001, 04002 ;
	eg:population 201322 ;
	eg:location <http://ex.com/loc/36.8333--2.45> .
<http://ex.com/city/Madrid> a eg:City ;
	eg:zipcode 28001, 28002, 28003, 28004, 28005, 28006;
	eg:population 3334730 ;
	eg:location <http://ex.com/loc/40.4189--3.6919> .
\end{lstlisting}}
\end{captionedlisting}

\noindent\paragraph{\textbf{Data sources.}} \cref{lst:chp4_cm_sources} shows the description of the json file ``coordinates.json" indicating the protocol from the SKOS concept scheme (\texttt{cmp:https}), media type (``application/json"), JSONPath to extract data, access URL  ``https://ex.com/geodata/coordi\-nates.json", and  fields that are going to be used in the transformation. There is no security scheme. The MySQL table ``cities" also has no security scheme, the protocol needed is \texttt{cmp:jdbc}, the database access is specified in the endpoint URL, and the table as an SQL query. The fields are also specified, with the special case of ``zipcodes" that needs a \texttt{cm:hasNestedFrame} to extract multiple values inside the field.

\begin{captionedlisting}{lst:chp4_cm_sources}{Description with the Conceptual Mapping of two data sources (a JSON file and a relational database), their access and fields.}
\centering
{\begin{lstlisting}[language=concm,firstnumber=1]
# Locations
:FrameLoc a cm:SourceFrame;
  cm:expression "$\dollar$.coordinates[*]";
  cm:language ql:JSONPath ;
  cm:hasField :lat;
  cm:hasField :long;
  cm:hasField :loc_city;
  cm:hasDataSource [ a cm:SynchronousSource;
    dcat:mediaType "text/json";
    dcat:accessService [
      cm:hasProtocol cmp:https;
      dcat:endpointURL "https://ex.com/geodata/coordinates.json" 
      cm:hasSecurityScheme [ a wotsec:NoSecurityScheme; ];
    ] ;
  ] .

:lat a cm:DataField ; cm:field "$\dollar$.latitude" .
:long a cm:DataField ; cm:field "$\dollar$.longitude" .
:loc_city a cm:DataField; cm:field "$\dollar$.city" .

# Cities
:FrameCities a cm:SourceFrame ;
  cm:expression "SELECT * FROM cities;";
  cm:hasField :c_city;
  cm:hasField :population;
  cm:hasField :year;
  cm:hasNestedFrame [
    cm:expression "$\dollar$.zipcodes[*]";
    cm:hasField :zipcode ];
  cm:hasDataSource [ a cm:SynchronousSource;
    dcat:mediaType "text/plain";
    dcat:accessService [
      cm:hasProtocol cmp:jdbc;
      dcat:endpointURL "jdbc:mysql://localhost:3306/citydb";
      cm:hasSecurityScheme [a wotsec:NoSecurityScheme;] ].

:c_city a cm:DataField; cm:field "city" .
:population a cm:DataField; cm:field "population" .
:year a cm:DataField; cm:field "year_modified" .
:zipcode a cm:DataField cm:field "zipcodes" .
\end{lstlisting}}
\end{captionedlisting}

%\noindent{\textbf{Locations.}} Sarah is working at the National Geographic Institute from Spain, and manages everyday information about places and locations. She wants to create a knowledge graph with the geographical data that she usually manages. She starts writing a mapping (\cref{lst:uc1}) to represent cities from a local JSON file, \texttt{Venue.json} (\cref{lst:data_venues}). This data will be transformed into instances of the class \texttt{City}. 

\noindent\paragraph{\textbf{Statements.}} \cref{lst:chp4_cm_spo} contains the rules needed to create instances of the classes \texttt{eg:Location} and \texttt{eg:City}; and their following attributes: \texttt{eg:lat} and \texttt{eg:long} for the former; \texttt{eg:zipcode} for the latter. To correctly generate the URI for the instances of \texttt{eg:City}, a replace function inside a concatenate function is needed to (1) remove the blank spaces in the field ``city" and (2) add the field to the base URI ``http://ex.com/city/".

\begin{captionedlisting}{lst:chp4_cm_spo}{Description with the Conceptual Mapping of the creation of regular statements from the data sources described in \cref{lst:chp4_cm_sources}.}
\centering
{\begin{lstlisting}[language=concm,firstnumber=1]
# Locations
:SubjectLoc a cm:ReferenceNodeMap ;
  cm:hasEvaluableExpression [
    cm:hasFunctionName cmf:concat; 
    cm:hasInput ([cm:constantValue "http://ex.com/loc/"]    :lat [cm:constantValue "-" ] :long)].

:StatementLoc1 a cm:StatementMap ;
  cm:hasFrame :FrameLoc ;
  cm:subject :SubjectLoc ;
  cm:predicate [ a cm:ReferenceNodeMap; 
    cm:hasEvaluableExpression [cm:constantValue rdf:type ] ];
  cm:object [cm:hasEvaluableExpression [cm:constantValue eg:Location]].

:StatementLoc2 a cm:StatementMap ;
  cm:hasFrame :FrameLoc ;
  cm:subject :SubjectLoc ;
  cm:predicate [ a cm:ReferenceNodeMap; 
    cm:hasEvaluableExpression [cm:constantValue eg:lat]];
  cm:object [ a cm:Literal; cm:hasEvaluableExpression :lat];
  cm:hasDatatype [cm:hasEvaluableExpression xsd:decimal].

:StatementLoc3 a cm:StatementMap ;
  cm:hasFrame :FrameLoc ;
  cm:subject :SubjectLoc ;
  cm:predicate [ a cm:ReferenceNodeMap; 
    cm:hasEvaluableExpression [cm:constantValue eg:long]];
  cm:object [ a cm:Literal; cm:hasEvaluableExpression :long];
  cm:hasDatatype [ cm:hasEvaluableExpression xsd:decimal].
    
# Cities
:city_ns a cm:FunctionExpression ;
  cm:functionName cmf:replace ;
  cm:hasInput (c_city " " "")

:SubjectCities a cm:ReferenceNodeMap;
  cm:hasEvaluableExpression [
    cm:hasFunctionName cmf:concat; 
    cm:hasInput ([cm:constantValue "http://ex.com/city/"] :city_ns)].

:StatementCit1 a cm:StatementMap ;
  cm:hasFrame :FrameCities ;
  cm:subject :SubjectCities ;
  cm:predicate [ a cm:ReferenceNodeMap; 
    cm:hasEvaluableExpression [cm:constantValue rdf:type]];
  cm:object [ a cm:ReferenceNodeMap; 
    cm:hasEvaluableExpression  [cm:constantValue eg:City]] .

:StatementCit2 a cm:StatementMap ;
  cm:hasFrame :FrameCities ;
  cm:subject :SubjectCities ;
  cm:predicate [ a cm:ReferenceNodeMap; 
    cm:hasEvaluableExpression [cm:constantValue rdfs:label]];
  cm:object [ a cm:ReferenceNodeMap; 
    cm:hasEvaluableExpression  [cm:constantValue :c_city]] .
  cm:hasLanguage [ cm:hasEvaluableExpression [ cm:constantValue "es" ] ].

:StatementCit3 a cm:StatementMap ;
  cm:hasFrame :FrameCities ;
  cm:subject :SubjectCities ;
  cm:predicate [ a cm:ReferenceNodeMap; 
    cm:hasEvaluableExpression [cm:constantValue eg:zipcode] ];
  cm:object [ a cm:Literal;
    cm:hasEvaluableExpression [cm:constantValue :zipcode] ];
  cm:hasDatatype [ cm:hasEvaluableExpression xsd:integer ].
\end{lstlisting}}
\end{captionedlisting}


\noindent\paragraph{\textbf{Special statements.}} \cref{lst:chp4_cm_general} describes how a conditional statement and a linking rule are generated. This description is represented by means of functions. With the property \texttt{cm:hasBooleanCondition}, the conditional statement declares that the field \texttt{:year} has to be greater than 2020. The linking rule performs the link between the instances of \texttt{eg:City} and \texttt{eg:Location} with the predicate \texttt{eg:location}, using a distance metric (levenshtein function) that has to be greater then a threshold of ``0.75". 

\begin{captionedlisting}{lst:chp4_cm_general}{Conditional and linking rules described with the Conceptual Mapping that complement the data source description and regular statements described in \cref{lst:chp4_cm_sources} and \cref{lst:chp4_cm_spo}.}
\centering
{\begin{lstlisting}[language=concm,firstnumber=1]
:StatementCit4 a cm:ConditionalStatementMap ;
  cm:hasFrame :FrameCities ;
  cm:subject :SubjectCities ;
  cm:predicate [ a cm:ReferenceNodeMap; 
    cm:hasEvaluableExpression [cm:constantValue eg:population] ];
  cm:object [ a cm:Literal; 
    cm:hasEvaluableExpression  [cm:constantValue :population] ];
  cm:hasDatatype [ cm:hasEvaluableExpression xsd:integer];
  cm:hasBooleanCondition [
    cm:functionName cmf:greater_than ;
    cm:hasInput ( :year 2020 ) ] .

:LinkExp1 a cm:LinkingExpression ;
  cm:source :StatementCit1 ;
  cm:target :StatementLoc1 ;
  cm:property eg:location ;
  cm:hasBooleanCondition [
    cm:functionName cmf:greater_than ; 
    cm:hasInput ( :levfun 0.75 ) ] .

:levfun a cm:FunctionExpression ;
  cm:functionName cmf:levenshtein_distance ;    
  cm:hasInput (:c_city :loc_city) .
\end{lstlisting}}
\end{captionedlisting}









\section{Mapping Rules for RDF-star: RML-star}
\label{sec:chp4_rml_star}

%In the late years, the RDF-star~(\cite{hartig2017foundations}) approach for reification has gained popularity. Along with its expansions, different semantic web technologies have been updated to include this new syntax\footnote{\url{https://w3c.github.io/rdf-star/implementations}}. This section describes one of these updates related to building RDF-star graphs: RML-star. \ana{esto no liga del todo bien con lo anterior}



Making statements about statements in RDF has
posed a challenge almost since the inception of RDF.
The W3C RDF Primer~(\cite{manola2004rdf}) already included a description of the standard reification approach to tackle this issue.
Other alternatives were proposed over the years,
such as named graphs~(\cite{carroll2005namedgraphs}), N-ary relationships~(\cite{naryw3c2006}), singleton properties~(\cite{nguyen2014don}), RDF$^+$~(\cite{schueler2008querying}), and more recently, \mbox{RDF-star}~(\cite{hartig2017foundations}). 

RDF-star does not leverage the characteristics of RDF as the others proposals. Instead it extends the RDF syntax to introduce the notion of \texttt{embedded triples}. This approach is currently under development to become a W3C standard\footnote{\url{https://www.w3.org/groups/wg/rdf-star}}. 
As a consequence, an increasing number of semantic technologies are supporting this proposal\footnote{\url{https://w3c.github.io/rdf-star/implementations}}, which include triplestores (e.g. GraphDB, AnzoGraph) and programming libraries (e.g. Apache Jena, Oxigraph).

It is possible to construct reified RDF graphs using mappings with the reification approaches abovementioned, except for RDF-star. As this proposal extends the original RDF syntax, it requires mapping languages to adapt and evolve in order to construct RDF-star graphs. This chapter presents the extension of the RML language to enable the construction of RDF-star graphs, RML-star. We first introduce how reified RDF graphs can be constructed using RML with different reification approaches, and then we present RML-star. This proposal is validated with \mbox{Morph-KGC$^{star}$}, a KG construction engine that implements our proposal.

%This section describes popular reification approaches and shows how they can be used in RML and RML-star with a running example. 
%Standard reification and singleton properties are considered in Section \ref{sec:chp4_validation}, showing that \mbox{Morph-KGC$^{star}$} does not add any overhead in the time required to generate the \mbox{RDF-star} triples compared to them.

\subsection{Statements about statements in mapping rules}
\label{sec:chp4_reif_mappings}

We illustrate each reification alternative presented throughout this section with a running example that uses the data shown in \cref{lst:chp4_csv_star}.
It contains CSV data related to pole vault:
the vaulter (\texttt{PERSON}),
the height of the jump (\texttt{MARK}),
the date when the jump was performed (\texttt{DATE}) and
an identifier of the jump (\texttt{ID}).
The running example represents
that a person jumped some height on a specific date, i.e., it adds the metadata about ``date''
to the statement ``a person jumped some height''.

\noindent\hspace{0.15\linewidth}\begin{minipage}{\linewidth}
\begin{captionedlisting}{lst:chp4_csv_star}{Contents of the logical source \texttt{:marks} in CSV format.}
\centering
\begin{tabular}{c}
\hspace{3em}
{\begin{lstlisting}[basicstyle=\ttfamily\small,label={list:example1},columns=flexible]
ID  , DATE       , MARK ,   PERSON
1   , 2022-03-21 , 4.80 ,   Angelica
2   , 2022-03-19 , 4.85 ,   Katerina
\end{lstlisting}}
\end{tabular}
\end{captionedlisting}
\end{minipage}

\subsubsection{Reification with RML}

There are several reification approaches for RDF as presented in \cref{chp2_reification}, such as standard reification, singleton properties, N-ary relationships or named graphs. 
These approaches use strategies that add metadata to triples
without modifying the original RDF syntax.
Thus, they can be used with RML without any further modification. RML mapping rules enable the generation of blank nodes (required for the standard reification approach), dynamically generated predicates (required for the singleton properties approach) and named graphs (required for the named graphs approach). %\ana{si voy con todos, influye en el ejemplo de aquí abajo y en los use cases. En ese caso, quitar somef (demasiado grande el mapping, con semmeddb es suficiente)}

%\ana{probablemente en sota estarán explicados cada uno de los modelos, probablemente no haga falta ponerlos aquí también.}


\noindent\textbf{\textit{Standard Reification}}~(\cite{manola2004rdf}) was proposed in the W3C RDF Primer~(\cite{manola2004rdf}).
It assigns statements to unique identifiers (typically blank nodes) typed with \texttt{rdf:Statement} and described using the properties \texttt{rdf:subject}, \texttt{rdf:predicate} and \texttt{rdf:object}.
This way, the unique identifier representing the statement can be further annotated with additional statements. \cref{lst:chp4_res-std-reif} shows an example of standard reification for the data in \cref{lst:chp4_csv_star}, created with the RML mapping rules in \cref{lst:chp4_std-reif}. 
This mapping creates blank nodes in the subject with the \texttt{ID} data field, typed with \texttt{rdf:Statement}; and has three predicate object maps to generate the \texttt{rdf:subject}, \texttt{rdf:predicate}, \texttt{rdf:object} of the triples (\textit{An athlete jumps certain height}) and a predicate object map to annotate the statements with \texttt{:date} (\textit{in a specific date}).




\begin{minipage}{\linewidth}
\begin{captionedlisting}{lst:chp4_std-reif}{Example RML mapping using standard reification that transforms data in \cref{lst:chp4_csv_star}.}
\centering
\begin{multicols}{2}
{\begin{lstlisting}[basicstyle=\ttfamily\small,label={list:example1},columns=flexible]
<#TM> a rr:TriplesMap ;
  rml:logicalSource :marks ;
  rr:subjectMap [ 
    rml:reference "ID" ;
    rr:termType rr:BlankNode ;
    rr:class rdf:Statement  ] ;
  rr:predicateObjectMap [ 
    rr:predicate rdf:subject ;
    rr:objectMap [
      rr:template ":{PERSON}" ] ] ;
  rr:predicateObjectMap [ 
    rr:predicate rdf:predicate ;
    rr:object :jumps ] ;
  rr:predicateObjectMap [ 
    rr:predicate rdf:object ;
    rr:objectMap [
      rml:reference "MARK" ] ] ;
  rr:predicateObjectMap [ 
    rr:predicate :date ;
    rr:objectMap [
      rml:reference "DATE" ] ] .
\end{lstlisting}}
\end{multicols}
\end{captionedlisting}
\end{minipage}

\begin{minipage}{\linewidth}
\begin{captionedlisting}{lst:chp4_res-std-reif}{RDF triples generated by the mapping in \cref{lst:chp4_std-reif}.}
\centering
\begin{multicols}{2}
{\begin{lstlisting}[basicstyle=\ttfamily\small,label={list:example1},columns=flexible]
_:1 rdf:type      rdf:Statement .
_:1 rdf:subject   :Angelica .
_:1 rdf:predicate :jumps .
_:1 rdf:object    "4.80" .
_:1 :date         "2022-03-21" .
_:2 rdf:type      rdf:Statement .
_:2 rdf:subject   :Katerina .
_:2 rdf:predicate :jumps .
_:2 rdf:object    "4.85" .
_:2 :date         "2022-03-19" .
\end{lstlisting}}
\end{multicols}
\end{captionedlisting}
\end{minipage}



\noindent\textbf{\textit{Singleton Properties}}~(\cite{nguyen2014don}). This approach uses unique predicates linked with \texttt{rdf:singletonPropertyOf} to the original predicate. 
This unique predicate can then be annotated as the subject of additional statements. 
\cref{lst:chp4_res-sp-reif} shows the reified triples for the data in \cref{lst:chp4_csv_star} created with the RML mapping rules in \cref{lst:chp4_sp-reif}. 
It uses a singleton property dynamically generated with the \texttt{ID} data field for the property \texttt{:jumps} (\textit{An athlete jumps certain height}), annotated with \texttt{:date} (\textit{in a specific date}).


\begin{minipage}{\linewidth}
\begin{captionedlisting}{lst:chp4_sp-reif}{Example RML mapping using a singleton property that transforms data in \cref{lst:chp4_csv_star}.}
\centering
\begin{multicols}{2}
{\begin{lstlisting}[basicstyle=\ttfamily\small,label={list:example1},columns=flexible]
<#TM> a rr:TriplesMap ;
  rml:logicalSource :marks ;
  rr:subjectMap [ 
    rr:template ":{PERSON}" ] ;
  rr:predicateObjectMap [ 
    rr:predicateMap [
     rr:template ":jumps#{ID}" ] ;
    rr:objectMap [
      rml:reference "MARK" ] ] .
<#TM-SP> a rr:TriplesMap ;
  rr:logicalSource :marks ;
  rr:subjectMap [ 
    rr:template ":jumps#{ID}" ] ;
  rr:predicateObjectMap [ 
    rr:predicate rdf:singletonPropertyOf;
    rr:object :jumps ] ;
  rr:predicateObjectMap [ 
    rr:predicate :date ;
    rr:objectMap [
      rml:reference "DATE" ] ] .
\end{lstlisting}}
\end{multicols}
\end{captionedlisting}
\end{minipage}



\noindent\hspace{0.12\linewidth}\begin{minipage}{\linewidth}
\begin{captionedlisting}{lst:chp4_res-sp-reif}{RDF triples generated by the mapping in \cref{lst:chp4_sp-reif}.}
\centering
\begin{tabular}{c}
\hspace{4em}
{\begin{lstlisting}[basicstyle=\ttfamily\small,label={list:example1},columns=flexible]
:Angelica   :jumps#1  "4.80" .
:jumps#1    :date     "2022-03-21" .
:jumps#1    rdf:singletonPropertyOf :jumps .
:Katerina   :jumps#2  "4.85" .
:jumps#2    :date     "2022-03-19" .
:jumps#2    rdf:singletonPropertyOf :jumps .
\end{lstlisting}}
\end{tabular}
\end{captionedlisting}
\end{minipage}



\textit{\textbf{Named Graphs}}~(\cite{carroll2005namedgraphs}). This approach encloses each reified triple inside a unique named graph. This named graph can then be annotated with an additional triple. In \cref{lst:chp4_graph-reif} a graph is created for the triples that state the height of the jump (\texttt{:G} with the field \texttt{ID}) (\textit{An athlete jumps certain height}). A second graph contains the annotated triple and references the first graph (\texttt{:GA} with the field \texttt{ID}) (\textit{in a specific date}). The resulting RDF triples in TriG syntax\footnote{\url{https://www.w3.org/TR/trig/}} are shown in \cref{lst:chp4_res-graph-reif}.

\begin{minipage}{\linewidth}
\begin{captionedlisting}{lst:chp4_graph-reif}{Example RML mapping using named graphs that transforms data in \cref{lst:chp4_csv_star}.}
\centering
\begin{multicols}{2}
{\begin{lstlisting}[basicstyle=\ttfamily\small,label={list:example1},columns=flexible]
<#TM> a rr:TriplesMap ;
  rml:logicalSource :marks ;
  rr:subjectMap [ 
    rr:template ":{PERSON}" ;
    rr:graphMap [
      rr:template ":G{ID}" ] ] ;
  rr:predicateObjectMap [ 
    rr:predicateMap [
     rr:template ":jumps" ] ;
    rr:objectMap [
      rml:reference "MARK" ] ] .
<#TM-GA> a rr:TriplesMap ;
  rr:logicalSource :marks ;
  rr:subjectMap [ 
    rr:template ":{ID}" ;
    rr:graphMap [
      rr:template ":GA{ID}"] ] ;
  rr:predicateObjectMap [ 
    rr:predicate :date ;
    rr:objectMap [
      rml:reference "DATE" ] ] .
\end{lstlisting}}
\end{multicols}
\end{captionedlisting}
\end{minipage}

\noindent\hspace{0.12\linewidth}\begin{minipage}{\linewidth}
\begin{captionedlisting}{lst:chp4_res-graph-reif}{RDF triples generated by the mapping in \cref{lst:chp4_graph-reif}.}
\centering
\begin{tabular}{c}
\hspace{4em}
{\begin{lstlisting}[basicstyle=\ttfamily\small,label={list:example1},columns=flexible]
:G1  {:Angelica :jumps  "4.80" .}
:GA1 {:G1       :date   "2022-03-21" .}
:G2  {:Katerina :jumps#2  "4.85" .}
:GA2 {:G2       :date     "2022-03-19" .}
\end{lstlisting}}
\end{tabular}
\end{captionedlisting}
\end{minipage}


\textit{\textbf{N-Ary Relationships}}~(\cite{naryw3c2006}). This approach  converts a relationship into an instance that describes the relation, which can have attached both the main object and additional statements. The mapping shown in \cref{lst:chp4_nary-reif} creates instances of the relationship ``jump" with the field \texttt{ID}, which contain the date and height of the jump.  (\textit{An athlete jumps a jump with certain height and in a specific date}). 
The resulting RDF triples are shown in \cref{lst:chp4_res-nary-reif}.

\begin{minipage}{\linewidth}
\begin{captionedlisting}{lst:chp4_nary-reif}{Example RML mapping using a N-ary relationship that transforms data in \cref{lst:chp4_csv_star}.}
\centering
\begin{multicols}{2}
{\begin{lstlisting}[basicstyle=\ttfamily\small,label={list:example1},columns=flexible]
<#TM> a rr:TriplesMap ;
  rml:logicalSource :marks ;
  rr:subjectMap [ 
    rr:template ":{PERSON}" ] ;
  rr:predicateObjectMap [ 
    rr:predicateMap [
     rr:template ":jumps" ] ;
    rr:objectMap [
      rr:template ":Jump{ID}";
      rr:termType rr:IRI] ] .
<#TM-JUMP> a rr:TriplesMap ;
  rr:logicalSource :marks ;
  rr:subjectMap [ 
    rr:template ":Jump{ID}" ] ;
  rr:predicateObjectMap [ 
    rr:predicate :height ;
    rr:objectMap [
      rml:reference "MARK" ] ] ;
  rr:predicateObjectMap [ 
    rr:predicate :date ;
    rr:objectMap [
      rml:reference "DATE" ] ] .
\end{lstlisting}}
\end{multicols}
\end{captionedlisting}
\end{minipage}

\begin{minipage}{\linewidth}
\begin{captionedlisting}{lst:chp4_res-nary-reif}{RDF triples generated by the mapping in \cref{lst:chp4_nary-reif}.}
\centering
\begin{multicols}{2}
{\begin{lstlisting}[basicstyle=\ttfamily\small,label={list:example1},columns=flexible]
:Angelica  :jumps   :Jump1 .
:Jump1     :date    "2022-03-21" .
:Jump1     :height  "4.80" .
:Katerina  :jumps   :Jump2 .
:Jump2     :date    "2022-03-19" .
:Jump2     :height  "4.85" .
\end{lstlisting}}
\end{multicols}
\end{captionedlisting}
\end{minipage}




\subsubsection{Reification with RML-star}



We propose RML-star as an extension of RML to generate \mbox{RDF-star} graphs (\cref{fig:rml-star}). 
\mbox{RML-star} adds a new kind of term map, the \texttt{rml:StarMap}, that allows using triples maps to generate quoted triples. Following the \mbox{RDF-star} data model, star maps can only be used in subject and object maps. Star maps use the property \texttt{rml:quotedTri\-plesMap} to refer to the triples map that generates the quoted triples. This referenced triples map will also generate asserted triples, since it is a \texttt{rr:TriplesMap}. To enable the generation of quoted triples without asserting them, RML-star introduces \texttt{rml:NonAssertedTriplesMap} as a subclass of \texttt{rr:TriplesMap}. Non-asserted triples maps can be referred by \texttt{rml:quotedTriplesMap} to generate quoted triples, but they will be ignored when generating asserted triples.

The \mbox{RML-star} specification~(\cite{iglesias2022rmlstar}) provides a complete description of the language, it is published as a W3C Draft Community Report, and it is maintained by the W3C Knowledge Graph Construction Community Group\cref{foot:kgc}.
Both, the language and the specification are kept up to date reflecting the modifications in \mbox{RDF-star}.
For instance, the latest \mbox{RML-star} releases update the term ``embedded'' to ``quoted'',
according to the modifications in \mbox{RDF-star}.
This update renamed the property \texttt{rml:embeddedTriplesMap} 
%from a previous version\footnote{\url{https://kg-construct.github.io/rml-star-spec/20210706/}} to \texttt{rml:quotedTriplesMap}.
An example of an \mbox{RML-star} mapping rule for the data in \cref{lst:chp4_csv_star} is in \cref{lst:chp4_rml-star} which generates the \mbox{RDF-star} triples in \cref{lst:chp4_res-rml-star}. 
The mapping rules use a non-asserted triples map (\texttt{<\#jumpTM>}) within the subject map of a triples map (\texttt{<\#dateTM>}) which annotates quoted triples (\textit{An athlete jumps certain height}) with \texttt{:date} (\textit{in a specific date}).

\begin{figure*}[!t]
\centering
\includegraphics[width=1\linewidth]{figures/rml-star_diagram.pdf}
\caption{The \mbox{RML-star} extension (represented using the Chowlk notation~(\cite{feria2022chowlk})) with additions and modifications over the RML ontology.}
\label{fig:rml-star}
\end{figure*}

\begin{minipage}{\linewidth}
\centering
\begin{captionedlisting}{lst:chp4_rml-star}{Example RML-star mapping that transforms data in \cref{lst:chp4_csv_star}.}
\centering
\begin{multicols}{2}
{\begin{lstlisting}[numbers=left,basicstyle=\ttfamily\small,label={list:example1},columns=flexible,]
<#jumpTM> 
 a rml:NonAssertedTriplesMap ;
 rml:logicalSource :marks ;
 rml:subjectMap [ 
  rr:template ":{PERSON}" ] ;
 rr:predicateObjectMap [ 
  rr:predicate :jumps ;
  rml:objectMap [
   rml:reference "MARK" ] ] .
<#dateTM> 
 a rr:TriplesMap ;
 rml:logicalSource :marks ;
 rml:subjectMap [ 
  rml:quotedTriplesMap <#jumpTM>];
 rr:predicateObjectMap [ 
  rr:predicate :date ;
  rml:objectMap [
   rml:reference "DATE" ] ] .
\end{lstlisting}}

\end{multicols}
\end{captionedlisting}
\end{minipage}

\noindent\hspace{0.1\linewidth}\begin{minipage}{\linewidth}
\begin{captionedlisting}{lst:chp4_res-rml-star}{RDF-star triples generated by the mapping in \cref{lst:chp4_rml-star}.}
\centering
\begin{tabular}{c}
\hspace{2em}
{\begin{lstlisting}[basicstyle=\ttfamily\small,label={list:example1},columns=flexible]
<< :Angelica :jumps "4.80" >> :date "2022-03-21" .
<< :Katerina :jumps "4.85" >> :date "2022-03-19" .
\end{lstlisting}}
\end{tabular}
\end{captionedlisting}
\end{minipage}







\subsection{Validation}
\label{sec:chp4_validation}

We present the development of the RML star test-cases (\cref{sec:chp4_star_testcases}), that enables new implementations to check their conformance to RML-star; and create RML-star mappings for two real-world use cases (in \cref{sec:chp4_star_usecases}): software metadata extraction~(\cite{kelley2021framework} (SoMEF)) and biomedical research literature~(\cite{SemMedDB}) (SemMedDB). We use for both cases \mbox{Morph-KGC$^{star}$}, an implementation of RML-star.




\subsubsection{RML-star Test Cases}
\label{sec:chp4_star_testcases}

Test cases are commonly used %in standardization processes 
to evaluate the conformance of an engine with respect to a language specification (e.g., RML test cases~(\cite{heyvaert2019conformance})). 
A set of \mbox{RDF-star} test cases was proposed
covering the syntax of various of its serializations\footnote{\url{https://w3c.github.io/rdf-star/tests/}}.
We adapted these test cases to create the RML-star test cases and evaluate the conformance of \mbox{Morph-KGC$^{star}$}
with respect to \mbox{RML-star}.

To create a representative set of test cases for \mbox{RML-star}, we selected the N-Triples-star syntax tests\footnote{\url{https://w3c.github.io/rdf-star/tests/nt/syntax}}.  %;and follow a reverse engineering process. 
For each \mbox{RDF-star} test case, we created two associated \mbox{RML-star} test cases that generate the original \mbox{RDF-star} dataset: one test case with a single input data source (i.e., the mapping does not include joins) and another with two input data sources (i.e., the mapping includes joins among triple maps).
For each test case, we manually created the input source(s) in the CSV format and the corresponding \mbox{RML-star} mapping rules to generate the output \mbox{RDF-star} datasets.
Following this approach, we obtained 16 \mbox{RML-star} test cases.
The test cases are openly available at the W3C Community Group on Knowledge Graph Construction~(\footnote{\url{https://github.com/kg-construct/rml-star-test-cases}}),
and can be reused by any engine to test its conformance with respect to \mbox{RML-star}.


\subsubsection{Use Cases}
\label{sec:chp4_star_usecases}

We create RML-star mappings for in two real-world use cases. 
The first generates \mbox{RDF-star} graphs from scientific software documentation, 
and the second annotates statements extracted from biomedical research publications. 
For both use cases, we use the tool \mbox{Morph-KGC$^{star}$}.









%In order to ensure that these results in terms of time are representative, we run each mapping three times and calculate the average time.

%In both use cases we report the time on generating the RDF-star datasets. 




%In summary, our test and use cases show that \mbox{Morph-KGC$^{star}$} generates valid RDF triples and does not add an overhead in the generation time of reified triples. However, a more thorough analysis (out of the scope of this paper) is required to describe the behaviour of each mapping approach.

\noindent\textbf{\textit{Scientific Software Metadata Extraction.}}
Scientific software has become a crucial asset to deliver and reproduce the results described in research publications~(\cite{chue_hong_fair_2021}). However, scientific software is often time consuming to understand and reuse due to incomplete and heterogeneous documentation, available only in a human-readable manner.
The Software Metadata Extraction Framework (SoMEF)~(\cite{somef}) proposes an approach to automatically extract relevant metadata (description, installation instructions, citation, etc.) from code repositories and their documentation. SoMEF includes different text extraction techniques (e.g., supervised classification, regular expressions, etc.) that yield results with different confidence values.
For example, \cref{lst:JSONsnippet} shows a JSON snippet with the description that SoMEF obtained from a software repository (Widoco) using the GitHub API.
The confidence in this case is high as the extracted description was manually curated by the creators of the code repository.
SoMEF extracts more than 30 different metadata fields about 
software, its source code, its released versions, and their corresponding authors. For transforming the output of SoMEF into RDF-star, we used a total of 35 triples maps to annotate software metadata fields and an additional triples map to annotate source code descriptions. All reified triples follow the same structure (Listings \ref{lst:JSONsnippet} \& \ref{lst:TTLsnippet}), i.e. the standard RDF triple contains the excerpt of the extracted feature, and it is annotated
with the \emph{technique} used and the \emph{confidence} value. 
The complete mapping and all input examples and results are available online\footnote{\url{https://github.com/oeg-upm/rdf-star-generation}}.
 
\noindent\hspace{0.1\linewidth}\begin{minipage}{0.8\linewidth}
\begin{captionedlisting}{lst:JSONsnippet}{JSON snippet showing the description metadata field extracted by SoMEF on a code repository using the GitHub API as extraction technique.}
\centering
\hspace{3em}
{
\begin{lstlisting}[basicstyle=\ttfamily\small,label={list:example1},columns=flexible]
"codeRepository": "https://github.com/oeg-upm/Widoco",
"description": [ 
  {
    "confidence": [
      1.0
    ],
    "excerpt": "Wizard for documenting ontologies. WIDOCO is ...",
    "technique": "GitHub API"
  }
]  
\end{lstlisting}
}
\end{captionedlisting}
\end{minipage}

Capturing the technique used and the confidence obtained for each extracted metadata field is key for obtaining an accurate representation of the result. Hence, the \mbox{RDF-star} representation corresponding to the JSON in \cref{lst:JSONsnippet} includes this information, as depicted in \cref{lst:TTLsnippet}.

%\begin{minipage}{0.9\linewidth}
%\begin{captionedlisting}{lst:TTLsnippet}{N-Triples-star snippet showing the results generated for the %description field shown in \cref{lst:JSONsnippet}. Each asserted triple is annotated with its %corresponding confidence and technique.}
%\centering
%\begin{tabular}{c}
%\hspace{1.4em}
%{
%\begin{lstlisting}[numbers=left]
%<< <https://example.org/oeg-upm/Widoco> <https://w3id.org/okn/o/sd#description> 
%    "Wizard for documenting ontologies. WIDOCO is ..." >> 
%        <https://www.w3id.org/okn/o/em#technique> "GitHub API" .
%<< <https://example.org/oeg-upm/Widoco> <https://w3id.org/okn/o/sd#description> 
%    "Wizard for documenting ontologies. WIDOCO is ..." >> 
%        <https://www.w3id.org/okn/o/em#confidence> "1.0" .
%<https://example.org/oeg-upm/Widoco> <https://w3id.org/okn/o/sd#description> 
%    "Wizard for documenting ontologies. WIDOCO is ..." .
%\end{lstlisting}
%}
%\end{tabular}
%\end{captionedlisting}
%\end{minipage}        


\begin{captionedlisting}{lst:TTLsnippet}{RDF-star triples snippet showing the results generated for the description field in \cref{lst:JSONsnippet}. Each asserted triple is annotated with its corresponding confidence and technique.}
\centering
{
\begin{lstlisting}[basicstyle=\ttfamily\small,label={list:example1},columns=flexible]
ex:oeg-upm/Widoco :description "Wizard for documenting ontologies. WIDOCO is ..." .
<<ex:oeg-upm/Widoco :description "Wizard for documenting ontologies. WIDOCO is ...">> 
    :technique "GitHub API" .
<<ex:oeg-upm/Widoco :description "Wizard for documenting ontologies. WIDOCO is ...">> 
    :confidence "1.0" .
\end{lstlisting}
}
\end{captionedlisting}


%For example, for \cref{lst:JSONsnippet}, the excerpt of the software's description ("Wizard for documenting ontologies. WIDOCO is ...") was extracted with the technique "GitHub API" with a confidence of 1.0. 
% talk about the execution
%The reified triples are used for describing the software, that uses


\noindent\textbf{\textit{Biomedical Research Literature.}} 
SemMedDB~(\cite{SemMedDB}), the Semantic MEDLINE Database, is a repository 
that contains information on extracted biomedical entities 
and predications (subject-predicate-object triples) 
from biomedical texts (titles and abstracts from PubMed citations). 
%consists of a set of triples and annotations automatically extracted from PubMed titles, abstracts, and citations. 
%a set of data sources with automatic semantic annotations from titles and abstracts of citations available in PubMed.
The tables comprising SemMedDB are available for download as a relational database or CSV files\footnote{ \url{https://lhncbc.nlm.nih.gov/ii/tools/SemRep_SemMedDB_SKR/SemMedDB_download.html}}.
We downloaded the MySQL files for (1)~predication predictions (PREDICATION and PREDICATION\_AUX tables), containing more than 117 million annotations; and (2)~entity predictions (ENTITY table), which include more than 410 million annotations.
Listings \ref{lst:chp4_semmeddb_entity}, \ref{lst:chp4_semmeddb_pred} and \ref{lst:chp4_semmeddb_predaux} illustrate the columns used from the tables with synthetic data. 
For predications, only data for subjects is shown; the missing columns regarding objects follow the same structure as subjects.
%This data contains information on predicted entities and predicted subject-predicate-object predications.
Subjects and objects, from predications, and entities are assigned a semantic type 
(that categorizes the extracted concept in the biomedical domain) annotated with a confidence score. 
In addition, the extraction of subjects and objects is assigned a timestamp on when it took place. 
Thus, the score and timestamp represent metadata about other statements.
We created an RML-star mapping with 5 triples maps quoting triples:
3 of them are used to annotate the assignation of semantic types to entities, subjects, and objects with confidence scores;
the remaining 2 provide the timestamps for the extraction of subjects and objects. 

\begin{minipage}{0.38\linewidth}
\begin{captionedlisting}{lst:chp4_semmeddb_entity}{ENTITY table snippet.}
\centering
\begin{tabular}{c}
\hspace{-1em}
{
\begin{lstlisting}[basicstyle=\ttfamily\small,label={list:example1},columns=flexible]
ENTITY_ID , SEMTYPE , SCORE
12345     , orga    , 790
\end{lstlisting}
}
\end{tabular}
\end{captionedlisting}
\end{minipage}
\,\,\,\,\hfill
\begin{minipage}{0.6\linewidth}
\begin{captionedlisting}{lst:chp4_semmeddb_pred}{PREDICATION table snippet.}
\centering
\begin{tabular}{c}
\hspace{-1em}
{
\begin{lstlisting}[basicstyle=\ttfamily\small,label={list:example1},columns=flexible]
PREDICATION_ID , SUBJECT_SEMTYPE , SUBJECT_NAME
13579          , Semtype         , SubjName
\end{lstlisting}
}
\end{tabular}
\end{captionedlisting}
\end{minipage}

\noindent\hspace{0.23\linewidth}\begin{minipage}{0.8\linewidth}
\begin{captionedlisting}{lst:chp4_semmeddb_predaux}{PREDICATION\_AUX table snippet.}
\centering
\begin{tabular}{c}
\hspace{-7em}
{
\begin{lstlisting}[basicstyle=\ttfamily\small,label={list:example1},columns=flexible]
PREDICATION_AUX_ID, PREDICATION_ID, SUBJECT_SCORE, TIMESTAMP
67890             , 13579         , 800          , 1651740766
\end{lstlisting}
}
\end{tabular}
\end{captionedlisting}
\end{minipage}

\noindent\hspace{0.1\linewidth}\begin{minipage}{0.78\linewidth}
\begin{captionedlisting}{lst:chp4_semmeddb_triples}{RDF-star triples generated from data in Listings \ref{lst:chp4_semmeddb_entity}, \ref{lst:chp4_semmeddb_pred} and \ref{lst:chp4_semmeddb_predaux}.}
\centering
\begin{tabular}{c}
\hspace{-1.5em}
{
\begin{lstlisting}[basicstyle=\ttfamily\small,label={list:example1},columns=flexible]
<<ex:12345 sem:semanticType "orga">> sem:score "790" .
<<ex:13579 sem:subject ex:SubjName>> sem:timestamp "1651740766" .
<<ex:SubjName sem:semanticType "Semtype">> sem:score "800" .
\end{lstlisting}
}
\end{tabular}
\end{captionedlisting}
\end{minipage}



\section{Conclusions}

In this chapter, we performed an in-depth analysis of the features of the current mapping languages: from the description of input data sources, to how the triples are generated and data transformations. This analysis and the challenges proposed by the Knowledge Graph Construction W3C Community Group are gathered as requirements for a mapping language to be able to address the complexity of use cases for KG construction. Then, we presented the Conceptual Mapping, an ontology that, based on the abovementioned requirements, gathers the expressiveness of current mapping languages. This ontology was developed following the LOT methodology, and each step of the process and how if was addressed when building the ontology is explained. Finally, we show of mapping languages evolve as the semantic technologies advance, in this case with the adoption of RDF-star. To that purpose, we extend one of the most used mapping languages to create RDF-star graphs, RML-star. Accordingly, we add an extension to the Conceptual Mapping to include this new feature. Our ambition with this contribution is to enhance the understanding of the capabilities of mapping languages and what they need to implement in order to cover the necessities for constructing knowledge graphs, and at the same time be able to evolve as the technologies evolve.


\chapter{Enhancing the Creation of Mapping Rules}
\label{chapter:creation}

The mappings used for constructing knowledge graphs can be created in different ways. As they are formatted as text documents, they can be directly written in any text editor. However, the verbosity of some languages and the need of following a specific syntax has foster the development of user-friendly tools and serializations to ease the mapping writing process. This chapter introduces two solutions for easing the construction of mappings for different potential users, (i) a spreadsheet-based approach to write the mapping rules, and (ii) a user-friendly mapping syntax updated with the latest needs in knowledge graph construction.

\section{Mapping rules in spreadsheets: Mapeathor}
\label{sec:chp5_mapeathor}

\ana{no sé muy bien qué poner por aquí, si ampliar la motivación antes de esta sección, poner algo de related work o solo referenciar, mirar otras tesis a ver}

\subsection{Spreadsheet design}

%The rules required to generate a knowledge graph can be specified in multiple languages. The language is chosen by the user depending on the specific use case. However, the rules themselves are equivalent across languages, so they can be written in a language-independent way, in this case, we chose a spreadsheet for rule specification. 

In this section we present the design of the spreadsheet template to write the mapping rules. 
These mapping rules are specified in the spreadsheets following a provided structure to only write the essential information, forgetting about the syntax peculiarities of each languages (e.g. keywords, semicolons, brackets, etc.).
This design is devised to contain the rules in a compact and understandable way using format widely used by the scientific community (i.e. Google spreadsheets, MS Excel). 
Using these spreadsheets along with their corresponding editors allow users to leverage their native advantages.
The mapping rules are specified in the spreadsheet across five different sheets, which are described in detail below: \textit{Prefix sheet}, \textit{Source sheet}, \textit{Subject sheet}, \textit{Predicate\_Object sheet} and \textit{Function sheet}. To illustrate this section we use the files in \ana{listing con ejemplo } as input data source.

\ana{poner al final de la sección como traduce o sección a sección; a RML}

\subsubsection{Prefix sheet} 
This sheet contains the namespaces and corresponding prefixes used when declaring the transformation rules. 
It is composed of two columns: \texttt{Prefix} for the prefix and \texttt{URI} for the corresponding namespace. The base namespace can be specified writing ``@base" in the \texttt{Prefix} column.The namespaces used by the target mapping language are automatically added in the translation (e.g. \texttt{rr}\footnote{\url{http://www.w3.org/ns/r2rml\#}}, \texttt{rml}\footnote{\url{http://semweb.mmlab.be/ns/rml\#}}).
The example shown in \cref{tab:prefix_sheet} presents how three namespaces and the base namespace are written in the sheet. 

\begin{table}[h!]
\caption{Prefix sheet.}
\label{tab:prefix_sheet}
\centering
\begin{tabular}{c|c}
\midrule
\textbf{Prefix} & \textbf{URI}                                 \\ \midrule
@base           & http://example.com/                          \\
rdf             & http://www.w3.org/1999/02/22-rdf-syntax-ns\# \\
ex              & http://ex.com/                               \\ 
grel            & http://semweb.datasciencelab.be/ns/grel\#     \\
\midrule
\end{tabular}
\end{table}

\ana{si tal poner que sin prefijo se genera ":", e implementar equisdé}

\subsubsection{Subject sheet} 
This sheet defines the subjects to be generated and their corresponding identifier (\texttt{ID}) that groups the mapping rules for each subject. It is organized in four columns: \texttt{ID}, \texttt{URI}, \texttt{Class} and \texttt{Graph}. \texttt{ID} contains a unique identifier for each subject's set of rules in order to relate to information on these rules in the remaining sheets.
\texttt{URI} defines the subject URI of the resources that are generated in the mapping. 
\texttt{Class} allows the assignation of the subject to a class with \texttt{rdf:type}. A subject may be type of one class, more than one or none at all. 
Finally, \texttt{Graph} is an optional column that enables the assignation of a named graph to the triples generated for a subject.

The example shown in \cref{tab:subject_sheet} shows how to write two subjects, each with a different identifier and base URI. The instances of the subject with the \texttt{PERSON} ID is type of two classes, (\texttt{ex:Person} and \texttt{ex:Athlete}), while all the triples of the instances of the subject identified with the \texttt{SPORT} ID are assigned to a named graph (\texttt{ex:SportsGraph}). 


\begin{table}[h!]
\caption{Subject sheet.}
\label{tab:subject_sheet}
\centering
\begin{tabular}{c|c|c|c}
\midrule
\textbf{ID} & \textbf{URI} & \textbf{Class} & \textbf{Graph} \\ \midrule
PERSON & http://ex.com/Person/\{name\} & ex:Person &  \\
PERSON & http://ex.com/Person/\{name\} & ex:Athlete &  \\
SPORT & http://ex.com/Sport/\{sport\} & ex:Sport & ex:SportsGraph \\ \midrule
\end{tabular}
\end{table}

\ana{release porque lo de los graphs no aparece en la demo}

\subsubsection{Source sheet} 

This sheets describes where the data for each ID is retrieved from. The information is organized in three columns: \texttt{ID}, \texttt{Feature} and \texttt{Value}. 
\texttt{ID} makes reference to the IDs declared in the \textit{Subject sheet}. 
\texttt{Feature} declares the type of information provided in \texttt{Value}. The allowed keywords in this column are: \texttt{source} for the path and name of the source, \texttt{format} for the data source format, \texttt{iterator} for hierarchical data (e.g. JSON, XML), \texttt{table} for the table names of RDBs, \texttt{query} for SQL queries and \texttt{SQLVersion} for the SQL version. Each ID must have at least the \texttt{source} and \texttt{format} written; the rest of the features are optional. 
Then, in \texttt{Value} the corresponding values of the \texttt{Feature} specified are written. 

\cref{tab:source_sheet} shows the data source description for the \texttt{PERSON} and \texttt{SPORT} IDs. The former corresponds to a CSV file, while the latter to a JSON file, for which the iterator is also written (\texttt{\$.sport}). 

\begin{table}[h!]
\caption{Source sheet.}
\label{tab:source_sheet}
\centering
\begin{tabular}{c|c|c}
\midrule
\textbf{ID} & \textbf{Feature} & \textbf{Value}              \\ \midrule
PERSON    & source          & /home/user/data/people.csv  \\
PERSON    & format          & CSV                         \\
SPORT     & source          & /home/user/data/sports.json \\
SPORT     & format          & JSON                        \\  
SPORT     & iterator        & \$.sport                    \\ \midrule
\end{tabular}
\end{table}

\subsubsection{Predicate\_Object sheet} In this sheet users can specify how to generate the triples, that is, predicate-object pairs for the subjects defined in the \textit{Subject sheet}. This sheet contains up to 8 columns: \texttt{ID}, \texttt{Predicate}, \texttt{Object}, \texttt{DataType}, \texttt{Language}, \texttt{ReferenceID}, \texttt{InnerRef} and \texttt{OuterRef}.

The column \texttt{ID} indicates the subject to which the triples belong.
The columns \texttt{Predicate} and \texttt{Object} specify the predicate and object of the triple respectively. 
The XSD datatype of \texttt{Object} is defined in \texttt{DataType}, and the language tag in \texttt{Language}. Both these columns are optional. 
When the object refers to a subject defined in another mapping rule, the rule is written as follows. 
There are three fields that allow the specification of the linking condition between the object of the current triple and the referenced subject. 
They specify which is the ID of the referred subject  (\texttt{ReferenceID}), and the ''join'' fields in the source data (\texttt{InnerRef} for the field of the object of the current triple, and \texttt{OuterRef} for the field of the referred subject).  

\cref{tab:po_sheet} shows an example of the \textit{Predicate\_Object sheet}. For each ID three predicate-object pairs are specified, both containing literals with different datatypes and some with language tags. The rule set identified as \texttt{PERSON} generates a link to the subject of the \texttt{SPORT} rule set using the predicate \texttt{ex:name}, and joining by equal values of the field \texttt{sport\_id} from \texttt{PERSON} and the field \texttt{id} from \texttt{SPORT}. Finally, one object of the \texttt{SPORT} id is created using a function, which is identified by being enclosed by ``\textless{}\textgreater{}".

\begin{table}[h!]
\caption{Predicate\_Object sheet.}
\label{tab:po_sheet}
\centering
\resizebox{\columnwidth}{!}{
\begin{tabular}{c|c|c|c|c|c|c|c}
\midrule
\textbf{ID} &\textbf{Predicate} & \textbf{Object}               & \textbf{DataType} & \textbf{Language} & \textbf{ReferenceID} & \textbf{InnerRef} & \textbf{OuterRef} \\ \midrule
PERSON & ex:name & \{name\} & string & en & & &  \\
PERSON & ex:birthdate & \{birthdate\} & date & & & & \\
PERSON & ex:sport & & & & SPORT& sport\_id & id \\
SPORT & ex:name & \{sport\} & string & en & & & \\
SPORT & ex:code & \{id\} & integer & & & & \\
SPORT & ex:comment & \textless{}Fun1\textgreater{} & & & & & \\ \midrule
\end{tabular}
}
\end{table}

\subsubsection{Function sheet} Some languages are able to process data transformation functions (e.g. FnO+RML), which can be detailed in this sheet. Some well known options are the SQL and GREL functions. The functions are referred in the Predicate\_Object sheet or in other function rows with the identifier specified in \texttt{FunctionID}. The column \texttt{Feature} is used to specify the type of information provided in \texttt{Value}, where the name of the function and the value of the parameters are written. The example shown in \cref{tab:function_sheet} uses the function \texttt{grel:toLowerCase} taking one data field as input parameter.

\begin{table}[h!]
\caption{Function sheet.}
\label{tab:function_sheet}
\centering
\begin{tabular}{c|c|c}
\midrule
\textbf{FunctionID} & \textbf{Feature} & \textbf{Value} \\ \midrule
\textless{}Fun1\textgreater{} & fno:executes & grel:toLowerCase \\  
\textless{}Fun1\textgreater{} & grel:valueParam & \{comment\} \\
\midrule
\end{tabular}
\end{table}

\ana{pensar fuertemente si incluir esto o dejarlo fuera, está completamente orientado a FnO. Si se deja, hay que modificarlo para que sea más general, tipo 'function' para el nombre de la función, y el resto de parámetros que se quede así.}

%\textit{hablar aquí del diseño de la spreadsheet, que puede ser google o excel, y cada una de las pestañas poquito a poquito con sus tablas de ejemplo}

\subsection{Experimental evaluation}

\textit{empezar hablando un poco de la herramienta y el setup de los experimentos}

\textit{Aquí por una parte va a estar el estudio de usuario (JE JE), incluyendo las encuestas. También se puede meter la conformance con los lenguajes, a ver si no sale muy mal eso}

\textit{y es que igual debería poner que nos limitamos a otras herramientas pero no a syntaxis, para que luego tenga más sentido al poner yarrrml después}

\section{User-friendly creation of RDF-star with YARRRML}
\label{sec:chp5_yarrrml_star}

\section{Conclusions}

This chapter addresses the second objective of this thesis \textit{O2: To help knowledge engineers and domain experts to build mappings documents providing the means for a user-friendly experience.}
%In this chapter, we extend the tooling assistance of mapping creation for users. 
The first contribution within this objective is \textit{C4: Design of a spreadsheet-based approach for writing mapping rules.} This contribution achieves a two-fold objective: (i) To ease the writing of mapping rules in a familiar environment (spreadsheets), while erasing the need of learning a mapping language syntax; and (ii) to increase the interoperability among languages, as the spreadsheet representation can be translated into more than one language. We test the developed approach in a user study with 30 volunteer practitioners, comparing it to a graphical approach (RMLEditor) and the baseline (writing mappings directly in the RML mapping language). The results suggest that participants manage to write mappings with higher quality and more complete than with the other approaches. Thus, this suggest that the spreadsheet-based approach is suitable for enhancing the mapping writing process for users. 

The second contribution tackles \textit{C5: Update of the a user-friendly syntax with extended features}. 
We present the latest updates to a user-friendly syntax for the RML mapping language, YARRRML, one of the most popular and adopted for writing RML mappings. 
We qualitatively validate the updates proposed by comparing them to other user-friendly syntaxes to ensure maximum expressiveness coverage. 

Finally present the implementations for both presented approaches, Mapeathor for generating [R2]RML and YARRRML mappings from spreadsheets, and Yatter for translating the updated YARRRML syntax into [R2]RML. The objective of the contributions presented in this chapter is to provide further support for user to ease the writing of mappings, aiming at reducing the barrier for adopting mappings to build knowledge graphs. 

%\ana{añadir otra contribución aquí con la interoperabilidad de las herramientas?}



\chapter{Mappings in Knowledge Graph Evolution}
\label{chapter:reframing}

\textit{Aquí va a ir primero el motivating example, que será el paper de ISI, para decir por qué surge la pregunta. Después, lo siguiente, de creacion vs construct, con su propia evaluación y demás.}



\section{Mappings in the re-construction of Knowledge Graphs from a reification perspective}
\label{sec:chp6-1_re-construction}

Knowledge graphs are not immutable, as they along with the ontology or network of ontologies that defines their schema are subject to modifications. This need for evolution may triggered by changes in the domain of knowledge or on the graph consumption requirements in downstream tasks. 
An example is the need to add information about a triple, thus making statements about statements (i.e. statement reification).
%Reification is usually used to incorporate metadata and provenance to existing triples~\citep{frey2019evaluation}.
%Throughout the years, different representations have been proposed for reifying statements in RDF 
%, such as Named Graphs~\citep{carroll2005named}, N-Ary Relationships~\citep{naryw3c2006} or RDF-star~\citep{hartig2017foundations}. 
For instance, Nanopublications~\citep{groth2010nanopubs} adopt a Named Graph-based~\citep{carroll2005named} model to publish minimal scientific statements along with their provenance and associated context. Meanwhile, in ontology engineering, N-Ary Relationships~\citep{naryw3c2006} are widely used as an ontology design pattern~\citep{gangemi2013multi}. It is not uncommon for an existing KG to strive for adopting one of these representations, such as the publication of the DisGeNet KG as Nanopublications~\citep{queralt2016disgenet} or introducing RDF-star in the European Agency of Railway  KG.\footnote{\url{https://data-interop.era.europa.eu/}}

When the situation comes that a re-structuration of the knowledge graph is needed, the following question arises: Which is more convenient, to construct the graph from the original input data sources again, or to re-construct it within the triplestore? This section analyses which of these approaches reports better performance and the parameters that influence the choice. 
To this end, we perform a comprehensive empirical study of this transformation in four representations (Standard Reification, N-Ary Relationships, Named Graphs and RDF-star) using (i) declarative mapping languages to construct each representation with KG construction engines and (ii) transformation per peers of representations with SPARQL \texttt{CONSTRUCT} queries in different triplestores. This section aims to unravel the aspects to consider when re-constructing a knowledge graph, to help in the decision on how to perform it. We also raise awareness of recently overlooked aspects of \texttt{CONSTRUCT} queries, while contributing to research on the impact of RDF reification to enhance KG management.

\subsection{Motivating Example}
\label{sec:chp6-1_mot-example}

\begin{figure}[t!]
    \centering
    \includegraphics[width=1\linewidth]{figures/chp6-1_motivating-example.pdf}
    \caption{Alternatives for re-constructing a pre-existing knowledge graph.}
    \label{fig:chp6-1_mot-example}
\end{figure}

Consider the SemMedDB dataset, that contains entities extracted from biomedical texts, as well as the semantic types of these concepts and a timestamp of when they were extracted. A stable version of a knowledge graph represents this information, where the timestamp and semantic type are related to its correspondent entity with an N-Ary relationship. However, the maintainers of this knowledge graph want to reduce the size of the graph by reducing the number of nodes, as well as to represent this information with RDF-star to start adopting the RDF 1.2 specification.

The graph maintainers have access to the original source data, so they can change the original KG construction pipeline, that uses declarative mappings to construct the graph. However, since this graph is not going to be updated with new data, it is also possible to change the structure of the graph with queries within the triplestore where it is maintained. They know the time it takes to generate the graph in its original representation. However, they are not certain whether generating the graph in the new representation using this pipeline will take more time, or if with queries the transformation could be performed faster (\cref{fig:chp6-1_mot-example}). Hence, the objective of this work consist of assessing the aspects that influence this situation to help in making an informed decision.


\subsection{Methodology}
\label{sec:chp6-1_methodology}

In this section we present the methodology followed to analyze the impact of different approaches for re-constructing knowledge graphs from a reification perspective. More in detail, we compare two ways of re-structuring a KG to change its reification approach, (i) with declarative mapping technologies from scratch (e.g., RML~\citep{iglesias2023rml}) and (ii) with \texttt{CONSTRUCT} queries in a triplestore from the previous version of the graph. We aim to answer the following research questions: 
\textbf{(RQ1)} In which cases is it more efficient in terms of time to either construct a reified knowledge graph from the original data sources or to re-construct it within a triplestore?
\textbf{(RQ2)} Which of these two approaches is more scalable as the size of the data increases?
\textbf{(RQ3)} What is the impact of the different reification representations on triplestores and mapping processors?
%In this section, we describe the experimental configuration for the evaluation carried out to answer these research questions. 
All resources to reproduce the experiments are available on GitHub.\footnote{\url{https://github.com/oeg-upm/kg-reconstruction-eval}}



\subsubsection{Dataset}
\label{sec:chp6-1_dataset}

\ana{ojo que esta sección es como para la de rml-star}

We use the Semantic MEDLINE Database (SemMedDB)~\citep{SemMedDB2012} in the experimental evaluation. This database consists of a repository with biomedical entities and relationships (subject-predicate-object) extracted from biomedical texts.
%, mainly titles and abstracts from PubMed citations. 
This dataset is available as a relational database and CSV files.\footnote{\url{https://lhncbc.nlm.nih.gov/ii/tools/SemRep\_SemMedDB\_SKR/SemMedDB\_down\-load.html}}
It is licensed under the UMLS - Metathesaurus License Agreement,\footnote{\url{https://www.nlm.nih.gov/research/umls/knowledge\_sources/metathesaurus/release\-/license\_agreement.html}} which does not allow its distribution, but it may be accessed with an account with the UMLS license.\footnote{An account with the UMLS license can be requested at \url{https://www.nlm.nih.gov/databases/umls.html}.}


We use the CSV files for (i)~\textit{entity} predictions (from ENTITY.csv), and (ii)~\textit{predication} predictions (from PREDICATION.csv and PREDICATION\_AU\-X.csv). Listings~\ref{lst:chp6-1_semmeddb_entity},~\ref{lst:chp6-1_semmeddb_pred} and~\ref{lst:chp6-1_semmeddb_predaux} illustrate the columns used from the files with synthetic data.
For \textit{predications}, only data for \textit{subjects} is shown; the missing columns regarding \textit{objects} follow the same structure as \textit{subjects}.
%This data contains information on predicted entities and predicted subject-predicate-object predications.
\textit{Subjects} and \textit{objects} (from \textit{predications}), and \textit{entities} are assigned a \textit{semantic type} with a confidence score.
These \textit{semantic types} categorize the extracted concept in the biomedical domain.\footnote{\url{https://www.nlm.nih.gov/research/umls/new\_users/online\_learning/SEM\_003.html}}
In addition, the extraction of \textit{subjects} and \textit{objects} is assigned a timestamp on when it took place. 
Thus, the score and timestamp represent metadata about other statements, which are fit to represent using reification.

We model this tabular dataset as five annotated statements: 
Three assign \textit{semantic types} to \textit{subjects}, \textit{objects}, and \textit{entities} with a confidence score; and two provide the timestamp for the extraction of \textit{subjects} and \textit{objects} from text.  


\noindent\begin{minipage}{0.42\linewidth}
\begin{captionedlisting}{lst:chp6-1_semmeddb_entity}{ENTITY.csv snippet.}
\centering
\begin{tabular}{c}
\hspace{-0.7em}
{
\begin{lstlisting}[basicstyle=\ttfamily\small,label={lst:chp6-1_semmeddb_entity},columns=flexible]
ENTITY_ID , SEMTYPE , SCORE
12345      , orga     , 790
\end{lstlisting}
}
\end{tabular}
\end{captionedlisting}
\end{minipage}
\,\,\,\,\hfill
\begin{minipage}{0.58\linewidth}
\begin{captionedlisting}{lst:chp6-1_semmeddb_pred}{PREDICATION.csv snippet.}
\centering
\begin{tabular}{c}
\hspace{-1em}
{
\begin{lstlisting}[basicstyle=\ttfamily\small,label={lst:chp6-1_semmeddb_pred},columns=flexible]
PREDICATION_ID , SUBJ_SEMTYPE , SUBJ_NAME
13579            , Semtype       , SubjName
\end{lstlisting}
}
\end{tabular}
\end{captionedlisting}
\end{minipage}

\noindent\hspace{0.23\linewidth}\begin{minipage}{\linewidth}
\begin{captionedlisting}{lst:chp6-1_semmeddb_predaux}{PREDICATION\_AUX.csv snippet.}
\centering
\begin{tabular}{c}
\hspace{-7em}
{
\begin{lstlisting}[basicstyle=\ttfamily\small,label={lst:chp6-1_semmeddb_predaux},columns=flexible]
PREDICATION_AUX_ID , PREDICATION_ID , SUBJ_SCORE , TIMESTAMP
67890                , 13579            , 800         , 1651740766
\end{lstlisting}
}
\end{tabular}
\end{captionedlisting}
\end{minipage}


To test the scalability of the evaluated approaches, we subset this dataset into four sizes taking as input from the abovementioned CSV files with (i)~1K rows, (ii)~10K rows, (iii)~100K rows and (iv)~1M rows. The number of triples produced for each reification version in each scale is shown in \cref{tab:chp6-1_data-triple-number}.

\begin{table}[t]
\caption{Number of triples of the SemMedDB graph in the selected representations for each scale.}
\label{tab:chp6-1_data-triple-number}
\centering
\begin{tabular}{cccccc}
    \cmidrule{2-6}
    & & \textbf{1K} & \textbf{10K} & \textbf{100K} & \textbf{1M} \\ \midrule
    \textbf{Standard Reification} & & 25,000 & 249,997 & 2,499,966 & 24,999,607 \\ \midrule
    \textbf{Named Graphs} & & 10,000 & 99,994 & 999,932 & 9,999,190 \\ \midrule
    \textbf{N-Ary Relationships} & & 15,000 & 149,997 & 1,499,966 & 14,999,595 \\ \midrule
    \textbf{RDF-star} & & 8,485 & 78,655 & 710,588 & 6,503,388 \\ \bottomrule
\end{tabular}
\end{table}




\subsubsection{Mappings and Queries}
\label{sec:chp6-1_map-queries}
The following resources are used: 
(i)~a set of declarative mappings that are used for constructing the knowledge graph from the SemMedDB tabular files in the four selected reification representations; and (ii)~a set of SPARQL queries that are used for re-constructing the graph within different triplestores. A summary of their characteristics is shown in \cref{tab:chp6-1_mapping-char}.

We use two sets of mappings, one set written in the RML mapping language~\citep{iglesias2023rml}, and the other in SPARQL-Anything~\citep{asprino2023sparql-anything}. These languages allow describing transformation of heterogeneous data sources into RDF following the schema provided by an ontology or vocabulary, and are processed by different engines (see \cref{sec:chp6-1_engines}). Each set of mappings is comprised of four mappings, to construct the knowledge graphs in the four representations selected (i.e. Standard Reification, N-Ary Relationships, Named Graphs and RDF-star).


RML extends the R2RML Recommendation~\citep{R2RML} to describe more data sources besides Relational Databases. RML rules are grouped within \textit{Triples Maps}, which contain one \textit{Logical Source}, one \textit{Subject Map} and zero to multiple \textit{Predicate Object Maps}. \textit{Logical Sources} describe the input data to be transformed. \textit{Subject Maps} indicates how the subjects of the triples are created, while \textit{Predicate Object Maps} specify how the predicates and objects of the triples are created. \cref{lst:chp6-1_rml-star} shows an example of a mapping that generates in RDF-star the annotation of the semantic types of \textit{entities} with a score from the CSV file shown in \cref{lst:chp6-1_semmeddb_entity}. This mapping uses the RML-star module~\citep{delva2021rml-star,iglesias2023rml} to generate RDF-star graphs, which allows RML to quote \textit{Triples Map} with the \texttt{rml:quotedTriplesMap} property. For these mappings, we show the number of sets of rules (i.e. \textit{Triples Map}) and  \textit{Predicate Object Maps} specified (\cref{tab:chp6-1_mapping-char}).

\noindent\begin{minipage}{1\linewidth}
\begin{captionedlisting}{lst:chp6-1_rml-star}{RML-star mapping snippet to create the RDF-star graph for \textit{entity} from data in \cref{lst:chp6-1_semmeddb_entity}. }
\centering
\begin{tabular}{cc}
{\begin{lstlisting}[basicstyle=\ttfamily\small,label={list:example1}]
<#Entity> 
 a rml:TriplesMap ;
 rml:logicalSource [
  rml:source "ENTITY.csv"
 ];
 rml:subjectMap [
  rml:template ":{ENTITY_ID}"
 ];
 rml:predicateObjectMap [
  rml:predicate :semanticType;
  rml:objectMap [
   rml:reference "SEMTYPE"
  ] ] .
\end{lstlisting}}
&
%\hspace{3em}
{\begin{lstlisting}[basicstyle=\ttfamily\small,label={list:example1},numbers=right,firstnumber=13]
<#EntityScore> 
  a rml:AssertedTriplesMap ;
 rml:logicalSource [
  rml:source "ENTITY.csv"
 ];
 rml:subjectMap [
  rml:quotedTriplesMap <#Entity>;
 ];
 rml:predicateObjectMap [
  rml:predicate :score ;
  rml:objectMap [
   rml:reference "SCORE"
  ] ] .
\end{lstlisting}}

\end{tabular}
\end{captionedlisting}
\end{minipage}


\begin{table}[t!]
    \caption{Characteristics of mappings in RML and SPARQL-Anything. \#TM stands for number of Triples Map, \#POM for Predicate Object Map, and \#TP for Triple Patterns. The shown operators appear usually in the \texttt{WHERE} clause; the ones marked with $^c$ appear in the \texttt{CONSTRUCT} clause. }
    \label{tab:chp6-1_mapping-char}
    \centering
    \begin{tabular}{ccc|cc}
        \cmidrule{2-5}
         & \multicolumn{2}{c|}{\textbf{RML}} & \multicolumn{2}{c}{\textbf{SPARQL-Anything}}  \\ \midrule
         & \textbf{\#TM} & \textbf{\#POM} & \textbf{\#TP$^c$} & \textbf{Additional operators}  \\ \midrule
         \textbf{Standard Reification} & 9 & 20 & 25 & UNION, BIND  \\ \midrule
         \textbf{Named Graphs} & 10 & 10 & 10 & UNION, BIND, GRAPH$^c$  \\ \midrule
         \textbf{N-Ary Relationships} & 13 & 15 & 15 & UNION, BIND  \\ \midrule
         \textbf{RDF-star} & 10 & 10 & 10 & UNION, BIND  \\  \midrule
         
    \end{tabular}
\end{table}

SPARQL-Anything heavily relies on the SPARQL syntax, overriding the \texttt{SERVICE} operator while leveraging the rest of its features. This operator is then always present in the queries and hence is not shown in the table. The graphs are built with \texttt{CONSTRUCT} clauses. \cref{lst:chp6-1_sparql-anything} shows a mapping example that generates in RDF-star the annotation of the semantic types of \textit{entities} with a score from the CSV file shown in \cref{lst:chp6-1_semmeddb_entity}.
For each mapping, we show the number of triple patterns in this clause and additional SPARQL clauses (\cref{tab:chp6-1_mapping-char}). All mappings contain the same number of triple patterns in the \texttt{WHERE} clause.




\noindent\begin{captionedlisting}{lst:chp6-1_sparql-anything}{SPARQL-Anything mapping snippet to create the RDF-star graph for \textit{entity} from data in \cref{lst:chp6-1_semmeddb_entity}.  }
\centering
{
\begin{lstlisting}[basicstyle=\ttfamily\small,label={list:example1},columns=flexible,language=sparql]
CONSTRUCT {
  ?entity_id_iri :semanticType ?entity_semtype .
  << ?entity_id_iri :semanticType ?entity_semtype >> :score ?entity_score .
 } WHERE {
  {  SERVICE <x-sparql-anything:location=./data/entity.csv>
   { [] xyz:ENTITY_ID ?entity_id;
     xyz:SEMTYPE ?entity_semtype;
     xyz:SCORE ?entity_score;
    BIND(uri(concat(str("http://semmeddb.com/entity/"),
                        encode_for_uri(?entity_id))) as ?entity_id_iri)
   }  } }

\end{lstlisting}
}
\end{captionedlisting}



SPARQL queries use the \texttt{CONSTRUCT} clause to re-construct a given graph represented with one of the reifications into the other three representations. We use twelve queries to transform all pairs of representations, whose characteristics are shown in \cref{tab:chp6-1_query-char}. We show for each query the number of triple patterns in the \texttt{WHERE} and \texttt{CONSTRUCT} clauses, and the additional operators used. Except for \texttt{GRAPH}, the rest of operators only appear within the \texttt{WHERE} clause. An example of a query is shown in \cref{lst:chp6-1_sparql-construct}, which generates RDF-star from Named graphs.


\noindent\begin{captionedlisting}{lst:chp6-1_sparql-construct}{SPARQL query snippet to create the RDF-star graph from the Named Graphs representation for \textit{entity}. }
\centering
{
\begin{lstlisting}[basicstyle=\ttfamily\small,label={list:example1},columns=flexible,language=sparql]
CONSTRUCT  {
  ?entity_id_iri :semanticType ?entity_semtype .
  << ?entity_id_iri :semanticType ?entity_semtype >> :score ?entity_score .
} WHERE {
  GRAPH ?entity_graph_iri { ?entity_id_iri :semanticType ?entity_semtype }
  ?entity_graph_iri :score ?entity_score . 
}
\end{lstlisting}
}
\end{captionedlisting}



\begin{table}[t!]
\caption{Characteristics of the SPARQL queries used for the evaluation to transform the graph from a \textit{source} representation to a \textit{target} representation. \#TP stands for the number of triple patterns. Fields marked with $^c$ appear in the \texttt{CONSTRUCT} clause, while $^w$ indicates the \texttt{WHERE} clause. }
\centering
\label{tab:chp6-1_query-char}
\resizebox{\columnwidth}{!}
{\begin{tabular}{cccccccccc}
    \cmidrule{2-10}
 & \textbf{Source} & \textbf{Target} & \textbf{\#TP$^w$} & \textbf{\#TP$^c$} & \textbf{UNION$^w$} & \textbf{BIND$^w$} & \textbf{FILTER$^w$} & \textbf{VALUES$^w$} & \textbf{GRAPH}  \\ \midrule
\textbf{Q1} & \multirow{4}{*}{\textbf{N-AryRel.}} & \textbf{Std. Reif.} & 6 & 10 & \checkmark &  &  & \checkmark &  \\ \cmidrule{3-10}
 \textbf{Q2} & & \textbf{RDF-Star} & 6 & 4 & \checkmark &  &  & \checkmark &  \\ \cmidrule{3-10}
 \textbf{Q3} & & \textbf{Named Graphs} & 6 & 4 & \checkmark &  &  & \checkmark & \checkmark $^c$ \\ \midrule
\textbf{Q4} & \multirow{4}{*}{\textbf{Std. Reif.}} & \textbf{RDF-star} & 8 & 4 & \checkmark &  &  & \checkmark &  \\ \cmidrule{3-10}
 \textbf{Q5} & & \textbf{N-Ary Rel.} & 20 & 15 & \checkmark &  & \checkmark &  &  \\ \cmidrule{3-10}
 \textbf{Q6} & & \textbf{Named Graphs} & 8 & 4 & \checkmark &  &  & \checkmark & \checkmark $^c$ \\ \midrule
\textbf{Q7} & \multirow{4}{*}{\textbf{RDF-star}} & \textbf{Std. Reif.} & 5 & 25 & \checkmark & \checkmark & \checkmark &  &  \\ \cmidrule{3-10}
 \textbf{Q8} & & \textbf{N-Ary Rel.} & 5 & 15 & \checkmark & \checkmark & \checkmark &  & \\ \cmidrule{3-10}
 \textbf{Q9} & & \textbf{Named Graphs} & 5 & 10 & \checkmark & \checkmark & \checkmark &  & \checkmark $^c$ \\ \midrule
\textbf{Q10} & \multirow{4}{*}{\textbf{Named Graphs}} & \textbf{Std. Reif.} & 4 & 10 & \checkmark &  &  & \checkmark & \checkmark $^w$ \\ \cmidrule{3-10}
 \textbf{Q11} & & \textbf{RDF-Star} & 4 & 4 & \checkmark &  &  & \checkmark & \checkmark $^w$ \\ \cmidrule{3-10}
 \textbf{Q12} & & \textbf{N-Ary Rel.} & 4 & 15 & \checkmark &  &  & \checkmark & \checkmark $^w$  \\  \bottomrule
\end{tabular}}
\end{table}



\subsubsection{Engines}
\label{sec:chp6-1_engines}
We choose a set of 2 mapping processors and 3 triplestores to be representative in our evaluation, while focusing on open-source tools. The selected mapping processors, Morph-KGC (v2.5.0)~\citep{arenas2022morph} and SPARQL-Anything (v0.8.1)~\citep{asprino2023sparql-anything} are the systems with a stable version capable of producing RDF-star at the time of writing this paper. The SDM-RDFizer~\citep{iglesias2020rdfizer} is in the process of implementing this feature, but no stable version has been released yet. Regarding triplestores, we use the free version of GraphDB (v10.2.1), and the open-source Jena Fuseki (v4.8.0) and Oxigraph (v0.3.16). While GraphDB and Jena Fuseki store the graph in physical memory, Oxigraph performs the queries in memory. All engines perform duplicate removal in the results by default, with the exception of Oxigraph. For this triplestore, we add and measure a second step of duplicate removal with BASH commands.


We did attempts to include Virtuoso in the evaluation; however, several issues raised. The number of queries this triplestore can perform is very limited in this evaluation (only 4 from 12): This triplestore
does not implement RDF-star yet, and cannot produce named graphs declared inside the \texttt{CONSTRUCT} clause. In addition, Virtuoso limits the number of triples that can be produced with this clause to 1M.\footnote{\url{https://lig-membres.imag.fr/rousset/publis/tess.pdf}} When this limit is removed, an error appears that impedes the writing of the result in a file when it is large, which has been unsolved for years.\footnote{\url{https://github.com/openlink/virtuoso-opensource/issues/11}} 




\subsubsection{Experimental setup and Metrics}
\label{sec:chp6-1_exp-setup}
We perform the evaluation in two steps. First, the KG construction system evaluation is run, taking as input the SemMedDb dataset in CSV format and the mappings, and producing the corresponding RDF datasets in the four selected representations. Then, the triplestore evaluation is carried out, taking as input the produced datasets in the KG construction system evaluation, and producing RDF datasets in another representation.


We measure the \textit{materialization time} to construct the RDF graph from the input data sources in KG construction systems.
For triplestores, we measure \textit{query execution time} as the total time from query execution until the complete answer is generated. 
We also report the geometric mean of all queries that generate the same reified RDF graph, similar to previously used by~\citep{morsey2011dbpedia,schmidt2009sp}. 
The \textit{geometric mean} reports the central tendency of all execution times for a set of queries, reducing the effect of outliers. 
With this metric, we provide a general measurement of the performance of each triplestore when generating graphs in each representation. 
This allows an easier comparison with the KG construction systems. 
We run every experiment 5 times with a timeout of 24h, measure the time and calculate the median.
We run all experiments over an Ubuntu 20.04 server with
32 cores Intel(R) Xeon(R) Gold 5218R CPU @ 2.10GHz
102400 Mb RAM RDIMM, 3200 MT/s and 
100 Gb HD SSD 6 Gb/s. 




\subsection{Results}
\label{sec:chp6-1_results}


\begin{figure}[t!]
    \centering
    \includegraphics[width=\linewidth]{figures/chp6-1_results-map-queries.pdf}
    \caption{Execution time for KG construction engines with declarative mappings, and triplestores with SPARQL queries. The results for the triplestores are grouped by the \textit{target} representation.}
    \label{fig:chp6-1_map-queries}
\end{figure}




The results of the evaluation performance are shown in Figs. \ref{fig:chp6-1_map-queries} and \ref{fig:chp6-1_queries}. \cref{fig:chp6-1_map-queries} reports the comparison between knowledge graph construction systems using declarative mappings, and triplestores with SPARQL queries. For the latter, the geometric mean of the query execution time for all queries that generate each representation is provided. \cref{fig:chp6-1_queries} reports a fine-grained comparison of the performance of each triplestore on the proposed queries that perform translations between each combination of pairs of representations.

Focusing on the comparison between triplestores and KG construction systems (\cref{fig:chp6-1_map-queries}), we generally observe that for small data sizes, triplestores obtain better results, while KG construction systems scale better as data size increases. However, in the case of producing RDF-star, Fuseki and GraphDB are competitive w.r.t. SPARQL-Anything or Morph-KGC. These triplestores obtain better results for the scales 1K and 10K, and similar ones for 100K and 1M. Additionally, despite the variety in mapping characteristics, neither Morph-KGC nor SPARQL-Anything seem to be highly affected by the different representations, as opposed to the triplestores. The differences are not remarkable, but in general Morph-KGC generates the fastest Names Graphs, in contrast with SPARQL-Anything, which performs better with RDF-star. For triplestores, producing N-Ary Relationships and Standard Reification is more costly than the other representations. Another relevant point to consider is that SPARQL 1.1 does not allow \texttt{GRAPH} clause within the \texttt{CONSTRUCT} operator. Thus, only Fuseki, which implements this extension natively, can generate the Named Graphs datasets.

Comparing the behavior of engines that perform the same task, GraphDB overcomes Fuseki and Oxigraph for SPARQL queries, while Morph-KGC reports better results than SPARQL-Anything for KG construction from heterogeneous data sources, as was reported in previous works~\citep{arenas2023morphstar}.
For small data sizes, GraphDB and Fuseki perform similar, while Oxigraph reports higher query execution time. This is because Oxigraph loads the complete RDF graph in memory and the physical data structures from Fuseki and GraphDB speed up query execution time. In larger datasets, GraphDB generally scales better than Fuseki, which for example, reports a timeout for generating N-Ary Rel. in scale 1M. 

%Answering the research questions, we can say that triplestores perform faster for small sizes (\textbf{RQ1}), while KG construction systems present a more robust behaviour with increasing data size (\textbf{RQ2}). In addition, triplestores are more influenced by the changes in representations, whereas KG construction systems are mostly unaffected (\textbf{RQ3}).

 


\begin{figure}[t!]
    \centering
    \includegraphics[width=\linewidth]{figures/chp6-1_results-queries.pdf}
    \caption{Execution time of triplestores with the SPARQL queries that perform translations for pairs of representations.}
    \label{fig:chp6-1_queries}
\end{figure}


Looking into the particular differences of the pairs of translation performed with triplestores, we can observe in more detail how the different reification models affect the performance of the graph re-construction with \texttt{CONSTRUCT} queries. \cref{fig:chp6-1_queries} present the results grouping the queries in the legend by the \textit{source} reification representation (i.e. from which representation the dataset is transformed). 

Queries with RDF-star as the source representation (Q7-Q9) generally perform the worst on every scale. This set of queries includes additional SPARQL operators that are usually not required in the rest of queries: \texttt{BIND} and \texttt{FILTER}. Additionally, the recursiveness of the model, together with the fact that it is the only one that makes structural changes over RDF, can produce a negative impact on the performance of the transformation queries. Only GraphDB manages these queries better compared to the other triplestores, reporting better results, while the others reach the time limit in scale 1M. It generates the graph faster than Oxigraph and Fuseki, while it is just slightly slower compared to the queries with other source representations. This may suggest that the additional operators are not the main responsible for the increase of time processing, but the inner performance of the triplestore w.r.t. RDF-star.

The behavior of the construction of N-Ary Relationships is also remarkable. Independently of the source model, it is more costly to produce it than the other models. Queries Q5 and Q12 report an out-of-memory error in Oxigraph for scales 100K and 1M, while reaching timeout in Fuseki and GraphDB in scale 1M. Query Q9 is more affected by the source representation, RDF-star, than the target, thus the result for this query is similar to the ones affected by the abovementioned issue of this model as source. 
Only in GraphDB, Q9 reports less time to produce N-Ary Relationship than Standard Reification, which may be due to having fewer triple patterns to construct (see \cref{tab:chp6-1_query-char}). Interestingly, query Q5 involves an additional operator (\texttt{FILTER}) and more triple patterns than Q4 and Q6, which may explain its low performance. However, query Q12 does not involve any additional operators w.r.t. Q10 and Q11, and has only a few more triple patterns, which does not seem enough justification for its poor performance. Looking at the queries, both Q5 and Q12 contain five joins. Their presence is more likely to be the reason why they take up to two magnitude orders more in the result time than the queries that share source representation. 


%%%%% This one is smaller but visually I think the other (below) is more understandable
%\begin{table}[t!]
%\caption{Geometric mean of the query result times (s) from all triplestores grouping by \textit{source} representation and \textit{target} representation. The lowest times are highlighted in \textbf{bold}, while the highest are %\underline{underlined}.}
%\centering
%\label{tab:closeup-queries}
%\resizebox{\columnwidth}{!}
%{\begin{tabular}{ccccc|cccc}
%    \cmidrule{2-9}
%    & \multicolumn{4}{c|}{\textbf{Source}} & \multicolumn{4}{c}{\textbf{Target}} \\ \midrule
%    \multicolumn{1}{c|}{\textbf{Scale}} & \textbf{Std. Reif.} & \textbf{RDF-star}  & \textbf{N-Graphs} & \textbf{N-Ary Rel.} & \textbf{Std. Reif.} & \textbf{RDF-star}  & \textbf{N-Graphs} & \textbf{N-Ary Rel.} \\ \midrule
%    \multicolumn{1}{c|}{1K} & 0.283 & \underline{0.952} & 0.260 & \textbf{0.204} & 0.387 & \textbf{0.199} & 0.303 & \underline{0.526} \\ \midrule
%    \multicolumn{1}{c|}{10K} & 4.126 & \underline{40.981} & 3.395 & \textbf{1.505} & 5.200 & \textbf{1.366} & 4.693 & \underline{21.218} \\ \midrule
%    \multicolumn{1}{c|}{100K} & 56.612 & \underline{2449.544} & 43.756 & \textbf{16.040} & 154.728 & \textbf{14.066} & 106.176 & \underline{1078.092} \\ \midrule
%    \multicolumn{1}{c|}{1M} & 1085.228 & \underline{15738.107} & 773.737 & \textbf{205.407} & 954.710 & \textbf{180.384} & 927.741 & \underline{28788.048} \\   \bottomrule
%    \end{tabular}}
%\end{table}


%%%% transposed table, occupies more space
\begin{table}[t!]
\caption{Geometric mean of the query result times (s) from all triplestores grouping by \textit{source} representation and \textit{target} representation. The lowest times are highlighted in \textbf{bold}, while the highest are \underline{underlined}.}
\centering
\label{tab:chp6-1_closeup-queries}
\begin{tabular}{cccccc}
    \cmidrule{2-6}
 & \textbf{Representation} & \textbf{1K} & \textbf{10K} & \textbf{100K} & \textbf{1M} \\ \midrule
 \multirow{5}{*}{\textbf{Source}} & \textbf{Std. Reif.} & 0.283 & 4.126 & 56.612 & 1085.228  \\ \cmidrule{2-6}
  & \textbf{RDF-Star} & \underline{0.952} & \underline{40.981} & \underline{2449.544} & \underline{15738.107}  \\ \cmidrule{2-6}
  & \textbf{Named Graphs} & 0.260 & 3.395 & 43.756 & 773.737 \\ \cmidrule{2-6}
 & \textbf{N-Ary Rel.} & \textbf{0.204} & \textbf{1.505} & \textbf{16.040} & \textbf{205.407}  \\ \midrule
 \multirow{5}{*}{\textbf{Target}} & \textbf{Std. Reif.} & 0.387 & 5.200 & 154.728 & 954.710  \\ \cmidrule{2-6}
  & \textbf{RDF-Star} & \textbf{0.199} & \textbf{1.366} & \textbf{14.066} & \textbf{180.384} \\ \cmidrule{2-6}
  & \textbf{Named Graphs}  & 0.303 & 4.693 & 106.176 & 927.741 \\ \cmidrule{2-6}
 \ & \textbf{N-Ary Rel.} & \underline{0.526} & \underline{21.218} & \underline{1078.092} & \underline{28788.048} \\   \bottomrule
\end{tabular}
\end{table}

\cref{tab:chp6-1_closeup-queries} shows a summary of the results from the triplestores grouped by \textit{source} and \textit{target} representation. In general, RDF-star is the fastest representation to construct, while becoming highly inefficient when acting as the source representation. Hence, most of the triplestores do not seem yet optimized to process it. On the contrary, N-Ary relationships perform the fastest being the source representation, but it requires performing joins in the \texttt{CONSTRUCT} that make it the least suitable as a target representation. Standard Reification and Named Graphs perform more consistently overall. Named Graphs take less time than Standard Reification as source and target representation, but can only be generated with Fuseki. 

%Answering \textbf{RQ3}, the combination of representations highly influences the behavior of the triplestores, both in the \texttt{WHERE} and \texttt{CONSTRUCT} clauses. It is especially remarkable for RDF-star and N-Ary Relationships. RDF-star is the fastest representation to generate, but performs the slowest being the source representation; while N-Ary Relationships present the exact opposite behavior. Named Graphs and Standard Reification report a more consistent behavior.




\subsection{Discussion}
\label{sec:chp6-1_discussion}

This section addresses the following question: given a knowledge graph where reificaiton is needed, is it faster to construct it from the original data sources with the new representation, or to re-construct it within a triplestore?\ana{ojo esta frase, cuadrar con objetivos/RQ cuando estén } To answer this question, we perform an empirical analysis using two KG construction engines and three triplestores, performing the re-construction with four data sizes in four different reification models: RDF-star, Standard Reification, N-Ary Relationships and Named Graphs. Our results show that KG construction engines in general are more scalable, almost independently of the kind of reification. Triplestores perform best for small data sizes. However, their performance is highly dependent on the source representation (in the \texttt{WHERE} clause) and the target representation (in the \texttt{CONSTRUCT} clause). The time for producing RDF-star and Named Graphs is competitive with the KG construction approach. Yet, this approach becomes inefficient for producing N-Ary Relationships due to the need of introducing more joins in the \texttt{CONSTRUCT} clause; and when RDF-star is the source model, since most triplestores are not optimized for its new syntax. 

%%%%%%%%%%%%% SUMMARY AND MAIN CONCLUSIONS %%%%%%%%%%%%
Our evaluation shows that the KG construction engines are more robust for increasing data size and changing the reification models. Performing the re-construction in the triplestore is suitable for small data sizes, but it is largely dependant on the KG structure to offer competitive performance. Thus, we can affirm that performing the re-construction with mappings is in general more reliable, as it is significantly less affected by data size and target representation. 


%%%%%% SPARQL CONSTRUCT rant %%%%%%%
The setup of the evaluation shows us the importance of optimizing SPARQL queries, which gains relevance as the triples to construct increase in number. For instance, the introduction of \texttt{UNION} clauses was needed to avoid costly cartesian products that prevented queries from finishing before the established timeout, or before reaching an out-of-memory error. While this is well known in proficient SPARQL practitioners, it does not come as easily for non-expert users. The performance of SPARQL-Anything, relying almost entirely on SPARQL and Jena processing, is also affected by these different manners of writing the mapping. In contrast, RML mappings are unaffected in this aspect, since how the user writes the mapping does not influence the performance of the compliant systems. SPARQL possess a flexibility and rich expressiveness that languages such as RML lack yet. Nevertheless, it poses the risk for non-expert users to hamper the result retrieval incurring in suboptimal queries. This opens up a challenge for improving the query processing, or even rewriting the queries automatically to be optimized, so as to reduce this accessibility gap for SPARQL.

In addition, this evaluation brings to light that the behavior of the \texttt{CONSTRUCT} clause is not as studied as other SPARQL operators. SPARQL benchmarks often overlook this clause, as it is considered as an extension of \texttt{SELECT}~\citep{schmidt2009sp}. Hence, it is assumed that their performance is comparable and not affected by the change of clause. However, we encountered \texttt{SELECT} queries that returned results in miliseconds, while taking several minutes with \texttt{CONSTRUCT}. 
%% support of GRAPH within CONSTRUCT
%https://stackoverflow.com/questions/59397371/sparql-converting-construct-query-w-named-graph-to-select-query 
This clause also limits the kinds of transformations. For instance, SPARQL 1.1 does not implement generating named graphs (with \texttt{GRAPH}) in \texttt{CONSTRUCT}~\citep{SPARQL}.
%\footnote{\url{https://stackoverflow.com/questions/59397371/sparql-converting-construct-query-w-named-graph-to-select-query}}


%%%%%%%%%%% LIMITS OF THE ENGINES %%%%%%%%%%%%
The reported results also open up new challenges for improvement. From the triplestores, only GraphDB could perform reasonably well when re-constructing from RDF-star. This highlights the need to continue to improve the processing of this new representation, as it is currently being included in the future RDF 1.2 specification~\citep{hartig2023rdf12}. Yet, this representation obtains overall better results when it comes to constructing it. This will facilitate its adoption, since evolving current KGs into this representation would not suppose an obstacle in terms of performance. 
In addition, all triplestores struggle when the \texttt{CONSTRUCT} clause includes several joins. This fact does not only affect the pairs of transformations studied in this paper, but all potential transformations for evolving knowledge graphs. Regarding KG construction systems, the main challenge still consists of their adoption, as the learning curve for mapping languages is still steep despite efforts to lower it (see \cref{chapter:creation}).


\section{Conclusions}

\chapter{Conclusions and Future Work \textcolor{red}{By 16/2}}
\label{chapter:conclusions}

\section{Achievements}
\label{sec:chp7-1_achievements}


\section{Future Work}
\label{sec:chp7-2_future_work}

\appendix
\include{appendix/appendix1}






% --------------------------------------------------------------
%:                  BACK MATTER: appendices, refs,..
% --------------------------------------------------------------

% the back matter: appendix and references close the thesis


%\include{appendix/questions}
%\include{appendix/wicusontos}
%\include{appendix/annotations}
%\include{appendix/files}
%\include{appendix/terminology}

%: ----------------------- bibliography ------------------------

% The section below defines how references are listed and formatted
% The default below is 2 columns, small font, complete author names.
% Entries are also linked back to the page number in the text and to external URL if provided in the BibTex file.

% PhDbiblio-url2 = names small caps, title bold & hyperlinked, link to page 
%\begin{multicols}{2} % \begin{multicols}{ # columns}[ header text][ space]
%\begin{tiny} % tiny(5) < scriptsize(7) < footnotesize(8) < small (9)

%\bibliographystyle{Latex/Classes/PhDbiblio-case} % Title is link if provided
\bibliographystyle{Latex/Classes/jmb}
%\bibliographystyle{plainnat}
%\renewcommand{\bibname}{Bibliography} % changes the header; default: Bibliography
\bibliography{bibliography/bibliography} % adjust this to fit your BibTex file

%\end{tiny}
%\end{multicols}



% --------------------------------------------------------------
% Various bibliography styles exit. Replace above style as desired.

% in-text refs: (1) (1; 2)
% ref list: alphabetical; author(s) in small caps; initials last name; page(s)
%\bibliographystyle{Latex/Classes/PhDbiblio-case} % title forced lower case
%\bibliographystyle{Latex/Classes/PhDbiblio-bold} % title as in bibtex but bold
%\bibliographystyle{Latex/Classes/PhDbiblio-url} % bold + www link if provided

%\bibliographystyle{Latex/Classes/jmb} % calls style file jmb.bst
% in-text refs: author (year) without brackets
% ref list: alphabetical; author(s) in normal font; last name, initials; page(s)

%\bibliographystyle{plainnat} % calls style file plainnat.bst
% in-text refs: author (year) without brackets
% (this works with package natbib)


% --------------------------------------------------------------
 
% according to Dresden med fac summary has to be at the end
%\include{0_frontmatter/abstract}

%: Declaration of originality
%\include{12_bibliography/declaration}



\end{document}