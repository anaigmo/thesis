\chapter{Objectives and Contributions}
\label{chapter:objectives}

This chapter presents the main objectives of this thesis and identifies the contributions to the state of the art. We enumerate the assumptions considered in this work, describe the main hypotheses and delimit the scope of the thesis describing the restrictions. \cref{fig:chp3_summary} provides an overview of the objectives and contributions presented in this thesis, along with their relations with identified hypotheses, assumptions and restrictions. 

\section{Objectives}
\label{sec:chp3-objectives}

The general objective of this thesis is to \textit{improve the mapping languages regarding expressiveness, usage and adoption to comply with the current evolving needs in knowledge graph construction}. In order to achieve this goal, the following sub-objectives are defined:

\begin{enumerate}
    %\item[\textbf{O1.}] To analyse, understand and gather the capabilities of the mapping languages for KG construction from heterogeneous data sources.
    \item[\textbf{O1}] To understand and gather the necessities for knowledge graph construction from heterogeneous data sources, that will support the evolution of mapping languages to address them. 
    \item[\textbf{O2}] To help knowledge engineers and domain experts to build mapping documents providing the means for a user-friendly experience.
    \item[\textbf{O3}] To assess the value of mapping technologies for supporting the evolution of knowledge graphs. 
\end{enumerate}

In order to achieve the first objective, the following open research problem must be solved:
\begin{itemize}
    \item Ever since the first mapping languages began to be developed, several of them have been proposed and released over the years. These languages share address different necessities and propose diverse features, but they share some characteristics, as in the end they all serve the same purpose of constructing RDF graphs. In order to keep improving the KG construction from heterogeneous sources, we need to understand which are the needs and the current challenges to build them. There is a lack of analysis of the capabilities and expressiveness of these mapping languages, as well as a formalization of their derived requirements. 
\end{itemize}

From a technological perspective, the following problem must be solved:
\begin{itemize}
    \item The needs for KG construction are not static, as they evolve with the use cases and evolution of surrounding technology. For instance, this is the case of the currently under development new specification of RDF 1.2, that includes RDF-star triples. Mapping languages so far cannot produce RDF-star graphs without a dedicated extension, thus new technological support is needed to assist the creation of KGs with the evolving needs, and more specifically, to build RDF-star graphs. 
\end{itemize}

In order to achieve the second objective, the following research problem must be solved:
\begin{itemize}
    \item Most mapping languages are formalized as ontologies, and they are required to be written in an RDF syntax, usually Turtle (TTL). Other common languages extend SPARQL to write the transformation rules. This supposes a double obstacle for new users, to learn the language and the language's syntax (i.e. TTL or SPARQL) in case they are not familiar with them. There is a lack of user-friendly environments that can support the writing process for both expert practitioners and new users to speed-up the process and become less error-prone, while being up-to-date to the current KGC needs. 
\end{itemize}

From a technological perspective, the following problem must be solved:
\begin{itemize}

    \item User-friendly approaches usually involve developing new syntaxes or approaches that are not directly compliant with KGC systems, what ultimately process mappings and data to produce graphs. In addition, different KGC systems are usually compliant with one single language. There is currently little interoperability among the existing languages, which limits the possibilities for users to use a wider variety of systems with different capabilities. Hence, new technological support is needed to allow translations among current mapping languages to improve the interoperability among them. 

\end{itemize}

Finally, for achieving the third objective, the following open research problem must be solved:

\begin{itemize}
    \item Knowledge graphs are modeled according to schemes that are subject to changes over time. These modifications can be triggered by different factors, such as new requirements or error fixing, that can help improve the quality of the KG and the performance of downstream tasks that consume it. Graph re-construction can be achieved in different manners, but it is not known which approach perform best this task in different situations. Thus, there is a lack of research on how mapping languages and compliant technology can assist this task to improve KG evolution. 
\end{itemize}

\section{Contributions}
\label{sec:chp3-contributions}

In this work, we describe the solutions corresponding to the objectives and open research problems described in \cref{sec:chp3-objectives}. We present as follows the contributions that support the advance of the current state of the art regarding the first objective (understand and gather the needs for KG construction): 

\begin{enumerate}
    \item[\textbf{C1}] Design of a \textbf{comparison framework analysing the state-of-the-art mapping languages} proposed to generate knowledge graphs from heterogeneous data sources. We extract the characteristics of these languages and compare them over a set of detailed features that show their expressiveness. 
    
    \item[\textbf{C2}] \textbf{Identification, definition and formal implementation of requirements for knowledge graph construction} from heterogeneous data sources. Based on the analysis performed on the comparison framework, we are able to extract what are the needs for building knowledge graphs, which are possible to achieve with the current languages and which need to be addressed yet. We implement these requirements in a formal language. 
    
    \item[\textbf{C3}] \textbf{Development of new features for the RML mapping language} to address the limitations of the language according to the new needs in knowledge graph construction identified. We focus on the development of the extension to create RDF-star graphs. 
\end{enumerate}

Regarding the second objective (to help knowledge engineers and domain experts to build mappings), we present new advances with the following contributions:

\begin{enumerate}
    \item[\textbf{C4}] \textbf{Design of a user-friendly approach for writing mapping rules based on spreadsheets}. We propose this approach as a way of writing mappings by only specifying the essential information, erasing the need of learning the language constructs and the syntax's peculiarities. We also develop an implementation of this approach to translate the mapping rules in spreadsheets into different mapping languages. 
    \item[\textbf{C5}] \textbf{Update of the YARRRML syntax for RML with new features}. We include in a new version of this syntax the features incorporated in the evolucion of the RML mapping language, to keep this language accessible for a broader community of users that already use YARRRML. We also provide an implementation to translate between the updated YARRRML version into other mapping languages to facilitate its adoption.
\end{enumerate}

Finally, with regard to the third objective (to assess the role of mapping technologies in KG evolution), this work presents the following contribution:

\begin{enumerate}
    \item[\textbf{C6}]\textbf{ Analysis of the scenarios in which mapping compliant systems can play a valuable role in the re-construction of pre-existing graphs}. We conduct an evaluation to study how mapping technologies can not only be useful for KG construction, but also for enhancing the evolution of knowledge graphs that need a change in their schema. 
\end{enumerate}


\section{Assumptions}
\label{sec:chp3-assumptions}
Our work is based on the assumptions listed below. These assumptions provide a background to facilitate the comprehension of the decisions taken during the development of this work. 


\begin{enumerate}
    \item[\textbf{A1}] Mapping languages provide human readable documentation available online.
    \item[\textbf{A2}] Mapping languages are declarative and follow W3C standards.
    \item[\textbf{A3}] Features characteristic of procedural languages are not requirements for KG construction.
    \item[\textbf{A4}] The schema of knowledge graphs (ontology) used for creating mapping documents is available online, and implemented in OWL or RDF(S). 
    \item[\textbf{A5}] Mapping rules can be translated into different languages with information preservation.
    %\item[\textbf{moar?}] KGs refer to RDF graphs?
\end{enumerate}


\section{Hypotheses}
\label{sec:chp3-hypotheses}

After the identification of the assumptions, we can describe the research hypotheses  of this thesis. They cover the general characteristics of the contributions:

\begin{enumerate}
    \item[\textbf{H1}] Current mapping languages do not cover the entire extension of features required to construct knowledge graphs from heterogeneous data sources.
    \item[\textbf{H2}] It is possible to include in current mapping languages new features that address the evolving needs in construction of knowledge graphs.
    \item[\textbf{H3}] Writing the mapping rules in spreadsheet environments can improve the user experience for practitioners of different backgrounds for writing mappings reducing errors. 
    \item[\textbf{H4}] Mapping-compliant technologies can bring benefits in the evolution of knowledge graphs within their life cycle, not only in their construction.
\end{enumerate}


\section{Restrictions}
\label{sec:chp3-restrictions}

Finally, there is a set of restrictions that describe the limitations and define the future work objectives:

\begin{enumerate}
    \item[\textbf{R1}] Requirements for KG construction are considered up to the beginning of 2023.
    \item[\textbf{R2}] The point of reference for the RML specification is the release on 2014~\cite{Dimou2014rml}.
    \item[\textbf{R3}] Features characteristic of procedural languages cannot be modelled in an ontology.
    \item[\textbf{R4}] Our proposal does not include all modules of the RML release on 2023~\cite{iglesias2023rml}.
    \item[\textbf{R5}] The evaluation of the value of mappings in KG evolution considers only schema changes, not data changes.
    \item[\textbf{R6}] Changes considered in KG evolution are a result of the schema changes for switching among reification approaches for RDF graphs.
    %\item[\textbf{moar?}] algo de la selección de lenguajes? algo para mapeathor y yarrrml? The spreadsheet template consider the features of RML v2014 and the current RML-FNML spec. The YARRRML updates include all RMLv2 modules except for the RML-CC module.
    \end{enumerate}


\begin{sidewaysfigure}[t!]
    \centering
    \includegraphics[width=1\linewidth]{figures/chp3_summary.pdf}
    \caption[Relations between objectives, contributions, hypotheses, assumptions and restrictions of this thesis]{Overview of objectives, contributions, hypotheses, assumptions and restrictions, and their relations.}
    \label{fig:chp3_summary}
\end{sidewaysfigure}

\section{Research Methodology?}

