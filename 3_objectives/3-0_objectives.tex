\chapter{Objectives and Contributions}
\label{chapter:objectives}

This chapter presents the main objectives of this thesis and identifies the contributions to the state of the art. We enumerate the assumptions considered in this work, describe the main hypothesis and delimit the scope of the thesis describing the restrictions.

\section{Objectives}
\label{sec:chp3-objectives}

The general objective of this thesis is to \textit{improve the mapping languages regarding expressiveness and usage to comply with the current evolving needs in knowledge graph construction}. In order to achieve this foal, the following sub-objectives are defined:

\begin{enumerate}
    %\item[\textbf{O1.}] To analyse, understand and gather the capabilities of the mapping languages for KG construction from heterogeneous data sources.
    \item[\textbf{O1.}] To understand and gather the necessities for knowledge graph construction from heterogeneous data sources.
    \item[\textbf{O2.}] To help knowledge engineers and domain experts to build mapping documents providing the means for a user-friendly experience.
    \item[\textbf{O3.}] To assess the role of mapping technologies for improving the evolution of knowledge graphs. 
\end{enumerate}

In order to achieve the first objective, the following open research problem must be solved:

\section{Contributions}
\label{sec:chp3-contributions}

In this work, we describe the solutions corresponding to the objectives and open research problems described in \cref{sec:chp3-objectives}. We present as follows the contributions that support the advance of the current state of the art regarding the first objective (understand and gather the needs for KG construction):

\begin{enumerate}
    \item[\textbf{C1.1.}] Comparison framework analysing the sota mapping languages
    \item[\textbf{C1.2.}] Identification and definition of requirements for KGC 
    \item[\textbf{C1.3.}] Implementation in formal language these requirements
    \item[\textbf{C1.4.}] Uptake of requirements and new needs in KGC into RML, focus on RML-star
\end{enumerate}

Regarding the second objective (to help knowledge engineers and domain experts to build mappings), we present new advances with the following contributions:

\begin{enumerate}
    \item[\textbf{C2.1.}] Proposal of spec of mapping rules in spreadsheets
    \item[\textbf{C2.2.}] Update of YARRRML w.r.t. updates in RML 
\end{enumerate}

Finally, with regard to the third objective (to assess the role of mapping technologies in KG evolution), this work presents the following contribution:

\begin{enumerate}
    \item[\textbf{C3.1.}] Identification of situations where mapping-compliant technologies work well with regards to other usual technologies involved in the evolution of knowledge graphs.  where they fail so it can be improved?
\end{enumerate}


\section{Assumptions}
\label{sec:chp3-assumptions}
Our work is based on the assumptions listed below. These assumptions provide a background to facilitate the comprehension of the decisions taken during the development of this work. 


\begin{enumerate}
    \item[\textbf{A1}] Mapping languages have human readable documentation
    \item[\textbf{A2}] The target schema (ontology) for creating the mapping documents is available and implemented in OWL or RDFS. 
\end{enumerate}


\section{Hypothesis}
\label{sec:chp3-hypothesis}

After the identification of the assumptions, we can describe the research hypothesis  of this thesis. They cover the general characteristics of the contributions \ana{si?}

\begin{enumerate}
    \item[\textbf{H1}] The expressiveness of mapping languages can be gathered and represented in a formal language (i.e. an ontology).
    \item[\textbf{H2}] New needs in KG construction ¿and updates in the RDF schema?  can be implemented in current mapping languages
    \item[\textbf{H3}] Writing the mapping rules in spreadsheets can improve the user experience for practitioners of different backgrounds with respect to other approaches for writting mappings
    \item[\textbf{H4}] Mappings can help enhance in the evolution of KGs withing the KG lifecycle, not only in construction.
\end{enumerate}


\section{Restrictions}
\label{sec:chp3-restrictions}

Finally, there is a set of restrictions that describe the limitations and define the future work objectives.

\begin{enumerate}
    \item[\textbf{R1}] \ana{algo de la eleccion de lenguajes?}
    \item[\textbf{R2}] The requirements in KGC are considered until summer 2023, time of writing this document. The updates made between this time until its publication are not considered.
    \item[\textbf{R3}] The representation of mapping language features that provide a procedural language (loops, embedded clauses) is out of the scope
    \item[\textbf{R4}] The spreadsheet template consider the features of RML v2014 and the current RML-FNML spec
    \item[\textbf{R5}] THe YARRRML updates include all RMLv2 modules except for the RML-CC module.
    \item[\textbf{R6}] The changes considered in KG evolution are a result of the schema changes of a switch in reification approach, not all possible modifications in KGs are considered.
    \item[\textbf{R7}] The evolution considered involves only changes in the schema of the KG, the data represented in the KG is stable with no additions or deletions, just change in how it is structured.
\end{enumerate}
