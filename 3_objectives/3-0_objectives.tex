\chapter{Objectives and Contributions}
\label{chapter:objectives}

This chapter presents the main objectives of this thesis and identifies the contributions to the state of the art. We enumerate the assumptions considered in this work, describe the main hypothesis and delimit the scope of the thesis describing the restrictions.

\section{Objectives}
\label{sec:chp3-objectives}

The general objective of this thesis is to \textit{improve the mapping languages regarding expressiveness and usage to comply with the current evolving needs in knowledge graph construction}. In order to achieve this foal, the following sub-objectives are defined:

\begin{enumerate}
    %\item[\textbf{O1.}] To analyse, understand and gather the capabilities of the mapping languages for KG construction from heterogeneous data sources.
    \item[\textbf{O1.}] To understand and gather the necessities for knowledge graph construction from heterogeneous data sources.
    \item[\textbf{O2.}] To help knowledge engineers and domain experts to build mapping documents providing the means for a user-friendly experience.
    \item[\textbf{O3.}] To assess the role of mapping technologies for improving the evolution of knowledge graphs. 
\end{enumerate}

In order to achieve the first objective, the following open research problem must be solved:

\section{Contributions}
\label{sec:chp3-contributions}

In this work, we describe the solutions corresponding to the objectives and open research problems described in \cref{sec:chp3-objectives}. We present as follows the contributions that support the advance of the current state of the art regarding the first objective (understand and gather the needs for KG construction):

\begin{enumerate}
    \item[\textbf{C1.1.}] Comparison framework analysing the sota mapping languages
    \item[\textbf{C1.2.}] Identification and definition of requirements for KGC 
    \item[\textbf{C1.3.}] Implementation in formal language these requirements
    \item[\textbf{C1.4.}] Uptake of requirements and new needs in KGC into RML, focus on RML-star
\end{enumerate}

Regarding the second objective (to help knowledge engineers and domain experts to build mappings), we present new advances with the following contributions:

\begin{enumerate}
    \item[\textbf{C2.1.}] Proposal of spec of mapping rules in spreadsheets
    \item[\textbf{C2.2.}] Update of YARRRML w.r.t. updates in RML 
\end{enumerate}

Finally, with regard to the third objective (to assess the role of mapping technologies in KG evolution), this work presents the following contribution:

\begin{enumerate}
    \item[\textbf{C3.1.}] Identification of situations where mapping-compliant technologies work well with regards to other usual technologies involved in the evolution of knowledge graphs.  where they fail so it can be improved?
\end{enumerate}


\section{Assumptions}
\label{sec:chp3-assumptions}



\section{Hypothesis}
\label{sec:chp3-hypothesis}



\section{Restrictions}
\label{sec:chp3-restrictions}
